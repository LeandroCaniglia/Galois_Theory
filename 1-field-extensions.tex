\chapter{Field Extensions}

\section{Field Extensions}

Recall that a \textsl{field extension} $K/\kappa$ is a ring monomorphism $\iota_{\kappa K}\colon\kappa\to K$ for the \textsl{base field} $\kappa$ to the \textsl{extension field} $K$. The inclusion map $\iota_{\kappa K}$ is usually denoted $\iota_K$ and $\kappa$ considered a subfield of $K$. The extension is also denoted~$K/\kappa$.

Note that $K$ can also be seen as a $\kappa$-vector space. The dimension $\dim_\kappa(K)$ is denoted $[K:\kappa]$. The extension is \textsl{finite} if $[K:\kappa]<\infty$, otherwise it is called \textsl{infinite}.

\begin{xmpls}${}$
    \begin{enumerate}[a), font=\upshape]
        \item The quotient field $\kappa(x)$ of the polynomial ring $\kappa[x]$, or more generally, the field $\kappa(x_1,\dots,x_n)$ of rational functions in the indeterminates $x_1,\dots,x_n$ are examples of field extensions over~$\kappa$.

        \item The ring of generalized formal series $\kappa((x))$ whose elements are series with a finite number of negative indexes, i.e., series of the form
        $$
            f = \sum_{n=n_0}^\infty a_nx^n,
        $$
        for some $n_0\in\Z$. The inverse $g=\sum_{n=n_0}^\infty b_nx^n$. of such an element $f\ne0$ can be defined inductively as follows. Take $n_0$ such that $a_{n_0}\ne0$. There are two cases        
        \begin{description}
            \item[Case $n_0=0$:] Put $b_0=a_0^{-1}$. For $n>0$, once $b_0,\dots,b_{n-1}$ have been defined, let $b_n$ be the (only) element of $\kappa$ that satisfies
            $$
                \sum_{i=0}^na_{n-i}b_i=0.
            $$
            
            \item[General case:] By the previous case, $x^{-n_0}f$ has an inverse $g$. Therefore, $gx^{-n_0}f=1$ and so $x^{-n_0}g$ is the inverse of~$f$.
        \end{description}

        \item The extension $\C/\R$ is finite with $[\C:\R]=2$.
    \end{enumerate}
\end{xmpls}

\begin{ntns}
    Let\/ $K$ be a field extension of\/ $\kappa$. If\/ $S\subseteq K$ is a subset, the ring\/ $\kappa[S]$ generated by\/ $\kappa$ and\/ $S$ denotes the intersection of all subrings of\/ $K$ that contain\/ $\kappa$ and\/ $S$. Similarly, the field\/ $\kappa(S)$ generated by\/ $\kappa$ and\/ $S$ denotes the intersection of all subfields of\/ $K$ that contain\/ $\kappa$ and\/ $S$. If\/ $S=\set{a_1, \dots, a_n}$ is finite, we will write\/ $\kappa[S]=\kappa[a_1, \dots, a_n]$ and\/ $\kappa(S)=\kappa(a_1, \dots, a_n)$.
\end{ntns}

\begin{defn}
     An extension $K/\kappa$ is \textsl{finitely generated} if $K=\kappa(a_1,\dots,a_n)$ for some finite set $\set{a_1,\dots,a_n}$.
\end{defn}

\begin{prop}\label{prop:extension-generators}
    Let\/ $K$ be a field extension of\/ $\kappa$ and let\/ $a_1,\dots,a_n\in K$.
    Then
    $$
    \kappa[a_1,\dots,a_n]
        = \set{f(a_1,\dots,a_n)\mid f\in\kappa[x_1,\dots,x_n]}
    $$
    and
    $$
    \kappa(a_1,\dots,a_n)=\bigg\{\frac{f(a_1,\dots,a_n)}{g(a_1,\dots,a_n)}
        \mid f,g\in\kappa[x_1,\dots,x_n],\ g(a_1,\dots,a_n)\ne0\bigg\}
    $$
    so\/ $\kappa(a_1,\dots,a_n)$ is the quotient field of\/ $\kappa[a_1, \dots,a_n]$.
\end{prop}

\begin{proof}
    The result is a direct consequence of the definitions.
\end{proof}

\begin{defn}
    Given a partially ordered set $(X,\le)$, a subset $T$ is \textsl{filtered} if given $a,b\in T$, there exists $c\in T$ such that $a\le c$ and $b\le c$. This notion also applies to families $(a_i)_{i\in I}$ of elements in $X$.
\end{defn}

\begin{lem}
    Let\/ $K/\kappa$ be a field extension. If\/ $\mathcal C$ is a filtered collection of subfields of\/ $K$, its union\/ $\bigcup\mathcal C$ is also a subfield of\/~$K$.
\end{lem}

\begin{proof}
    The result is a direct consequence of the definitions.
\end{proof}

\begin{prop}\label{prop:finite-filtered-union}
    Let\/ $K$ be a field extension of\/ $\kappa$ and let\/ $S$ be a subset of\/~$K$. Then
    $$
        \kappa(S) = \bigcup
            \set{\kappa(a_1,\dots,a_n)\mid a_1,\dots, a_n\in S},
    $$
    where the union runs over all finite subsets of\/ $S$.
\end{prop}

\begin{proof}
    Given ${a_1,\dots,a_n}\subseteq S$, it is clear that $\kappa(a_1,\dots,a_n)\subseteq\kappa(S)$ because $\kappa(S)$ is a field and includes $\kappa$ and $a_1,\dots,a_n$. On the other hand, the union of all these field extensions is a field extension because the family of extensions under union is filtered.
\end{proof}

\begin{defns}
    If $K/\kappa$ is a field extension, then an element $\alpha \in K$ is \textsl{algebraic} over $\kappa$ if there is a nonzero polynomial $f(x) \in \kappa[x]$ with $f(\alpha)=0$. If $\alpha$ is not algebraic over $\kappa$, then $\alpha$ is said to be \textsl{transcendental} over $\kappa$. If every element of $K$ is algebraic over $\kappa$, then $K$ is said to be \textsl{algebraic} over $\kappa$, and $K / \kappa$ is called an \textsl{algebraic extension}.
\end{defns}

\begin{defn}
    If $\alpha$ is algebraic over a field $\kappa$, the \textsl{minimal polynomial} of $\alpha$ over $\kappa$ is the monic polynomial $f(x)$ of least degree in $\kappa[x]$ for which $f(\alpha)=0$; it is denoted $\min_{\kappa,\alpha}(x)$. Equivalently, $\min_{\kappa,\alpha}(x)$ is the monic generator of the kernel of $\ev\alpha\colon\kappa[x]\to K$.
\end{defn}

\begin{xmpl}
    The minimal polynomial of the complex number $i$ in the field extension $\Q(i)/\Q$ is $\min_{\Q,i}(x) = x^2+1$.
\end{xmpl}

\begin{prop}\label{prop:algebraic-minimal}
    Let $K/\kappa$ be a field extension and $\alpha\in K$ algebraic over $\kappa$.
    \begin{enumerate}[a), font=\upshape]
        \item $\min_{\kappa,\alpha}$ is irreducible over $\kappa$.
        
        \item If $g\in\kappa[x]$, then $g(\alpha)=0$ if, and only if, $\min_{\kappa,\alpha}\mid g$.

        \item If $n=\deg(\min_{\kappa,\alpha})$, then $\set{1,\alpha,\dots, \alpha^{n-1}}$ is a basis for $\kappa(\alpha)$ over $\kappa$, so $[\kappa(\alpha): \kappa]=\deg(\min_{\kappa,\alpha})<\infty$. Moreover, $\kappa(\alpha)=\kappa[\alpha]$.

        \item If $\zeta\in\kappa(\alpha)$ is another root of $\min_{\kappa,\alpha}$ there is a unique $\kappa$-automorphism $\varepsilon\colon\kappa(\alpha)\to\kappa(\alpha)$ such that $\varepsilon(\alpha)=\zeta$. 
    \end{enumerate}
\end{prop}

\begin{proof}${}$
    \begin{enumerate}[a), font=\upshape]
        \item The minimal polynomial is irreducible because $\kappa[x]/\gen{\min_{\kappa,\alpha}}$ is a subring of~$K$, 

        \item Trivial.

        \item Firstly $\kappa[\alpha]=\kappa[x]/\gen{\min_{\kappa,\alpha}}$. Secondly, the quotient is a field because $\gen{\min_{\kappa,\alpha}}$ is maximal.

        \item Consider the evaluations $\ev\zeta\colon\kappa[x]\to\kappa(\zeta)$ and $\ev\alpha\colon\kappa[x]\to\kappa(\alpha)$ that map, respectively, $x$ to $\zeta$ and $x$ to $\alpha$. By part~a) $\min_{\kappa,\alpha}=\min_{\kappa,\zeta}$. By part~c) $\kappa(\alpha)=\kappa(\zeta)$. We have a commutative diagram
        $$
            \begin{tikzcd}
                {\kappa[x]}
                        \arrow[r,"\ev\zeta"]
                        \arrow[d,"\ev\alpha"']
                    &\kappa(\alpha)
                    &x
                        \arrow[r,mapsto]
                        \arrow[d,mapsto]
                    &\zeta\\
                \kappa(\alpha)
                        \arrow[ru,dashed]
                    &&\alpha
                        \arrow[ru,mapsto]&
            \end{tikzcd}
        $$
    \end{enumerate}
    where the map from $\kappa(\alpha)$ to $\kappa(\alpha)$ is unique.
\end{proof}

Since we will be dealing with irreducible polynomials it is important to keep in mind the following well-known criterion

\begin{lem}\label{lem:eisenstein}{\rm[Eisenstein Criterion]}
    Let\/ $f(x) = a_n x^n + \cdots + a_0$ be a polynomial with integer coefficients. Suppose there exists a prime number\/ $p$ such that:
    \begin{enumerate}[\rm i)]
        \item $p$ divides\/ $a_i$ for all\/ $i = 0,1,\dots,n-1$,
        \item $p$ does not divide\/ $a_n$, and
        \item $p^2$ does not divide\/ $a_0$.
    \end{enumerate}
    Then\/ $f(x)$ is irreducible over\/ $\Q$.
\end{lem}

\begin{proof}
    Suppose that $f=gh$ with $g,h\in\Z[x]$ of positive degree. Consider the morphism $\Z[x]\to\Z_p[x]$ that reduces coefficients modulo~$p$. By hypothesis $\bar f=\bar a_nx^n$. Therefore, $\bar g\bar h=\bar a_nx^n$, which implies that both $\bar g$ and $\bar h$ are monomials. In particular, $p\mid g(0)$ and $p\mid h(0)$. Thus, $p^2\mid g(0)h(0)=f(0)=a_0$. Contradiction.
\end{proof}

\begin{rem}
    Condition~iii) of Eisenstein Criterion is needed. For instance $x^4+4=(x^2-2x+2)(x^2+2x+2)$.
\end{rem}

Let $A$ be a commutative ring. Recall that a polynomial in $A[x]$ is \textsl{primitive\/} when the ideal generated by its coefficients is $A$. In the case where the ring is a unique factorization domain, the \textsl{primitive part\/} of a polynomial $f$ is defined (up to multiplication by unit) as the primitive polynomial $\pp(f)$ for which there exists a constant $c$ satisfying $f=c\pp(f)$ (note that $c$ can be taken as the $\gcd$ of the coefficients of~$f$).

\begin{lem}\label{lem:pp-is-multiplicative}
    In a unique factorization domain $D$ the product of primitive polynomials in $D[x]$ is primitive. In particular, the primitive part is multiplicative, i.e., $\pp(fg)=\pp(f)\pp(g)$.
\end{lem}

\begin{proof}
    Take two primitive polynomials $f=a_nx^n+\cdots+a_0$ and $g=b_mx^m+\cdots+b_0$. Suppose that $fg$ is not primitive. Pick a prime $p\in D$ that divides all coefficients of $fg$. Since $p$ does not divide $f$ or $g$, we can take $r=\min\set{i\mid p\nmid a_i}$ and $s=\min\set{j\mid p\nmid b_j}$. The $(r+s)$th coefficient of $fg$ is the sum of products of the form $a_ib_j$ with $i+j=r+s$. Of these only one has $i\ge r$ and $j\ge s$, namely $a_rb_s$. Therefore, all terms except one are divisible by $p$. Since $p$ divides this coefficient, we arrive at a contradiction.
\end{proof}

\begin{lem}\label{lem:gauss-irreducibility}{\rm[Gauss]}
    Let\/ $D$ be a unique factorization domain and\/ $Q$ its field of fractions. Then, a nonconstant polynomial in\/ $D[x]$ is irreducible if, and only if, it is both irreducible in\/ $Q[x]$ and primitive in\/ $D[x]$.
\end{lem}

\begin{proof}
    Put $f(x) = a_nx^n+\cdots+a_1x+a_0$.
    
    \begin{description}
        \item[\rm\it only if:] If $f$ is irreducible in $D[x]$ it must be primitive; otherwise, it would have a constant factor. Moreover, if $f=gh$ in $Q[x]$, we can take $a,b\in D$, such that $ag$ and $bh$ belong in $D[x]$. Taking primitive parts on both sides of the equation $abf=(ag)(bh)$, we get a factorization of $f$ in $D[x]$, which must be trivial (i.e., one of the factors must be a unit). It follows that the original factorization is trivial.

        \item[\rm{\it if\/} part:] Suppose that $f$ is primitive and irreducible in $Q[x]$. If $f=gh$ in $D[x]$ then $g$ or $h$ must be a constant. Moreover, since $f$ is primitive such a constant must be a unit in~$D$.
    \end{description}
\end{proof}

\begin{prop}\label{prop:primitive+irreducible=irreducible}
    Let\/ $\kappa$ be a field and\/ $\kappa[x,y]$ the ring of polynomials in two variables. If\/ $f$ is primitive in\/ $\kappa[x][y]$ and irreducible in\/ $\kappa(x)[y]$, then it is irreducible in\/ $\kappa(y)[x]$.
\end{prop}

\begin{proof}
    Suppose that $f=hk$ in $\kappa(y)[x]$. We have to prove that one of the factors belongs in $\kappa(y)$. Pick $a,b\in\kappa[y]$ such that $abf=(ah)(bk)$ is a factorization in $\kappa[y][x]$. Taking primitive parts, we get 
    $$
        \pp_y(f)=\pp_y(abf)
            =\pp_y(ah)\pp_y(bk),\quad\text{in }\kappa[x,y].
    $$
    Since $\pp_y(f)$ is a factor of $f$ in $\kappa[x,y]$, it must be irreducible in $\kappa(x)[y]$. Thus, without loss of generality, we may assume that $\pp_y(ah)$ is constant in $\kappa[x][y]$, i.e., $\pp_y(ah)\in\kappa[x]$. Given that $f$ is primitive in $\kappa[x][y]$ and $\pp_y(ah)$ is a factor of $f$, we must have $\pp_y(ah)\in\kappa$. Therefore, $h\in\kappa$.
\end{proof}

\begin{xmpl}\label{xmpl:k(t)/k(u)}
    Let $\kappa$ be a field and $K=\kappa(t)$ the field of rational functions. Let $u \in K\setminus\kappa$. Write $u=f(t) / g(t)$ with $f, g \in \kappa[\,t\,]$ and $f(t)\perp g(t)$, and let $F=\kappa(u)$. We claim that
    $$
    [K: F]=\max\set{\deg f(t), \deg g(t)}.
    $$

    Since $\kappa(t)\subseteq F(t)\subseteq\kappa(t)$, both fields are equal, i.e., $K=F(t)$. To compute the minimal $\min_{F,t}\in F[x]$ consider $p(x)=ug(x)-f(x)$. Clearly $t$ is a root of $p$ and so $t$ is algebraic over $F$. It follows that
    $$
        [K:F]=[F(t):F]\le\deg p=\max\set{\deg f,\deg g}.
    $$
    Note that $u$ is not algebraic over $\kappa$. Otherwise, any algebraic equation on~$u$ would contradict the fact that $f\perp g$. Indeed
    $$
        \sum_{i=0}^db_iu^i=0
            \iff \sum_{i=0}^db_if^ig^{d-i}=0
            \iff b_0g^d = -\sum_{i=1}^db_if^ig^{d-i}
            \implies f\mid g^d.
    $$
    Therefore, $u$ is transcendental over $\kappa$, hence, over $\kappa(t)$. Since $f\perp g$, $p$ is primitive over $\kappa[x][u]$ and so irreducible in $\kappa[x][u]$. It follows from the previous proposition that $p$ is irreducible in $\kappa(u)[x]=F[x]$ as well. Thus, $p=\min_{F,t}$, proving the claim.
\end{xmpl}


\begin{xmpl}\label{xmpl:sqrt3 x-has-degree-3}
    Let $\kappa=\F_2(x)$ and $K=\kappa(\sqrt[3]x)$, where $\sqrt[3]x$ denotes one root of $t^3-x$, is algebraic. We claim that $[K:\kappa]=3$ because $t^3-x$ is primitive in $\F_2[t][x]$ and irreducible in $\F_2(t)[x]$, hence irreducible in $\F_2(x)[t]=\kappa[t]$.

    Let's now consider $F=\kappa(\sqrt x)$. We claim that $[F(\sqrt[3]x):F]=3$. To see this put $\alpha=\sqrt[3]x$ and observe that $t^3-x=(t-\alpha)(t^2+\alpha t+\alpha^2)$ cannot be decomposed in $F[t]$ because $\alpha\notin\kappa$.
\end{xmpl}


\begin{lem}\label{lem:euler} {\rm[Euler]}
    There do not exist four rational numbers whose squares are in arithmetic progression.
\end{lem}

\begin{proof}
    Suppose, for a contradiction, that $a$, $b$, $c$, $d$ are rational numbers such that $b^2-a^2=c^2-b^2=d^2-c^2$ and let's denote by $k$ this common value. After multiplying by all denominators we reduce ourselves to the case where the five quantities involved are integers. Suppose that $p$ is a prime that divides $a$ and $b$, then $p\mid k$ and therefore, it divides $c$ and finally $d$. Hence, we may assume that $a\perp b$. Now take $p^n\mid\gcd(a+b,a-b)$. Then $p^n\mid2a$ and $p^n\mid2b$. Thus, $p^n=1$, or $p^n=2$. We claim that the latter equation holds. To see this assume that $a$ and $b$ have different parity. Then $k=b^2-a^2$ is odd. Then $c$ and $b$ have different parity and so $a$ and $c$ are both odd or both even. Since $(c+a)(c-a)=c^2-a^2=2k$, it follows that $k$ is even and so $a$ and $b$ have the same parity. Recalling that $a\perp b$, we deduce that $a$ and $b$ are both odd and satisfy $\gcd(a+b,a-b)=2$. Similar conclusions hold for pairs $(b,c)$ and $(c,d)$.

    Using that $a\equiv c\equiv1\pmod2$, we can write $c+a=2u$ and $c-a=2v$. Solving for $c$ and $a$, we obtain $c=u+v$ and $a=u-v$. Hence,
    $$
        2k=c^2-a^2=(c+a)(c-a)=4uv,
    $$
    i.e., $k = 2uv$. Then,
    $$
        b^2 = c^2-2uv = (u+v)^2-2uv = u^2+v^2.
    $$
    In particular $u\perp v$. Note moreover that $u$ and $v$ have different parity, so we may assume $u$ odd and $v$ even.

    Write $d^2-b^2=2k=4uv$, i.e.,
    $$
        (d+b)/2(d-b)/2=uv,\quad\text{with }(d+b)/2\perp(d-b)/2\text{ and }u\perp v.
    $$
    Split $u=AB$ and $v=CD$ where $A$ is the part of $u$ that divides $(d+b)/2$ and $C$ is the part of $v$ that divides $(d+b)/2$. We have
    $$
        d+b = 2AC,\quad d-b = 2BD,\quad\set{A,B,C,D} \text{ pairwise coprime}.
    $$
    In particular, $b=AC-BD$. After substituting in the equation $b^2=u^2+v^2$, we get
    $$
        (AC-BD)^2 = (AB)^2 + (CD)^2.
    $$
    Since $v$ is even, one of $C$ and $D$ is even. After replacing, if necessary, $b$ with $-b$, we may assume that $C$ is even. Note that $A$, $B$ and $D$ are odd. Rewriting the last equation we obtain
    $$
        (A^2-D^2)C^2 - 2ABDC + B^2D^2-A^2B^2 = 0,
    $$
    which, as quadratic equation in $C$ has discriminant
    \begin{align*}
        \Delta &= 4(A^2B^2D^2-(A^2-D^2)(B^2D^2-A^2B^2))\\
            &= 4(\cancel{A^2B^2D^2}-\cancel{A^2B^2D^2}+A^4B^2+B^2D^4-A^2B^2D^2)\\
            &= 4B^2(A^4+D^4-A^2D^2).
    \end{align*}
    In consequence, $\sqrt{\vphantom{A^|}A^4+D^4-A^2D^2}\in\Q$, which implies that
    $$
        A^4+D^4-A^2D^2=m^2
    $$
    for some $m\in\Z$, $m$ odd.

    Given that $A$ and $D$ are odd, we can write $A^2=U+V$ and $D^2=U-V$, for $U\perp V$. Therefore, the last equation leads to
    \begin{align*}
        m^2 &= (U+V)^2 + (U-V)^2 - (U^2-V^2)\\
            &= U^2+V^2 + \cancel{2UV} + U^2 + V^2 - \cancel{2UV} - U^2 + V^2\\
            &= U^2+3V^2,
    \end{align*}
    which implies that one, and only one, of $U$ and $V$ must be odd. Interchanging $U$ with $V$, if needed, we may assume that $U$ is odd and $V$ even. Rewriting the equation as $3V^2=(m+U)(m-U)$ we see that we may assume that $3\mid m+U$, if necessary, by replacing $U$ with $-U$. Thus, we get
    $$
        3V^2 = (m+U)/2(m-U)/2,
    $$
    with no prime $p\ne3$ common to $(m+U)/2$ and $(m-U)/2$. In consequence,
    $$
        (m+U)/2 = 3R^2,\quad(m-U)/2 = S^2
    $$
    for some $R,S\in\Z$, with no common prime factor $p\ne3$. We obtain,
    $$
        m = 3R^2+S^2,\quad U = 3R^2-S^2,\quad V = \pm2RS.
    $$
    Therefore,
    \begin{align*}
        A^2 &= U+V = 3R^2-S^2 \pm2RS = (R\pm S)(3R\mp S)\\
        D^2 &= U-V = 3R^2-S^2 \mp2RS = (R\mp S)(3R\pm S).
    \end{align*}
    Since $A\perp D$, we see that the factors on the right hand sides are coprime. Indeed, a prime dividing both factors cannot be $2$ because $A$ and $D$ are odd. Moreover, any other prime dividing both factors would divide $R$ and $S$, which would then be a common factor of $A$ and $D$.

    It follows that
    $$
        S-3R,\quad S-R,\quad S+R,\quad S+3R  
    $$
    must all be $\pm$ odd squares in arithmetic progression of $2R$. We claim that they all have the same sign. To see this, let $q_1^2,q_2^3,q_3^2,q_4^2$ denote the four (odd) squares. Suppose that the sign remains the same from index $i$ to index $i+1$. Then
    $$
        (q_{i+1}+q_i)(q_{i+1}-q_i)=q_{i+1}^2-q_i^2=\pm2R,
    $$
    which implies that $R$ is even. Since the sign must change from $q_1^2$ to $q_4^2$, we have
    $$
        (q_4-q_1)(q_4+q_1) = q_4^2-q_1^2 = \pm2S,
    $$
    which shows that $S$ is even too. But this is impossible because $S$ and $R$ have no common prime factor $p\ne3$.

    It follows that we have four squares in arithmetic progression of $2R$. Since $S-3R$ and $S+3R$ have the same sign, we must have $3|R|<|S|$. In consequence $9R^2<S^2$, and from $m=3R^2+S^2$ we get
    $$
        12R^2 < m = \sqrt{A^4+D^4-A^2D^2} \le A^2+D^2\le 2\max(|A|,|D|)^2.
    $$
    Hence, $2|R|<\sqrt{2/3}\max(|A|,|D|)$. But
    $$
        2|R| < \sqrt{2/3}\max(|A|,|D|)\le \sqrt{2/3}|ABCD| = \sqrt{2/3}|uv| < |2uv| = |k|,
    $$
    which means that we have four integers in an arithmetic progression with distance less than $|k|$ between consecutive terms. The conclusion follows by induction on $|k|$.
    
\end{proof}

\begin{cor}
    If $f\in\Q[x]$ has degree $2$, then one of the polynomials $f(x)$, $f(x)+1$, $f(x)+2$ or $f(x)+3$ is irreducible.
\end{cor}

\begin{proof}
    Put $f=ax+bx+c$. For $i=0,1,2,3$, write
    $$
        f(x)+i = a\Big(x+\frac b{2a}\Big)^2+\Big(c-\frac{b^2}{4a}+i\Big).
    $$
    If none of the $f(x)+i$ is irreducible, we can pick a rational root $q_i$ for $f(x)+i$. Then
    $$
        \Big(q_0+\frac b{2a}\Big)^2
            ,\quad\Big(q_1+\frac b{2a}\Big)^2
            ,\quad\Big(q_2+\frac b{2a}\Big)^2
            ,\quad\Big(q_3+\frac b{2a}\Big)^2
    $$
    would be in arithmetic progression with distance $1/a$ between consecutive terms.
    
\end{proof}


\begin{lem}\label{lem:finite-extension->finitely-generated}
    If\/ $[K:\kappa]<\infty$, then\/ $K$ is algebraic and finitely generated over\/ $\kappa$, i.e., $K=\kappa(\alpha_1,\dots,\alpha_n)$ for some finite set\/ $\set{\alpha_1,\dots,\alpha_n}$ of algebraic elements over\/~$\kappa$.
\end{lem}

\begin{proof}
    Let $\set{\alpha_1,\dots,\alpha_n}$ be a (finite) basis of $K$ as $\kappa$-vector space. Then
    $$
        \kappa(\alpha_1,\dots,\alpha_n)\subseteq K
    $$
    and equality is attained because every element of $K$ is a $\kappa$-linear combination of $\alpha_1,\dots,\alpha_n$. In addition, every element $\alpha\in K$ is algebraic over $\kappa$ because $\set{1,\alpha,\dots,\alpha^n}$ is linearly dependent.
\end{proof}

\begin{rem}
    For the converse of the lemma see Proposition~\ref{prop:finitely-generated-extensions-are-finite}.
\end{rem}

\begin{prop}\label{prop:algebraic-iff-finite}
    Let\/ $K/\kappa$ be a field extension. Then\/ $\alpha\in K$ is algebraic over\/ $\kappa$ if, and only if, $[\kappa(\alpha):\kappa]<\infty$. Moreover, if\/ $[K:\kappa] < \infty$ then $K$ is algebraic over\/ $\kappa$.
\end{prop}

\begin{proof}
    The first statement follows from Proposition~\ref{prop:algebraic-minimal} and Lemma~\ref{lem:finite-extension->finitely-generated}. The second statement is consequence of the first.
\end{proof}

\begin{thm}
    Let\/ $F \subseteq L \subseteq K$ be two field extensions. Then
    $$
        [K : F] = [K : L][L : F].
    $$
\end{thm}

\begin{proof}
    Let $(\alpha_i)_{i\in I}$ be a basis of $L$ as $F$-vector space and $(\beta_j)_{j\in J}$ a basis of $K$ as $L$-vector space. Take $x\in K$. We can write
    $$
        x = \sum_{j\in J} b_j\beta_j
    $$
    where $b_j\in L$ and $b_j=0\aew$. Similarly, ever $b_j$ can be written as
    $$
        b_j = \sum_{i\in I}a_{ij}\alpha_i,
    $$
    where $a_{ij}\in F$ and $a_{ij}=p\aew$. In consequence
    $$
        x = \sum_{i\in I}\sum_{j\in J}a_{ij}\alpha_i\beta_j,
    $$
    which shows that $(\alpha_i\beta_j)_{i\in I,\;j\in J}$ generates $K$ over $F$. For the linear independence suppose that
    $$
        \sum_{i\in I,\;j\in J}c_{ij}\alpha_i\beta_j=0,
    $$
    with $c_{ij}\in F$ and $c_{ij}=0\aew$. Then
    $$
        0 = \sum_{j\in J}\Big(\sum_{i\in I} a_{ij}\alpha_i\Big)\beta_j,
    $$
    which implies first $\sum_{i\in I}c_{ij}\alpha_i=$ for every $j\in J$ and then that $c_{ij}=0$ for all $i\in I$ and $j\in J$.
\end{proof}

\begin{prop}\label{prop:finitely-generated-extensions-are-finite}
    Let\/ $K/\kappa$ be a field extension. If every element of a finite subset\/ $\set{\alpha_1,\dots,\alpha_n}$ of $K$ is algebraic over\/ $\kappa$, then\/ $\kappa[\alpha_1, \dots ,\alpha_n]$ is a finite dimensional field extension of\/ $\kappa$ with
    $$
        [\kappa[\alpha_1,\dots,\alpha_n]:\kappa]
            \le\prod_{i=1}^n[\kappa(\alpha_i):\kappa].
    $$
\end{prop}

\begin{proof}
    By induction on $n$. The case $n=1$ corresponds to Proposition~\ref{prop:algebraic-minimal}, where equality is attained.

    For $n>1$ put $L=\kappa[\alpha_1,\dots,\alpha_{n-1}]$. By the inductive hypothesis $L$ is a finite field extension of $\kappa$ and
    $$
        [L:\kappa]
            \le\prod_{i=1}^{n-1}[\kappa(\alpha_i):\kappa].
    $$
    Moreover, since $\alpha_n$ is algebraic over $L$, the case $n=1$ implies that $L[\alpha_n]$ is a field extension of $L$. Then, according to the theorem, we get
    $$
        [\kappa[\alpha_1,\dots,\alpha_n:\kappa]
            = [L[\alpha_n]:L][L:\kappa]
                \le[L(\alpha_n):L]\prod_{i=1}^{n-1}
                    [\kappa(\alpha_i):\kappa].
    $$
    This, the question reduces to showing that $[L(\alpha_n):L]\le[\kappa(\alpha_n):\kappa]$. But this is trivial because $\min_{L,\alpha_n}\mid\min_{\kappa,\alpha_n}$ since the $\alpha_n$ is a root of the latter.
\end{proof}

\begin{rem}
    Proposition~\ref{prop:finitely-generated-extensions-are-finite} proves the converse of Lemma~\ref{lem:finite-extension->finitely-generated}, i.e., every finitely generated algebraic field extension is finite.

    Note however that the field of algebraic complex numbers over $\Q$ is indeed algebraic but of infinite dimension as $\Q$-vector space.
\end{rem}

\begin{cor}\label{cor:algebraic-generators}
    Let\/ $K/\kappa$ be a field extension and let\/ $S$ be a subset of\/ $K$ such that each element of\/ $S$ is algebraic over\/ $\kappa$. Then\/ $\kappa(S)$ is algebraic over\/~$\kappa$. If\/ $|S|<\infty$, then\/ $[\kappa(S):\kappa]<\infty$.
\end{cor}

\begin{proof}
    Take $\alpha\in K(S)$. By Proposition~\ref{prop:finite-filtered-union} there exist $\alpha_1,\dots,\alpha_n\in S$ such that $\alpha\in\kappa(\alpha_1,\dots,\alpha_n)$. By Proposition~\ref{prop:finitely-generated-extensions-are-finite}, 
    $$
        [\kappa(\alpha):\kappa]
            \le[\kappa(\alpha_1,\dots,\alpha_n):\kappa]<\infty.
    $$
    Therefore, $\alpha$ is algebraic over $\kappa$ by Proposition~\ref{prop:algebraic-iff-finite}.
\end{proof}

\begin{thm}\label{thm:algebraic-transitivity}
    Let\/ $\kappa\subseteq F\subseteq K$ be field extensions. If\/ $F/\kappa$ and\/ $K/F$ are algebraic, then\/ $K/\kappa$ is algebraic.
\end{thm}

\begin{proof}
    Let $\alpha\in K$. We can write $\min_{F,\alpha}(x)=x^n+a_1x^{n-1}+\cdots+a_n$ with $a_1,\dots,a_n\in F$. Therefore, $\alpha$ is algebraic over $\kappa(a_1,\dots,a_n)$. In particular, $[\kappa(a_1,\dots,a_n,\alpha):\kappa(a_1,\dots,a_n)]<\infty$. Since $a_1,\dots,a_n$ are algebraic over~$\kappa$, $[\kappa(a_1,\dots,a_n):\kappa]<\infty$. Thus,
    \begin{align*}
        [\kappa(\alpha):\kappa]&\le [\kappa(a_1,\dots,a_n,\alpha):\kappa]\\
            &= [\kappa(a_1,\dots,a_n,\alpha):\kappa(a_1,\dots,a_n)]
            [\kappa(a_1,\dots,a_n):\kappa]\\
            &< \infty
    \end{align*}
    and son $\alpha$ is algebraic over $\kappa$ by Proposition~\ref{prop:algebraic-iff-finite}.
\end{proof}

\begin{defn}
    Let $K/\kappa$ be a field extension. The \textsl{algebraic closure\/} of $\kappa$ in $K$ is the set $\set{\alpha\in K\mid[\kappa(\alpha):\kappa]<\infty}$.
\end{defn}

\begin{cor}
    Let\/ $K/\kappa$ be a field extension and let\/ $L$ be the algebraic closure of\/ $\kappa$ in\/ $K$. Then\/ $L$ is a field, and therefore is the largest algebraic extension of\/ $\kappa$ contained in\/ $K$.
\end{cor}

\begin{proof}
    Take $\alpha$ and $\beta$ in $L$. Then $\alpha+\beta$ and $\alpha\beta$ are algebraic over $\kappa$ because they belong in $\kappa(\alpha,\beta)$, which is algebraic by Corollary~\ref{cor:algebraic-generators}.
\end{proof}

\begin{defn}
    Let $K/\kappa$ be a field extension. A subextension $\kappa \subseteq C\subseteq K$ is \textsl{algebraically closed in} $K$ if all $\alpha\in K$ algebraic over $C$ belongs to $C$, i.e., if every polynomial in $C[x]$ has all its roots in $C$.
\end{defn}

\begin{cor}\label{cor:closures-are-alg-closed}
    Let\/ $K/\kappa$ be a field extension. If\/ $C$ is the algebraic closure of\/~$\kappa$ in\/ $K$, then\/ $C$ is algebraically closed.
\end{cor}

\begin{proof}
    Take $\alpha\in K$, $\alpha$ algebraic over $C$. Then $[C(\alpha):C]<\infty$ and therefore there exist a finite extension $F/\kappa$, namely the one generated by the coefficients of $\min_{C,\alpha}$, such that $\alpha$ is algebraic over $F$. Thus 
    $$
        [\kappa(\alpha):\kappa]\le[F(\alpha):\kappa]=[F(\alpha):F][F:\kappa]<\infty,
    $$
    which implies that $\alpha$ is algebraic over $\kappa$, i.e., $\alpha\in C$.
\end{proof} 

\begin{defn}
    Let $K/\kappa$ be a field extension. If $L/\kappa$ and $L'/\kappa$ are subfield extension of $K$, then their \textsl{composite} $LL'$ is the field extension $L(L')=L'(L)$.
\end{defn}

\section{Problems}

\begin{probl}
    Let\/ $\kappa$ be a field.
    \begin{enumerate}[a), font=\upshape]
        \item Show that all the matrices 
        $$
        \begin{bmatrix}
            \hphantom-a & b \\
            -b & a
        \end{bmatrix},
        $$
        where\/ $a, b \in \kappa$, form a field\/ $L$ with respect to matrix addition and matrix multiplication if, and only if, the equation\/ $x^2 + 1 = 0$ has no solution in\/~$\kappa$.
        
        \item Show that\/ $L$ contains a field isomorphic to\/ $\kappa$.
        
        \item Use\/ {\rm a)} in order to construct a field with nine elements and find its characteristic.
    \end{enumerate}
\end{probl}

\begin{solution}${}$
    \begin{enumerate}[a),font=\upshape]
        \item The matrices are invertible if, and only if, their determinant, $a^2+b^2\ne0$.

        On the other hand
        $$
            \begin{bmatrix}
                \phantom-a  &b\\
                -b &a
            \end{bmatrix}
            \begin{bmatrix}
                \phantom-a'  &b'\\
                -b'  &a'
            \end{bmatrix}
            =
            \begin{bmatrix}
                \phantom-aa'-bb'  
                    &ab'+ba'\\
                -ba'-ab'  &-bb'+aa'
            \end{bmatrix},
        $$
        which shows that the product is commutative. Finally, the zero matrix and the identity have this form.

        \item The isomorphism $\kappa\to L$ is given by $a\mapsto a\op{Id}$.

        \item Take $\kappa=\F_3$, which is valid because $x^2+1\ne0$ in this field.
    \end{enumerate}
\end{solution}

\begin{probl}${}$
    \begin{enumerate}[a),font=\upshape]
    \item Formulate a suitable condition such that all matrices
    $$
        \begin{bmatrix}
            \hphantom-a & b \\
            -b & a - b
        \end{bmatrix},
    $$
    $a, b \in\kappa$, form a field with respect to matrix addition and multiplication.
    
    \item Use {\rm a)} in order to construct a field with four elements and write down the addition and the multiplication tables for the elements in this field.
    \end{enumerate}
\end{probl}

\begin{solution}${}$
    \begin{enumerate}[a),font=\upshape]
        \item The matrix is invertible if, and only if,
        $$
            a^2+b^2-ab=a(a-b)+b^2\ne0.
        $$
        If $a\ne0$ or $b\ne0$, this corresponds to
        $$
            x^2-x+1\ne0.
        $$
        For the commutativity, just observe that
        \small
        $$
            \begin{bmatrix}
                \hphantom-a & b \\
                -b & a - b
            \end{bmatrix}
            \begin{bmatrix}
                \hphantom-a' & b' \\
                -b' & a' - b'
            \end{bmatrix}
            =
            \begin{bmatrix}
                aa'-bb' & ab'+ba'-bb'\\
                -ba'-ab'+bb' & -bb' +(a-b)(a'-b')
            \end{bmatrix}.
        $$
        \normalsize

        \item The field of these matrices has $4$ elements when $\kappa=\F_2$, which satisfies the inequality given in part~a).
        
        If we identify these matrices with pairs $(a,b)\in\F_2^2$, the addition is coordinate to coordinate and the multiplication is given by
        $$
            (a,b)(a',b') = (aa'+bb',ab'+ba'+bb'),
        $$
        with unit $(1,0)$. Therefore,
        \begin{table}[h!]
          \centering
          \begin{tabular}{c|ccc}
                $\cdot$& $(1,0)$&$(0,1)$ & $(1,1)$\\ \hline
                $(1,0)$&$(1,0)$&$(0,1)$ & $(1,1)$\\ \hline
                $(0, 1)$&(0,1)&(1,1)&(1,0)\\ \hline
                $(1, 1)$&(1,1)&(1,0)&(0,1)
          \end{tabular}
    \end{table}
    \end{enumerate}
\end{solution}

\begin{probl}
   Show that if a monic polynomial\/ $f(x) \in \Z[x]$ has reductions modulo two prime numbers\/ $p, q$ such that:
    \begin{enumerate}[a),font=\upshape]
        \item $f(x) \pmod{p}$ is a product of two irreducible factors of degrees\/ $n_1,m_1$,
        \item $f(x) \pmod{q}$ is a product of two irreducible factors of degrees\/ $n_2,m_2$ and
        \item $\set{n_1,m_1}\ne\set{n_2,m_2}$,
    \end{enumerate}
    then\/ $f(x)$ is irreducible in\/~$\Q[x]$.
\end{probl}

\begin{solution} Let's begin with the following

    \textbf{Lemma.} \textit{If\/ $f\in\Z[x]$ is monic and\/ $q$ is a factor of\/ $f$ in\/ $\Q[x]$, then\/ $q\in\Z[x]$ and $q$ is monic.}

    \textsc{Proof.} Write $f(x)=q(x)r(x)$ in $\Q[x]$. Take positive integers $a$ and $b$ such that $aq(x),br(x)\in\Z[x]$ satisfy $\pp(aq(x))=\pp(br(x))=1$. Applying the $\pp$ map to $abf(x)=(ag(x))(br(x))$, by Lemma~\ref{lem:pp-is-multiplicative}, we get $ab=1$, meaning that $a=b=1$. Thus, $q(x),r(x)\in\Z[x]$. Consequently, both factors are monic.\hfill$\blacksquare$
    
    Suppose, for a contradiction, that $f=gh$ with $\deg(g),\deg(h)>0$. By the lemma, both $g$ and $h$ are monic and belong to $\Z[x]$. In particular, they keep their degrees unchanged when reduced modulo $p$ or $q$.

    Suppose that $f=s_1t_1$ in $\F_p[x]$ and $f=s_2t_2$ in $\F_q[x]$, where $\deg(s_i)=n_i$ and $\deg(t_i)=m_i$ for $i=1,2$. It follows that $gh=s_1t_1$ in $\F_p[x]$ and $gh=s_2t_2$ in $\F_q[x]$. By the irreducibility of the factors $s_i$ and $t_i$, after renaming the polynomials, we may assume that $g=s_1$ and $h=t_1$ in $\F_p[x]$ and $g=s_2$ and $h=t_2$ in $\F_q[x]$. However, this implies that
    \begin{align*}
        n_1&=\det(s_1)=\deg(g)=\deg(s_2)=n_2
    \intertext{and}
        m_1&=\det(t_1)=\deg(h)=\deg(t_2)=m_2,
    \end{align*}
    in contradiction with c).
\end{solution}

\begin{probl}
    Show that if\/ $f \in \kappa[x]$ is irreducible over\/ $\kappa$ and\/ $L \supseteq \kappa$ is a field extension such that the degree of\/ $f$ and the degree\/ $[L:\kappa]$ are relatively prime, then\/ $f$ is irreducible over\/~$L$.
\end{probl}

\begin{solution}
    Let $\alpha$ be a root of $f$ in an extension of $L$. Consider the following diagram
    $$
        \begin{tikzcd}
                    &L(\alpha)\\
            L
                    \arrow[ru,"d",no head]
                &{\kappa[\alpha]}
                    \arrow[u,"m"',no head]\\
            \kappa
                    \arrow[u,"{[L:\kappa]}",no head]
                    \arrow[ru,"\deg(f)"',no head]
        \end{tikzcd}
    $$
    Since $d[L:\kappa]=m\deg(f)$ and $\deg(f)\perp[L:\kappa]$, we deduce that $[L:\kappa]\mid m$. In particular, $[L:\kappa]\le m$. Put $n=[L:\kappa]$. If $\set{\beta_1,\dots,\beta_n}$ generates $L$ over $\kappa$, it is also generates $L(\alpha)$ over $\kappa(\alpha)$. Consequently, $m\le n$ and so $m=[L:\kappa]$. Therefore, $[L(\alpha):L]=\deg(f)$, which implies that $f=\min_{L,\alpha}$, which is irreducible over~$L$.
\end{solution}

\begin{probl} Show the following facts
    \begin{enumerate}[a),font=\upshape]
        \item If\/ $\alpha = p(x)/q(x)\in \kappa(x)$, where\/ $p, q \in \kappa[x]$ and\/ $p\perp q$, is not a constant, then\/ $[\kappa(x) : \kappa(\alpha)] \le n$, where $n=\max(\deg p, \deg q)$.
    
        \item If\/ $\alpha \in \kappa(x) \setminus \kappa$, then\/ $\alpha$ is transcendental over\/ $\kappa$.
    
        \item With the notations of\/ {\rm a)}, $[\kappa(x):\kappa(\alpha)]=n$.
    \end{enumerate}
\end{probl}

\begin{solution}${}$
    \begin{enumerate}[a),font=\upshape]
        \item We know that $\alpha q(x)-p(x)=0$. Therefore, $x$ is root of $f(t)=\alpha q(t)-p(t)$, which belongs in $\kappa(\alpha)[t]$. Moreover, $f(t)\ne0$. Otherwise, pick $\zeta\in\kappa$ such that $q(\zeta)\ne0$. Then $\alpha q(\zeta)-p(\zeta)=0$, implying that $\alpha\in\kappa$. Thus, $\deg(f(t))=n$.

        \item The reason why $\alpha$ cannot be algebraic over $\kappa$ is that, by part (a), this would imply $\kappa(x)$ is algebraic over $\kappa$, which is false.

        \item The polynomial $sq(t)-p(t)$ is primitive in $\kappa[t][s]$ and irreducible in $\kappa(t)[s]$. By Proposition~\ref{prop:primitive+irreducible=irreducible}, we deduce that it is irreducible in $\kappa(s)[t]$ and has degree $n$ (in $t$). By part~b) $\kappa(s)\cong\kappa(\alpha)$ and so $\alpha q(t)-p(t)$ is irreducible over $\kappa(\alpha)$, has degree $n$ and vanishes at~$t=x$.
    \end{enumerate}

    \textbf{Theorem.} {\rm[L\"uroth's theorem]} 
        Let\/ $\kappa(x)$ be the quotient field of the polynomial ring\/ $\kappa[x]$. Then every field\/ $K$ such that\/ $\kappa\varsubsetneq K\subseteq\kappa(x)$ is of the form\/ $K=\kappa(\alpha)$, where\/ $\alpha$ is transcendental over\/~$\kappa$.
    
    \textsc{Proof}
    Take $\alpha \in K \setminus \kappa$. Let $n = [\kappa(x) : K]$. Since $K$ contains $\kappa(\alpha)$, we have $n \leq [\kappa(x) : \kappa(\alpha)] < \infty$. Consider the minimal polynomial of $x$ over $K$:
    $$
        \varphi(t)=t^n+u_{n-1}t^{n-1}+\cdots+u_0.
    $$
    Since $\varphi(x) = 0$ and $x$ is transcendental over $\kappa$, at least one of the coefficients $u_i$ must lie outside $\kappa$. Fix such a $u_i$, and let $m = [\kappa(x) : \kappa(u_i)]$. Note that $u_i \in K$ because $\varphi(t) \in K[t]$. By the tower law, we have:
    \begin{equation}\label{eq:lüroth}
        n=[\kappa(x):K][K:\kappa(u_i)]=n[K : \kappa(u_i)].
    \end{equation}
    Write $u_i(x)=f(x)/g(x)$, where $f, g \in \kappa[x]$ are coprime. Then, $m = \max(\deg f, \deg g)$.
    
    Now, consider the polynomial $\phi(t) = u_i g(t) - f(t) \in K[t]$. Since $\phi(x) = 0$ and $\varphi(t)$ is the minimal polynomial of $x$ over $K$, it follows that $\varphi(t)$ divides $\phi(t)$ in $K[t]$. However, $\phi(t)$ is irreducible in $\kappa(x)[t]$. Therefore, $\varphi(t)$ and $\phi(t)$ are associates in $\kappa(x)[t]$. Comparing degrees in $t$, we obtain $n = m$. Substituting into \eqref{eq:lüroth}, we conclude that $[K : \kappa(u_i)] = 1$, which implies $K = \kappa(u_i)$.
            
\end{solution}


\begin{probl}\label{probl:sqrt5,sqrt7=sqrt5+sqrt7}
    Show that\/ $\Q(\sqrt5,\sqrt7) = Q(\sqrt5+\sqrt7)$.
\end{probl}

\begin{solution}
    First observe that
    $$
        \Q(\sqrt5)=\set{a+b\sqrt5\mid a,b\in\Q}.
    $$
    Second, $\set{1,\sqrt5,\sqrt7}$ is linearly independent over $\Q$. To see this suppose it suffices to show that $\sqrt7\notin\Q(\sqrt5)$. Suppose otherwise. Then we would have
    $$
        \sqrt7=a+b\sqrt5
    $$
    for some $a,b\in\Q$. Taking squares
    $$
        7= a^2+5b^2+2ab\sqrt5,
    $$
    which implies $ab=0$. Impossible. In particular, $\sqrt7\notin\Q(\sqrt5)$. Therefore,
    $$
        [\Q(\sqrt5,\sqrt7):\Q]=4.
    $$
    It follows that $[\Q(\sqrt5+\sqrt7):\Q]=2$ or $4$. Thus, the question reduces to show that $[\Q(\sqrt5+\sqrt7):\Q]>2$. Suppose otherwise. Therefore, there exist $a$ and $b$ in $\Q$ such that
    $$
        (\sqrt5+\sqrt7)^2+a(\sqrt5+\sqrt7)+b=0.
    $$
    In particular, we must have $a\ne0$. Multiplying by $\sqrt7-\sqrt5$, we get
    $$
        2(\sqrt5+\sqrt7)+2a+b(\sqrt7-\sqrt5)=0
    $$
    or
    $$
        (2-b)\sqrt5 + (2+b)\sqrt7 + 2a,
    $$
    which contradicts the linear independence because we cannot have $2-b=0$ and $2+b=0$.
\end{solution}

\begin{probl}
    Let\/ $K=\kappa(a)$ be a finite extension of\/ $\kappa$. For\/ $\alpha\in K$, let\/ $\eta_\alpha$ be the map from\/ $K$ to\/ $K$ defined by\/ $\eta_\alpha(x) = \alpha x$. Show that\/ $\det(xI-\eta_a)$ is the minimal polynomial\/ $\min_{\kappa, a}$ of\/ $a$. For which\/ $\alpha\in K$ is\/ $\det(xI-\eta_\alpha) = \min_{\kappa,\alpha}$?
\end{probl}

\begin{solution}
    Let $\min_{\kappa,a}(x)=x^n+a_1x^{n-1}+\cdots+a_n$. Then $\mathcal B=\set{1,a,a^2,\dots,a^{n-1}}$ is a basis of $K$ over $\kappa$. The matrix of $\eta_a$ in $\mathcal B$ is
    $$
        [\eta_a]_{\mathcal B}=\begin{pmatrix}
            0   &0  &\cdots &0 &-a_n\\
            1   &0  &\cdots &0 &-a_{n-1}\\
            \vdots&\vdots&\vdots&\vdots&\vdots\\
            0   &0  &\cdots &1 &-a_1
        \end{pmatrix}
    $$
    with $1$ at every entry of the subdiagonal and minus the $a_i$ in the last column. Therefore,
    $$
        \det(xI-\eta_a) = \det
        \begin{pmatrix}
            \hphantom-x   &0  &\cdots &\hphantom-0 &a_n\\
            -1   &x  &\cdots &\hphantom-0 &a_{n-1}\\
            \hphantom-\vdots&\vdots&\vdots&\hphantom-\vdots&\vdots\\
            \hphantom-0   &0  &\cdots &-1 &x+a_1
        \end{pmatrix}
    $$
    which equals $\min_{\kappa,a}(x)$ as it can be seen by developing the determinant by the last column.

    Since $\deg(\det(xI-\eta_\alpha))=n$, it equals the minimal $\min_{\kappa,\alpha}$ exactly when $\alpha$ generates $K$, i.e., when $\kappa(\alpha)=K$.
\end{solution}

\begin{probl}
    Let\/ $K/\kappa$ be a field extension. If\/ $a \in K$ is such that\/ $[\kappa(a):\kappa]$ is odd, show that\/ $\kappa(a) = \kappa(a^2)$. Give an example to show that this can be false if the degree of\/ $\kappa(a)$ over\/ $\kappa$ is even.
\end{probl}

\begin{solution}
    Let $f=\min_{\kappa,a}$ and $g=\min_{\kappa,a^2}$ with $\deg f=n$ and $\deg g=d$. If $\kappa(a^2)\ne\kappa(a)$, then $d<n$ and $d\mid n$ or $n=md$ with $m=[\kappa(a):\kappa(a^2)]>1$.
    
    On the other hand, given that $g(a^2)=0$, we see that $g(x^2)$ must be divisible by~$f$, i.e., there exists $q\in\kappa[x]$ such that
    $$
        g(x^2) = f(x)q(x).
    $$
    Taking degrees
    $$
        2d = n + \deg q = md + \deg q.
    $$
    Hence, $0=(m-2)d+\deg q$ which implies that $\deg q=0$ and $m=2$. Thus, $n=md=2d$, in contradiction with the hypothesis on the parity of~$n$.

    For a counterexample in the case $[\kappa(a):\kappa]$ even consider $\Q(\sqrt2)$.
\end{solution}

\begin{probl}\label{probl:subring-is-field}
    Let\/ $K/\kappa$ be an algebraic extension. If\/ $A$ is a subring of\/ $K$ with\/ $\kappa \subseteq A \subseteq K$, show that\/ $A$ is a field.
\end{probl}

\begin{proof}
    Take $a\in A\setminus\kappa$. We have to show that $a^{-1}\in A$. Since $a$ is algebraic over $\kappa$ the minimal $\min_{\kappa,a}$ produces a linear relation
    $$
        a^n+c_1a^{n-1}+\cdots+c_n=0,
    $$
    where $c_1,\dots,c_n\in\kappa$. Since $n>1$ (because $a\notin\kappa$) and the minimal is irreducible, $c_n\ne0$. Therefore, the equation
    $$
        a(a^{n-1}+c_1a^{n-2}+\cdots+c_{n-1})(-c_n^{-1})=1
    $$
    shows that $a^{-1}\in A$.
\end{proof}

\begin{probl}
    Show that $\Q(\sqrt2)$ and $\Q(\sqrt3)$ are not isomorphic as fields but are isomorphic as vector spaces over $\Q$.
\end{probl}

\begin{solution}
    Both fields are isomorphic as $\Q$-vector spaces simply because they both have dimension $2$. 

    Suppose that $\phi\colon\Q(\sqrt2)\to\Q(\sqrt3)$ is a isomorphism of fields. Put $\alpha=\phi(\sqrt2)$. Then 
    $$
        \alpha^2=\phi(\sqrt2)^2=\phi(2)=2.
    $$
    Since $\Q(\sqrt3)=\set{a+b\sqrt3\mid a,b\in\Q}$, we must have
    $$
        \alpha = a + b\sqrt3
    $$
    for some $a,b\in\Q$. Taking squares:
    $$
        2 = a^2+3b^2+2ab\sqrt3
    $$
    which implies $ab=0$. Since $b\ne0$, we must have $a=0$, i.e., $2=3b^2$, which is impossible.
\end{solution}

\begin{probl}
    Let\/ $\A$ denote the algebraic closure of\/ $\Q$ in\/ $\C$. Prove that\/ $[\A:\Q]=\infty$.
\end{probl}

\begin{solution}
    By the Eisenstein criterion, $x^n-2$ is irreducible for all $n\in\N$.
\end{solution}


\begin{probl}
    If\/ $L_1 = \kappa(a_1, \dots, a_n)$ and\/ $L_2 = \kappa(b_1, \dots, b_m)$, show that the composite\/ $L_1 L_2$ is equal to\/ $\kappa(a_1, \dots, a_n, b_1, \dots, b_m)$.
\end{probl}

\begin{solution}
    The inclusion $\kappa(a_1, \dots, a_n, b_1, \dots, b_m)\subseteq L_1L_2$ is clear. For the the other inclusion, observe that
    $$
        L_1L_2 = L_1(L_2)\subseteq \kappa(a_1, \dots, a_n, b_1, \dots, b_m)
    $$
    because $L_1\subseteq\kappa(a_1, \dots, a_n, b_1, \dots, b_m)$ and $L_2\subseteq\kappa(a_1, \dots, a_n, b_1, \dots, b_m)$.
\end{solution}

\begin{probl}
    Let\/ $K$ be a finite extension of\/ $\kappa$. If\/ $L_1$ and\/ $L_2$ are subextensions of\/ $K/\kappa$, show that\/ $[L_1 L_2:\kappa] \leq [L_1:\kappa] \cdot [L_2:\kappa]$. If\/ $[L_1:\kappa]\perp[L_2:\kappa]$, prove that\/ $[L_1 L_2:\kappa] = [L_1:\kappa] \cdot [L_2:\kappa]$.
\end{probl}

\begin{solution}
    By the previous problem the composite $L_1L_2$ is a finite extension of $\kappa$. Take a basis $\set{\alpha_1,\dots,\alpha_n}$ of the $\kappa$-vector space $L_1$ and another $\set{\beta_1,\dots,\beta_m}$ of $L_2$.

    First observe that
    $$
        L_1L_2=\bigg\{
            \sum_{k=1}^na_kb_k\mid n\in\N,\ a_k\in L_1,\ b_k\in L_2,
                \text{ for }1\le k\le n
            \bigg\}.
    $$
    Indeed, the RHS is a $\kappa$-algebra because it is closed under sums and products. Moreover, it included in $L_1L_2$. Therefore, it is a field by Problem~\ref{probl:subring-is-field}. Since $L_1$ and $L_2$ are included in this field, the LHS is included in the RHS. The other inclusion is clear.

    This description shows that
    $$
        \set{\alpha_i\beta_j\mid 1\le i\le n,\ 1\le j\le m}
    $$
    is a set of $nm$ generators of $L_1L_2$.

    The last statement follows from the fact that both $[L_1:\kappa]$ and $[L_2:\kappa]$ divide $[L_1L_2:\kappa]$.
\end{solution}


\begin{probl}
    Show that $[\Q(\sqrt[4]2,\sqrt3):\Q]=8$.
\end{probl}

\begin{solution}
    Since $[\Q(\sqrt[4]{2}):\Q]=4]$, it suffices to show that $\sqrt3\notin\Q(\sqrt[4]2)$. Suppose otherwise. Let $\omega=\sqrt2$. Observe that $\Q(\sqrt[4]{2})=\Q(\omega)(\sqrt\omega)$. Thus, our supposition means the existence of $\alpha,\beta\in\Q(\omega)$ such that
    $$
        \sqrt3 = \alpha + \beta\sqrt\omega.
    $$
    Then
    $$
        3 = \alpha^2+\beta^2\omega + 2\alpha\beta\sqrt\omega,
    $$
    which implies that $\alpha\beta\sqrt\omega\in\Q(\omega)$. Therefore, $\alpha\beta=0$.
    
    If $\alpha=0$, then $\sqrt3=(a+b\omega)\sqrt\omega$ for $a,b\in\Q$, we would have
    $$
        3=(a^2+2b^2+2ab\omega)\omega = (a^2+2b^2)\omega+4ab,
    $$
    which implies $a=b=0$, i.e., $\sqrt3=0$ $\to\leftarrow$.
    
    If $\beta=0$ then $\sqrt3\in\Q(\omega)$, which isn't the case either.
\end{solution}

\begin{probl}
    Give an example of field extensions\/ $L_1, L_2$ of\/ $\kappa$ for which\/ $[L_1 L_2:\kappa] < [L_1:\kappa] \cdot [L_2:\kappa]$.
\end{probl}

\begin{solution}
    The case $L_2=L_1$ is clearly an example. More generally, we can take $L_2\subseteq L_1$.
\end{solution}

\begin{probl}
    Give an example of a field extension\/ $K/\kappa$ with\/ $[K:\kappa] = 3$ but with\/ $K \neq \kappa(\sqrt[3]b)$ for any\/ $b \in \kappa$.
\end{probl}

\if{false}
\begin{solution}
    Take $f(t)=t^3-t-1$. Following Example~\ref{xmpl:k(t)/k(u)}, introduce $K=\Q(t)$, $u=f(t)\in K$ and $F=\Q(u)$. Then $[K:F]=\deg f=3$. Suppose that $K=F(\sqrt[3]g)$ for some $g\in F$. Write $\gamma=\sqrt[3]g$. We would have
    $$
        t=a_0+a_1\gamma+a_2\gamma^2
    $$
    for some $a_0,a_1,a_2\in F$. Then
    \begin{align*}
        t^3 &= a_0^3+(6a_0a_1a_2+a_1^3)g + a_2^3g^2\\
            &\quad+ 3a_0^2a_1\gamma + 3(a_0a_2^2+a_1^2a_2)\gamma^4\\
            &\quad+ 3(a_0^2a_2+a_0a_1^2)\gamma^2 + 3a_1a_2\gamma^5.\\
            &= a_0^3 + 3(a_0^2a_1+a_0a_2^2g+a_1^2a_2g)\gamma\\
            &\quad+ 3(a_0^2a_2+a_0a_1^2+a_1a_2g)\gamma^2.
    \end{align*}
    Using that $t^3-t=u+1$, we get
    \begin{align*}
        a_0^3+a_1^3g+a_2^3g^2+6a_0a_1a_2 &= a_0+u+1\\
         3(a_0^2a_1+a_0a_2^2g+a_1^2a_2g) &= a_1\\
         3(a_0^2a_2+a_0a_1^2+a_1a_2g) &= a_2.
    \end{align*}
\end{solution}

Another (failed) approach

\begin{solution}
    The polynomial $f(x)=x^3-x-1$ is irreducible in $\Q[x]$ (it doesn't have rational roots). Therefore, if $\alpha$ is a (real) root of $f$, we have $[\Q(\alpha):\Q]=3$.
    
    Suppose that $\Q(\alpha)=\Q(\sqrt[3]b)$ for some $b\in\Q$. Put $\beta=\sqrt[3]b$.
    
    Firstly let's see that $\set{1,\alpha,\beta}$ is linearly independent. Suppose otherwise. Then
    $$
        \beta = a_0 + a_1\alpha
    $$
    for some $a_0,a_1\in\Q$. Since $\alpha^3=\alpha+1$, we have
    $$
        b = a_0^3 + 3a_0^2a_1\alpha + 3a_0a_1^2\alpha^2+a_1^3(\alpha+1).
    $$
    The linear independence of $\set{1,\alpha,\alpha^2}$ implies
    $$
        3a_0a_1^2=0,\quad
        3a_0^2a_1+a_1^3=0,\quad
        a_0^3+a_1^3=b,
    $$
    i.e.,
    $$
        a_0a_1=0,\quad a_1=0,\quad a_0^3=b,
    $$
    or $\beta=a_0$, which is impossible.

    Now write
    \begin{align*}
        \beta\hphantom{{}^2} &= a_0 + a_1\alpha + a_2\alpha^2,\\
        \beta^2 &= c_0 + c_1\alpha + c_2\alpha^2.
    \end{align*}
    Then,
    \begin{align*}
        b
            &= (a_0 + a_1\alpha + a_2\alpha^2)
                (c_0 + c_1\alpha + c_2\alpha^2)\\
            &= a_0c_0 + (a_0c_1+a_1c_0)\alpha
                + (a_0c_2+a_1c_1+a_2c_0)\alpha^2\\
            &\quad+ (a_1c_2+a_2c_1)(\alpha+1)
                + a_2c_2(\alpha^2+2\alpha).
    \end{align*}
    Therefore,
    \begin{align*}
        b &= a_0c_0+a_1c_2+a_2c_1\\
        0 &= a_0c_1+a_1c_0+a_1c_2+a_2c_1+2a_2c_2\\
        0 &= a_0c_2+a_1c_1+a_2c_0 + a_2c_2.
    \end{align*}
\end{solution}
\fi

\begin{solution}
    Take the fields $\F_2$ and $\F_8$ with $2$ and $2^3$ elements. Note that $\F_8$ is an extension of $\F_2$. Moreover, $|\F_8|=2^3$ and $|\F_2|=2$, which implies that $[\F_8:\F_2]=3$. Now suppose that there exists $b\in\F_2$ such that $\F_8=\F_2(\sqrt[3]b)$. Then $b=1$ and $\alpha=\sqrt[3]b$ must satisfy $\alpha^3=1\in\F_2$. Therefore, $\alpha^4=\alpha$ and squaring both sides $\alpha^8=\alpha^2$. But $\alpha^7=1$ because $\alpha\in\F_8^*$, which is a group of order $7$. Hence, $\alpha^8=\alpha$. Therefore $\alpha^2=\alpha^8=\alpha$ and $\alpha=1$, in contradiction with the equation $\F_8=\F_2(\alpha)$.
    
\end{solution}

\begin{probl}${}$\label{probl:x^p-1/(x-1)-irred}
    \begin{enumerate}[a), font=\upshape]
        \item Let\/ $\kappa$ be a field, and let\/ $f(x) \in \kappa[x]$. If\/ $f(x) = \sum_i a_i x^i$ and\/ $\alpha \in \kappa$, let\/ $f(x + \alpha) = \sum_i a_i (x + \alpha)^i$. Prove that\/ $f$ is irreducible over\/ $\kappa$ if, and only if, $f(x + \alpha)$ is irreducible over\/ $\kappa$ for any\/ $\alpha \in \kappa$.
        
        \item Show that\/ $x^{p-1} + x^{p-2} + \cdots + x + 1$ is irreducible over\/ $\Q$ if\/ $p$ is an odd prime.\\
        {\rm(Hint: Replace $x$ by $x + 1$ and use the Eisenstein criterion.)}
    \end{enumerate}

    \begin{solution}${}$
        \begin{enumerate}[a), font=\upshape]
            \item The map $x\mapsto x+\alpha$ is an isomorphism of rings with inverse $x\mapsto x-\alpha$.

            \item Following the hint, consider
            $$
                f(x) = \sum_{i=0}^{p-1} (x+1)^i.
            $$
            Then
            $$
                f(x) = \sum_{i=0}^{p-1}\sum_{j=0}^i\binom ij x^j
                    = \sum_{j=0}^{p-1}\bigg[\sum_{i=j}^{p-1}\binom ij\bigg]x^j.
            $$
            The $k$th coefficient of $f$ is
            $$
                a_k = \sum_{i=k}^{p-1}\binom ik.
            $$
        \end{enumerate}
        In order to show that $a_k$ is divisible by $p$ for $0\le k<p-1$, we need the following

        \paragraph{Lemma {\rm[Hockey-Stick Identity]}}
        $$
            \sum_{i=k}^n\binom ik = \binom{n+1}{k+1}.
        $$
        \begin{quote}\small
            \proof
                By induction on $n$. The case $n=0$ is trivial. For $n>0$ we have
                \begin{align*}
                    \sum_{i=k}^n\binom ik &= \sum_{i=k}^{n-1}\binom ik+\binom nk\\
                        &= \binom n{k+1} + \binom nk\\
                        &= \binom{n+1}{k+1}.
                \end{align*}
            \hfill $\checkmark$
        \end{quote}
        
        Using the lemma, we obtain
        $$
            a_k = \binom p{k+1},
        $$
        which is known to be divisible by $p$ for all $k\le p-2$ (proof: $x^{p-1}=1$ in $\F_p^*$. Therefore $x^p=x$ and, for the very same reason $(x+1)^p=x+1$, hence $(x+1)^p=x^p+1$).
    \end{solution}
\end{probl}

\section{Automorphisms}

Let $K/\kappa$ be a field extension. Recall that a \textsl{$\kappa$-automorphism} of $K$ is a $\kappa$-algebra morphism $\phi\colon K\to K$. Since $\phi(1)=1$, we know that $\phi\ne0$ and therefore $\phi$ is an injection. Moreover, in the case where $[K:\kappa]<\infty$, $\phi$ is also onto because it is also $\kappa$-linear. More generally, a \textsl{$\kappa$-extension morphism} (succinctly, a \textsl{$\kappa$-morphism}) is a $\kappa$-algebra morphism between two field extensions of~$\kappa$. If $\phi\colon K\to F$ is such a morphism and $[K:\kappa]=[F:\kappa]<\infty$, then $\phi$ is an isomorphism.

\begin{defn}
    Let $K/\kappa$ be a field extension. The \textsl{Galois group\/} $\Gal(K/\kappa)$ is the set of all $\kappa$-automorphisms of $K$.
\end{defn}

\begin{rem}\label{rem:polynomials-commute-with-morphisms}
    Given a $\kappa$-morphism $\sigma\colon K\to F$ and $f\in\kappa[x_1,\dots,x_n]$, for any sequence of $n$ elements $a_1,\dots,a_n$ of $K$, we have
    $$
        \sigma\circ f(a_1,\dots,a_n)= f(\sigma(a_1),\dots,\sigma(a_n)).
    $$
\end{rem}

\begin{lem}\label{lem:automorphisms-and-generators}
    Let\/ $K = \kappa(S)$ be a field extension of\/ $\kappa$ that is generated by a subset\/ $S$ of\/ $K$. If\/ $\sigma, \theta\in\Gal(K/\kappa)$ with\/ $\sigma|_S=\omega|_S$, then\/ $\sigma=\omega$. Therefore, $\kappa$-automorphisms of\/ $K$ are determined by their action on a generating set.
\end{lem}

\begin{proof}
    This is a direct consequence of Proposition~\ref{prop:extension-generators} and Remark~\ref{rem:polynomials-commute-with-morphisms}.
\end{proof}

\begin{lem}
    Let\/ $\sigma\colon K\to F$ be a\/ $\kappa$-morphism and let\/ $\alpha\in K$ be algebraic over\/ $\kappa$. If\/ $f(x)$ is a polynomial over\/ $\kappa$ with\/ $f(\alpha) = 0$, then\/ $f(\sigma(\alpha))=0$. Therefore, $\sigma$ permutes the roots of\/ $\min_{F,\alpha}$. Also, $\min_{F,\alpha}=\min_{F,\sigma(\alpha)}$.
\end{lem}

\begin{proof}
    This is a direct consequence of Remark~\ref{rem:polynomials-commute-with-morphisms}.
\end{proof}

\begin{thm}\label{thm:finite-extension-finite-galois}
    If\/ $[K: \kappa] < \infty$, then\/ $|\Gal(K/\kappa)|<\infty$.
\end{thm}

\begin{proof}
    By Lemma~\ref{lem:finite-extension->finitely-generated}, we can write $K = \kappa(\alpha_1, \dots, \alpha_n)$, where $\alpha_1, \dots, \alpha_n$ are algebraic over $\kappa$. Let $Z$ be the set of roots of all the minimal polynomials $\min_{\kappa, \alpha_1}, \dots, \min_{\kappa, \alpha_n}$. According to the previous lemma, every $\sigma \in \Gal(K/\kappa)$ permutes the elements of $Z$. Since $Z$ is finite, there are at most $|Z|$ possibilities for $\sigma(z)$ for every $z \in Z$. It follows that there are finitely many possibilities for $\sigma(\alpha_1), \dots, \sigma(\alpha_n)$. The conclusion follows because, by Lemma~\ref{lem:automorphisms-and-generators}, these values determine $\sigma$.


\end{proof}

\begin{xmpls}\label{xmpls:minimal}${}$
    \begin{enumerate}[a), font=\upshape]
        \item Consider $\Q(\alpha)$ where $\alpha=\sqrt[3]2$. Since $\min_{\Q,\alpha}(x)=x^3-2$, we know that the other two roots of this polynomial are the complex numbers $\omega\alpha$ and $\omega^2\alpha$ where $\omega$ is a primitive $3$rd root of $1$, say $\omega=e^{2\pi i/3}$. If $\sigma\in\Gal(\Q(\alpha),\Q)$, then $\sigma(\alpha)$ cannot be a complex root, otherwise we would deduce that $\omega$ or $\omega^2$ belongs to $\Q(\alpha)$ which is impossible because $[\Q(\omega):\Q]=2$ ($\omega^2+\omega+1=0$) while $[\Q(\alpha):\Q]=3$. It follows that $\sigma(\alpha)=\alpha$, hence $\sigma=\id$, i.e., $\Gal(\Q(\alpha)/\Q)$ is trivial.

        \item Let $K=\F_2(t)$ and put $F=\F_2(t^2)$. The polynomial $f(x)=x^2-t^2$ is irreducible in $F[x]$ because $t\notin F$. Thus, $f$ is the minimal $\min_{F,t}$ and $[K:F]=2$. Since $x^2-t^2=(x-t)^2$ there is only one root of $\min_{F,t}$, namely $t$ and so $\Gal(K/F)$ is trivial.

        \item Let $f(x) = x^2+x+1\in\F_2[x]$. Since $f$ has no roots in $\F_2$, it is irreducible. Let $K=\F_2[x]/\gen{\,f\,}$. Then,
        $$
            K=\set{a+b\alpha\mid a,b\in\F_2},
        $$
        where $\alpha$ satisfies $\alpha^2+\alpha+1=0$. Let $\beta$ be the other root of $f$ then
        $$
            (x-\alpha)(x-\beta) = x^2+x+1\iff \alpha+\beta=1
                \text{ and }\alpha\beta=1.
        $$
        Since $\alpha(\alpha+1)=1$, the second equality is redundant. Therefore, $\beta=\alpha+1$. Consider the permutation $\sigma$ between $\alpha$ and $\beta$. We have
        $$
            \sigma(a+b\alpha)=a+b(\alpha+1)=a+b+b\alpha.
        $$
        \needspace{2\baselineskip}
        We claim that $\sigma$ is an $\F_2$-automorphism. Indeed,
        \begin{enumerate}[-]
            \item $\sigma(0)=0$ and
            \begin{align*}
                \sigma(a+a'+(b+b')\alpha) &= a+a'+b+b'+(b+b')\alpha\\
                    &=\sigma(a+b\alpha)+\sigma(a'+b'\alpha).
            \end{align*}

            \item $\sigma(1)=1$ and

            \item $\sigma$ preserves products:
            \begin{align*}
                \sigma((a+b\alpha)(a'+b'\alpha))
                    &= \sigma(aa'+(ab'+ba')\alpha + bb'(\alpha+1))\\
                    &= \sigma(aa'+bb'+(ab'+a'b + bb')\alpha)\\
                    &= aa'+bb'+ab'+a'b+bb'\\
                    &\quad+ (ab'+a'b+bb')\alpha\\
                    &= aa'+ab'+a'b+(ab'+a'b+bb')\alpha.\\
                \sigma(a+b\alpha)\sigma(a'+b'\alpha)
                    &= (a+b+b\alpha)+(a'+b'+b'\alpha)\\
                    &= (a+b)(a'+b')\\
                    &\quad+ ((a+b)b'+b(a'+b'))\alpha+bb'(\alpha+1)\\
                    &= (a+b)(a'+b')\\
                    &\quad+ bb'+(ab'+bb'+ba'+bb'+bb')\alpha\\
                    &= aa'+ab'+ba'+(ab'+ba'+bb')\alpha.
            \end{align*}
        \end{enumerate}
        In conclusion, $\Gal(K/\F_2)=\set{\id,\sigma}$.
    \end{enumerate}
\end{xmpls}

\begin{defns}
    Let $K$ be a field. Given a subset $S\subseteq\Aut(K)$ the \textsl{fixed field\/} of $S$ is
    $$
        \mathcal F(S) = \set{a\in K\mid \sigma(a)=a\text{ for all } \sigma\in S}.
    $$
    If $K/\kappa$ is a field extension a field $\kappa\subseteq F\subseteq K$ is an \textsl{intermediate field} of $K/\kappa$.
\end{defns}

\begin{rem}
    Note that $\mathcal F(S)$ is a subfield of $K$. In particular, in the case where $S\subseteq\Gal(K/\kappa)$, the fixed field $\mathcal F(S)$ is an intermediate field of $K/\kappa$.
\end{rem}

\begin{lem}\label{lem:fixed-field-properties}
    Let\/ $K$ be a field. The following properties hold true:
    \begin{enumerate}[a), font=\upshape]
        \item If\/ $L_1 \subseteq L_2$ are subfields of\/ $K$, then\/ $\Gal(K/L_2) \subseteq \Gal(K/L_1)$.
        \item If\/ $L$ is a subfield of\/ $K$, then\/ $L \subseteq \mathcal F(\Gal(K/L))$.
        \item If\/ $S_1 \subseteq S_2$ are subsets of\/ $\Aut(K)$, then\/ $\mathcal F(S_2) \subseteq \mathcal F(S_1)$.
        \item If\/ $S$ is a subset of\/ $\Aut(K)$, then\/ $S \subseteq \Gal(K/\mathcal F(S))$.
        \item If\/ $F = \mathcal F(S)$ for some\/ $S \subseteq \Aut(K)$, then\/ $F = \mathcal F(\Gal(K/F))$.
        \item If\/ $G = \Gal(K/L)$ for some subfield\/ $L$ of\/ $K$, then\/ $G = \Gal(K/\mathcal F(G))$.
    \end{enumerate}
\end{lem}

\needspace{2\baselineskip}
\begin{proof}${}$
    \begin{enumerate}[a), font=\upshape]
        \item If $\sigma$ fixes every element of $L_2$, it fixes every element of~$L_1$.

        \item Every element of $L$ is fixed by every $\sigma\in\Gal(K/L)$.

        \item If $a$ is fixed by every element of $S_2$, it is fixed by every element of~$S_1$.

        \item Every element of $S$ is an automorphism of $K$ that fixes the fixed field of~$S$.

        \item From part~b) we get $F\subseteq\mathcal F(\Gal(K,F))$. From part~d)
        $$
            S\subseteq\Gal(K/\mathcal F(S))=\Gal(K/F).
        $$
        Then $\mathcal F(\Gal(K/F))\subseteq\mathcal F(S)=F$ by part~c).

        \item By hypothesis, $L\subseteq\mathcal F(G)$. Then $\Gal(K/\mathcal F(G))\subseteq\Gal(K/L)=G$ by part~a). The other inclusion follows from part~d).
    \end{enumerate}
\end{proof}

\begin{thm}\label{thm:inclusion-reversing-correspondence}
    If\/ $K/\kappa$ is a field extension, there is a bijective inclusion-reversing correspondence between the set of subgroups of\/ $\Gal(K/\kappa)$ of the form\/ $\Gal(K/F)$, for some intermediate field\/ $F$ of\/ $K/\kappa$, and the set of intermediate fields of\/ $K/\kappa$ of the form\/ $\mathcal F(S)$ for some subset\/ $S$ of\/ $\Aut(K)$. This correspondence is given by\/ $F\mapsto \Gal(K/F)$, and its inverse is given by\/ $G\mapsto\mathcal F(G)$.
\end{thm}

\begin{proof}
    Introduce
    \begin{align*}
        \mathcal F &= \set{\mathcal F(S)\mid S\subseteq\Aut(K)},\\
        \mathcal G &= \set{\Gal(K/\mathcal F(S))\mid S\subseteq\Aut(K)}.
        \intertext{Then,}
        \mathcal F&\to\mathcal G\\
        F&\mapsto\Gal(K/F)
        \intertext{and}
        \mathcal G&\to\mathcal F\\
        G&\mapsto\mathcal F(G).
    \end{align*}
    We have to show that $\mathcal F(\Gal(K/F))=F$ and $\Gal(K/\mathcal F(G))=G$. The first equation is part~e) of the previous lemma. The second is part~f).
\end{proof}

\begin{defn}
    If\/ $G$ is a group and if\/ $K$ a field, a \textsl{character} is a  group morphism from\/ $G$ to\/ $K^*$. In particular, the elements of\/ $\Gal(K/\kappa)$ can be seen as characters from\/ $K^*$ to\/ $K^*$
\end{defn}


\begin{lem}\label{prop:dedekind-lemma}{\rm[Dedekind Lemma]}
    Let\/ $G$ be a group and\/ $K$ a field. Every set\/ $\set{\chi_1,\dots,\chi_n}$ of characters from\/ $G$ to\/ $K^*$ is linearly independent over\/~$K$.
\end{lem}

\begin{proof}
    By induction on $n$. The case $n=1$ is trivial. If $n>1$, pick $y\in K^*$ with $\chi_n(y)\ne\chi_1(y)$. Suppose that $a_1\chi_1+\cdots+a_n\chi_n=0$. Multiplication by $\chi_n(y)$ and evaluation at $x$ and $yx$, we get
    $$
        \sum_{i=1}^na_i\chi_n(y)\chi_i(x)=0
        \quad\text{and}\quad
        \sum_{i=1}^na_i\chi_i(y)\chi_i(x)=0.
    $$
    Subtraction leads to
    $$
        \sum_{i=1}^{n-1}a_i(\chi_n(y)-\chi_i(y))\chi_i=0.
    $$
    By induction, $a_i(\chi_n(y)-\chi_i(y))=0$ for $1\le i<n$. Since $\chi_n(y)\ne\chi_1(y)$, we must have $a_n=0$. The conclusion follows by the hypothesis.

\end{proof}

\begin{thm}\label{thm:gal-order-bound}
    If\/ $K/\kappa$ is a finite extension, then\/ $|\Gal(K/\kappa)|\le [K:\kappa]$.
\end{thm}

\begin{proof}
    By Theorem~\ref{thm:finite-extension-finite-galois}, $m=|\Gal(K/\kappa)|$ is finite. Put $n=[K:\kappa]$ and consider the $n\times m$ matrix $A=(\chi_j(\alpha_i))$, where $\set{\chi_1,\dots,\chi_m}=\Gal(K/\kappa)$ and $\set{\alpha_1,\dots,\alpha_n}$ is a basis of $K$ over $\kappa$. If we suppose that $n<m$, the $\kappa$-linear map $f\colon K^m\to K^n$ induced by $A$ must have a nontrivial kernel. Pick $(a_1,\dots, a_m)\in\ker(f)\setminus\set{(0,\dots,0)}$. Then, for every $1\le i\le n$, we have
    $$
        \sum_{j=1}^m\chi_j(\alpha_i)a_j=\pi_i\circ f(a_1,\dots,a_m)=0,
    $$
    which implies that $a_1\chi_1+\cdots+a_m\chi_m=0$ in $\Gal(K/\kappa)$, in contradiction with Dedekind's Lemma.
\end{proof}

\begin{rem}
    As shown in Examples~\ref{xmpls:minimal}~a) and~b), equality is not always attained in the inequality of the theorem. 
\end{rem}

\begin{cor}\label{cor:galois-group=dim}
    Let\/ $G$ be a finite group of\/ $\Aut(K)$ and\/ $F = \mathcal F(G)$. Then\/ $|G| = [K:F]$ and\/ $G=\Gal(K/F)$.
\end{cor}

\begin{proof}
    If $[K:F]<\infty$, the theorem implies that $|G|\le|\Gal(K/F)\le[K:F]$. Otherwise, $|G|\le[K:F]$ because $|G|<\infty$. Thus, $|G|\le[K:F]$ in any case. Suppose, toward a contradiction, that equality is not attained. Put $G=\set{\chi_1,\dots,\chi_n}$ and pick $n+1$ linearly independent elements $\alpha_1,\dots,\alpha_{n+1}$ in $K$. Let $A$ be the $n\times(n+1)$ matrix $(\chi_i(\alpha_j))$. Then $A$ induces a linear map $f\colon K^{n+1}\to K^n$. Take $(a_1,\dots,a_{n+1})$ in $\ker(f)\setminus\set{(0,\dots,0)}$ with the minimum number of nonnull entries. Without loss of generality we may assume that $a_1=1$. For every $1\le i\le n$ we get
    \begin{equation}\label{eq:K-linear-dependence}
        \sum_{j=1}^{n+1}a_j\chi_i(\alpha_j)=\pi_i\circ f(a_1,\dots,a_{n+1})=0.
    \end{equation}
    Take $\chi\in G$. Given $1\le i\le n$, $\chi\circ\chi_i$ is an element of $G=\set{\chi_1,\dots,\chi_n}$. Thus, there exists $\sigma\in S_n$ such that $\chi\circ\chi_i=\chi_{\sigma(i)}$. Therefore, for $1\le i\le n$,
    $$
        \sum_{j=1}^{n+1}\chi(a_j)\chi_{\sigma(i)}(\alpha_j)=0,
    $$
    i.e., for $1\le i\le n$,
    $$
        \sum_{j=1}^{n+1}\chi(a_j)\chi_i(\alpha_j)=0,
    $$
    which means that $(1,\dots,\chi(a_{n+1}))=(\chi(a_1),\dots,\chi(a_{n+1}))\in\ker(f)$. Thus,
    $$
        (0,a_2-\chi(a_2),\dots,a_{n+1}-\chi(a_{n+1}))\in\ker(f).
    $$
    It follows that the definition of $(a_1,\dots,a_{n+1})$ implies that $a_j=\chi(a_j)$ for all $1\le j\le n+1$. Since $\chi$ was arbitrarily chosen, this means that $a_j\in\mathcal F(G)=F$ for all $j$. Then
    $$
        \chi_i\bigg(\sum_{j=1}^{n+1}a_j\alpha_j\bigg)
            = \sum_{j=1}^{n+1}a_j\chi_i(\alpha_j)
            = 0,
    $$
    which implies that $a_1\alpha_1+\cdots+a_n\alpha_{n+1}=0$, in contradiction with Dedekind's Lemma. The contradiction originated in the inequality $|G|<[K:F]$. Therefore, $|G|=[K:F]$ and, by Theorem~\ref{thm:gal-order-bound}, 
    $$
        [K:F]=|G|\le|\Gal(K:F)|\le[K:F],
    $$
    which shows that $G=\Gal(K/F)$.
\end{proof}

\begin{defn}
    Let\/ $K/\kappa$ be an algebraic extension. Then\/ $K/\kappa$ is a \textsl{Galois extension\/} if\/ $\kappa= \mathcal F(\Gal(K/\kappa))$.
\end{defn}

\begin{cor}\label{cor:galois-iff-order=dim}
    Let\/ $K/\kappa$ be a finite extension. Then\/ $K/\kappa$ is Galois if, and only if, $|\Gal(K/\kappa)|=[K:\kappa]$.
\end{cor}

\begin{proof}${}$
    \begin{description}
        \item[\rm\textit{only if\/}:] This follows from Corollary~\ref{cor:galois-group=dim}.
        
        \item[\rm\textit{if\/} part:] Put $G=\Gal(K/\kappa)$ and $F=\mathcal F(G)$. By Corollary~\ref{cor:galois-group=dim}, $|G|=[K:F]$. Hence,
        $$
            [K:\kappa] =|G|=[K:F],
        $$
        which shows that $F=\kappa$, i.e., $K/\kappa$ is Galois.
    \end{description}
\end{proof}

\begin{cor}\label{cor:galois-group-order-and-min-deg}
    Let\/ $K/\kappa$ be a field extension and\/ $\alpha \in K$ algebraic over\/~$\kappa$. Then\/ $|\Gal(\kappa(\alpha)/\kappa)|$ is equal to the number of distinct roots of\/ $\min_{\kappa, \alpha}$ in\/ $\kappa(\alpha)$. Therefore, $\kappa(\alpha)$ is Galois over\/ $\kappa$ if, and only if, $\min_{\kappa,\alpha}$ has\/ $n$ distinct roots in\/ $\kappa(\alpha)$, where\/ $n = \deg(\min_{\kappa, \alpha})$.
\end{cor}

\begin{proof}
    Introduce the following notation
    \begin{align*}
        \hspace*{-3.5cm}G &= \Gal(\kappa(\alpha)/\kappa)\\
        \hspace*{-3.5cm}f &= \textstyle\min_{\kappa,\alpha}\\
%        \hspace*{-3.5cm}d &= \deg(f)\\
        \hspace*{-3.5cm}Z &= \set{\zeta\in\kappa(\alpha)
            \mid f(\zeta)=0}.
    \end{align*}
    Firstly assume that $|Z|=n$. Then $|G|\le |Z|$ because the evaluation at $\alpha$
    \begin{align*}
        G&\to Z\\
        \sigma&\mapsto\sigma(\alpha)
    \end{align*}
    is well-defined and injective.
    
    Now take $\zeta\in Z$. Recall from Proposition~\ref{prop:algebraic-minimal}~d), that there is a (unique) element $\varepsilon_\zeta\in G$, induced by the evaluation at $\zeta$: $\kappa[x]\to\kappa(\alpha)$ ($x\mapsto\zeta$),
    defined as
    \begin{align*}
        \varepsilon_\zeta\colon\kappa(\alpha)&\to\kappa(\alpha)\\
        g(\alpha)&\mapsto g(\zeta).
    \end{align*}
    Moreover, the map
    \begin{align*}
        Z&\to G\\
        \zeta&\mapsto\varepsilon_\zeta
    \end{align*}
    is injective because
    $$
        \varepsilon_\zeta=\varepsilon_\xi\implies
            \zeta=\varepsilon_\zeta(\alpha)=\varepsilon_\xi(\alpha)=\xi.
    $$
    By the previous corollary,
    $$
        [\kappa(\alpha):\kappa]=\deg(f)=|Z|=|G|\implies \kappa(\alpha)\text{ is Galois}.
    $$
    Conversely, if $\kappa(\alpha)/\kappa$ is Galois, the previous corollary implies that $|G|=n$. It follows that $\set{\sigma(\alpha)\mid\sigma\in G}$ has $|G|=n$ elements and is included in $Z$. Then $|Z|=n$, as desired.
\end{proof}

\begin{xmpls}\label{xmpl:galois-extensions}${}$
    \begin{enumerate}[a), font=\upshape]
        \item The extension $\Q(\sqrt[3]2)/\Q$ isn't Galois because the minimal $x^3-2$ has only one real root.

        \item Let $\kappa$ be a field of characteristic $p>0$. The field extension $\kappa(t)/\kappa(t^p)$ isn't Galois either because the minimal of $t$ satisfies $\min_{\kappa(t^p),t}\mid x^p-t^p=(x-t)^p$, which implies that it has only one root in $\kappa(t)$, namely $t$ while, of course, $\kappa(t)\ne\kappa(t^p)$.

        \item Let $\kappa$ be a field with $\fchar(\kappa)\ne2$ and $a\in\kappa\setminus\kappa^2$. Then the polynomial $x^2-a$ is irreducible in $\kappa[x]$ and the quotient field $F=\kappa[x]/\gen{x^2-a}$ satisfies $[F:\kappa]=2$, with basis $\set{1,\alpha}$, where $\alpha$ is the class of $x$ in~$F$. Since the other root of this polynomial is $-\alpha\ne\alpha$, we see that $F/\kappa$ is Galois. Note in particular that $\Gal(F/\kappa)=\set{\id,\sigma}$, where $\sigma(s+t\alpha)=s-t\alpha$ for all $s,t\in\kappa$.

        \item The extension $\Q(\sqrt[3]2,\omega)/\Q$, where $\omega= e^{2\pi i/3}$, is Galois. To see this first observe that $\min_{\Q,\alpha}(x)=x^3-2=(x-\alpha)(x-\omega\alpha)(x-\omega^2\alpha)$, where $\alpha=\sqrt[3]2\in\R$. Moreover, $\min_{\Q,\omega}=x^2+x+1$, which shows that $\omega\notin\Q(\alpha)$ because $[\Q(\omega):\Q]=2$ while $[\Q(\alpha):\Q]=3$. Thus $[Q(\alpha,\omega):\Q]=6$.

        Therefore, to show that the extension is Galois, by Corollary~\ref{cor:galois-iff-order=dim}, it is enough to exhibit $6$ elements of the Galois group. These elements are the ones defined as follows
        \begin{align*}
            \hspace{-3cm}\sigma_0\colon\alpha&\mapsto\alpha
                &\hspace{-2.5cm}\omega\mapsto\omega\hphantom{{}^2.}\\
            \hspace{-3cm}\sigma_1\colon\alpha&\mapsto\alpha
                &\hspace{-2.5cm}\omega\mapsto\omega^2\hphantom.\\
            \hspace{-3cm}\sigma_2\colon\alpha&\mapsto\alpha\omega
                &\hspace{-2.5cm}\omega\mapsto\omega\hphantom{{}^2.}\\
            \hspace{-3cm}\sigma_3\colon\alpha&\mapsto\alpha\omega
                &\hspace{-2.5cm}\omega\mapsto\omega^2\hphantom.\\
            \hspace{-3cm}\sigma_4\colon\alpha&\mapsto\alpha\omega^2
                &\hspace{-2.5cm}\omega\mapsto\omega\hphantom{{}^2.}\\
            \hspace{-3cm}\sigma_5\colon\alpha&\mapsto\alpha\omega^2
                &\hspace{-2.5cm}\omega\mapsto\omega^2.
        \end{align*}
    \end{enumerate}
\end{xmpls}

\section{Automorphisms of the Complex Numbers}

In this section all fields are subfields of $\C$.

\begin{lem}\label{lem:automorphisms-fix-Q}
    If\/ $F\subseteq\C$ is a field, then every automorphism of\/ $F$ extends\/~$\id_\Q$.
\end{lem}

\begin{proof}
    Let $\sigma\colon F\to F$ be an automorphism. Since $\sigma(1)=1$, it follows that $\sigma(n)=n$ for all $n\in\N$, then in $\Z$ and finally in $\Q$.
\end{proof}

\begin{thm}
    Let\/ $F$ and\/ $K$ be intermediate fields of\/ $\C/\R$. If\/ $\phi\colon F\to K$ is an isomorphism such that\/ $\phi(\R)\subseteq\R$, then\/ $\phi\subseteq\id_\R$,\/ $\id_\C$, or\/ $\sigma$, where\/ $\sigma$ is the complex conjugation.
\end{thm}

\begin{proof}
    Let $\phi_\R\in\Aut(\R)$ be defined by $\phi_\R\subseteq\phi$. Since $\phi_\R(1)=1$, the previous lemma implies that $\phi_\R|_{\Q}=\id_{\Q}$. Moreover, if $a>0$, there exists $b\in\R$ such that $a=b^2$, which implies that $\phi_\R(a)=\phi_\R(b)^2>0$. It follows that $a<b$ implies $\phi_\R(a)<\phi_\R(b)$ ($\phi_\R(b-a)>0)$). In consequence, $|x|<\delta$ implies $|\phi_\R(x)|<\phi_\R(\delta)$. Therefore, given $\epsilon>0$, by taking $\delta=\phi_\R^{-1}(\epsilon)$, which is positive because $\phi_\R^{-1}$ is also an automorphism, we get $|x|<\delta\implies|\phi_\R(x)|<\epsilon$. This means that $\phi_\R$ is continuous at $0$. By translation, $\phi_\R$ is continuous everywhere and so $\phi_\R=\id$. Now take $a+ib\in\C$. We know that $\phi(a+ib)=a+\phi(i)b$. And since $\phi(i)^2=\phi(i^2)=-1$, we deduce that $\phi(i)=\pm i$, i.e., $\phi\subseteq\id_\C$ or $\phi\subseteq\sigma$.
    
\end{proof}

\begin{cor}
    The only automorphism of a subfield of\/ $\R$ that can be extended to\/ $\R$ is the identity.
\end{cor}

\begin{xmpls}\label{xmpls:proto-wild-automorphisms}${}$
    \begin{enumerate}[a), font=\upshape]
        \item The conjugation $a+bi\mapsto a-bi$ is an automorphism of $\C$.

        \item $\sigma\colon\Q(\sqrt7)\to\Q(\sqrt7)$, defined by $\sigma(a+b\sqrt7)=(a-b\sqrt7)$ is an automorphism.

        \item $\psi\colon\Q(\sqrt[4]7)\to\Q(i\sqrt[4]7)$ defined by
        $$
            \psi(a_0+a_1\sqrt[4]7+a_2\sqrt7+a_3\sqrt[4]{7^3})
                = a_0 + ia_1\sqrt[4]7-a_2\sqrt7-ia_3\sqrt[4]{7^3}
        $$
        is an extension of $\sigma$ defined in part~b).

        In fact, $\psi$ maps $\sqrt[4]7$ to $i\sqrt[4]7$. Since $\sqrt7=-\big(i\sqrt[4]7\big)^2$, we deduce that $\Q(\sqrt7)\subseteq\Q(i\sqrt[4]7)$.
    \end{enumerate}
\end{xmpls}

\begin{defn}
    A \textsl{wild automorphism} of $\C$ is an automorphism which is not the identity nor the complex conjugation.
\end{defn}

\begin{thm}
    If $\phi$ is a wild automorphism of $\C$ then $\phi$ is a discontinuous mapping of the complex plane onto itself; in fact, $\phi$ leaves a dense subset of the real line pointwise fixed but maps the real line onto a dense subset of the complex plane.
\end{thm}

\begin{proof}
    From Lemma~\ref{lem:automorphisms-fix-Q} we know that $\phi$ extends $\id_\Q$. Thus, $\phi$ leaves this dense subset of the real line pointwise fixed.

    Given $\phi$ is wild there exists $b\in\R$ such that $\phi(b)\notin\R$. And since every neighborhood of $b$ contains a rational number, which is fixed by $\phi$, we see that $\phi$ is discontinuous at $b$. The set
    $$
        D = \set{r+s\phi(b)\mid r,s\in\Q}
    $$
    is dense in $\C$ because for every $s\in\Q$ the set $\set{r+s\phi(b)\mid r\in\Q}$ is dense in the line passing though $0$ and $\phi(b)$. Since $D$ is the image of $\set{r+sb\mid r,s\in\Q}$, we see that $\phi$ maps (a dense subset of) the real line onto a dense subset of the complex plane.
\end{proof}

As a generalization of Proposition~\ref{prop:algebraic-minimal}, we have the following

\begin{prop}\label{prop:algebraic-extension-morphism}
    Let\/ $\phi\colon F\to F'$ be an isomorphism. If\/ $\alpha$ is algebraic over\/ $F$ then there is an isomorphism extending\/ $\phi$ to\/ $F(\alpha)$ and sending\/ $\alpha$ to\/ $\beta$ if, and only if, $\beta$ is a root of the polynomial obtained by applying\/ $\phi$ to the coefficients of the minimal polynomial\/~$\min_{F,\alpha}$.
\end{prop}

\begin{proof}
    Let $f\in F[x]$ denote the minimal $\min_{F,\alpha}$ and let $g\in F'[x]$ be the polynomial obtained by $f$ applying $\phi$ to its coefficients. If $\phi$ can be extended to $F(\alpha)$, then
    $$
        0=\phi(f(\alpha)) = g(f(\alpha)),
    $$
    i.e., $f(\alpha)$ is a root of $g$. Conversely, let $\beta$ be a root of $g$. The map
    \begin{align*}
        F[x]&\to F'(\beta)\\
        \sum_{i=0}^na_ix^i&\mapsto\sum_{i=0}^n\phi(a_i)\beta^i
    \end{align*}
    is a ring morphism that sends $\min_{F,\alpha}$ to $0$. Therefore, it induces a morphism from $F(\alpha)=F[x]/\gen{\min_{F,\alpha}}$ to $F'(\beta)$. Moreover, by definition, the induced morphism extends $\phi$.
\end{proof}

\begin{prop}\label{prop:transcendent-extension-morphism}
    Let\/ $\psi\colon F\to F'$ be an isomorphism. If\/ $\alpha$ is transcendental over\/ $F$, then there is an isomorphism extending\/ $\psi$ to\/ $F(\alpha)$ and sending\/ $\alpha$ to\/ $\beta$ if, and only if, $\beta$ is transcendental over\/~$F'$.
\end{prop}

\begin{proof}
    Suppose that $\alpha$ is transcendent over $F$. If an extension of $\psi$ maps $\alpha$ to $\beta$, then $\beta$ cannot be algebraic because the polynomial obtained from $\min_{F',\beta}$ by replacing its coefficients with their preimages in $F$ via $\psi$. Therefore, $\beta$ is transcendent over $F'$. On the other hand, if $\beta$ is transcendent over $F'$, then $\psi$ can be extended to $F[\alpha]$ by the map
    \begin{align*}
        \psi_{\alpha\beta}\colon F[\alpha]&\to F'[\beta]\\
        \sum_{i=0}^na_i\alpha^i&\mapsto\sum_{i=0}^n\psi(a_i)\beta^i,
    \end{align*}
    which is well-defined because the map $F[x]\to F[\alpha]$, $x\mapsto\alpha$, is an isomorphism of $F$-algebras. Since $\psi_{\alpha\beta}$ is an isomorphism, it's extension to $F(\alpha)\to F'(\beta)$ is straightforward.
\end{proof}

\begin{xmpls}
    By Proposition~\ref{prop:algebraic-extension-morphism}, the only extensions of $\id_\Q$ to $\Q(\sqrt7)$ are the identity and the automorphism $\sigma$ of Example~\ref{xmpls:proto-wild-automorphisms}~b).
    
    Since $\min_{\Q(\sqrt7),\sqrt[4]7}$ is $x^2-\sqrt7$. And given that $\sigma(\sqrt7)=-\sqrt7$, this polynomial is sent to $x^2+\sqrt7$, whose roots are $\pm i\sqrt[4]7$. According to the same proposition, $\sigma$ can be extended to $\Q(\sqrt[4]7)$ in exactly two ways: one is $\psi$, from part~c) of the same example, and the other the one that maps $\sqrt[4]7$ to $-i\sqrt[4]7$.
\end{xmpls}

\begin{rem}
    There are uncountably many complex numbers which are transcendental over the image of $\psi$, $\Q(i\sqrt{7})$. Thus by Proposition~\ref{prop:transcendent-extension-morphism} there are uncountably many ways of extending $\psi$ to $\Q(\sqrt7,\pi)$. A few of these possibilities send $\pi$ to $1/\pi$, $1-\pi$, $\pi+\sqrt{57}$, or $e/17$.
    

    These examples should convince the reader that there are many isomorphisms between finitely generated extensions of $\Q$. Since many of these are clearly automorphisms differing radically in their action from the identity or complex conjugation, it will follow from the forthcoming result (any automorphism can be extended to an automorphism of $\C$) that there are many wild automorphisms of~$\C$.
\end{rem}

\begin{thm}
    Let\/ $F$ and\/ $F'$ be subfields of\/ $\C$ with algebraic closures\/ $\bar F$ and\/~$\bar{F'}$. If\/ $\phi\colon F\to F'$ is an isomorphism, then\/ $\phi$ can be extended to an isomorphism with domain\/ $\bar F$ and range\/~$\bar{F'}$.
\end{thm}

\begin{proof}
    Let's say that an isomorphism $\varphi$ with domain in a subfield of $\bar F$ and codomain in a subfield of $\bar{F'}$ is \textsl{good} if $\phi\subseteq\varphi$ . Then introduce
    $$
        \mathcal F = \set{\varphi\mid \varphi \text{ is good}}.
    $$
    Since $\mathcal F\subseteq\C\times\C$, we see that $\mathcal F$ is a set. It is not empty because $\phi$ is good. Let $\mathcal C$ be a filtrant subfamily of~$\mathcal F$. The union $\Phi=\bigcup\mathcal C$ is a relation that satisfies 
    $$
        \dom\Phi = \bigcup\set{\dom\varphi\mid\varphi\in\mathcal C}.
    $$
    Clearly, $\dom\Phi\subseteq\bar F$. Moreover, if $(z,w)$ and $(z,w')$ are in $\Phi$, by definition there exist $\varphi$ and $\varphi'$ such that $\varphi(z)=w$ and $\varphi'(z)=w'$. Since the subfamily is filtrant, there exits $\psi\in\mathcal C$ such that $\varphi\cup\varphi'\subseteq\psi$. Therefore,
    $$
        w=\varphi(z)=\psi(z)=\varphi'(z)=w',
    $$
    i.e., $\Phi$ is a function. Similar arguments show that (1)~$\dom\Phi$ is a subfield of $\bar F$ and (2)~$\Phi$ is a morphism. In consequence, $\Phi$ is an isomorphism onto its image. To see that $\im\Phi$ is algebraic over $F'$ note every element in it is of the form $\varphi(z)$ for some good isomorphism $\varphi$ and some $z\in\bar F$, which allows us to invoke Proposition~\ref{prop:algebraic-extension-morphism}.

    In sum $\Phi$ is good and we can apply Zorn's lemma to $\mathcal F$ to conclude that it contains some maximal element $\bar\Phi$. We claim that equality is attained in the inclusion $\dom\bar\Phi\subseteq\bar F$. Otherwise, there would exist $\alpha$ algebraic over $F$ and not in $\dom\Phi$. Let $f$ denote the minimal of $\alpha$ over $\dom\bar\Phi$. If $g$ is obtained from $f$ by applying $\bar\Phi$ to its coefficients, then $g$ must have a root $\beta$ in $\bar{F'}$ because, by Corollary~\ref{cor:closures-are-alg-closed}, $\bar{F'}$ is algebraically closed. Then, by Proposition~\ref{prop:algebraic-extension-morphism}, $\Phi$ could be extended to $\dom\bar\Phi(\alpha)$, in contradiction with its maximality. Finally, since $\bar\Phi$ is an isomorphism and $\bar F$ is algebraically closed, $\im\bar\Phi$ is algebraically closed too, hence equal to $\bar{F'}$.
\end{proof}

\begin{thm}
    Any automorphism of a subfield of\/ $\C$ can be extended to an automorphism of\/~$\C$.
\end{thm}

\begin{proof}
    Let $\phi\colon F\to F$ be an automorphism of a subfield $F$ of $\C$. Similarly to what we did in the proof of the previous theorem, let's say that an automorphism $\varphi$ of a subfield of $\C$ is \textsl{good}, $\phi\subseteq\varphi$. Then introduce
    $$
        \mathcal F = \set{\varphi\mid\varphi \text{ is good}}.
    $$
    Given a filtrant subfamily $\mathcal C$ of $\mathcal F$, we define the relation $\Phi=\bigcup\mathcal C$, which satisfies
    $$
        \dom\Phi = \bigcup\set{\dom\varphi\mid\varphi\in\mathcal C}.
    $$
    As in the previous proof, $\mathcal F$ is set, it is nonempty, $\Phi$ is a function, $\dom\Phi$ is a subfield of $\C$, and $\Phi$ is an isomorphism onto its image. It remains to be verified that $\im\Phi=\dom\Phi$. But
    $$
        \im\Phi =\bigcup\set{\im\varphi\mid\varphi\in\mathcal C}
            = \bigcup\set{\dom\varphi\mid\varphi\in\mathcal C}
            = \dom\Phi.
    $$
    Therefore, we can invoke Zorn's lemma and conclude that there is a maximal automorphism $\bar\Phi$ that extends $\phi$ to a subfield of~$\C$. We claim that such a subfield is $\C$. To check this suppose, for a contradiction, that there exists some complex number $z$ not contained in $F=\dom\Phi$. There are two cases. If $z$ is algebraic over $F$, then we can use Proposition~\ref{prop:algebraic-extension-morphism} to extend $\Phi$ to $F(z)$, which contradicts the maximality of $\bar\Phi$. If $z$ is transcendent over $F$, then we can arrive to the same conclusion using Proposition~\ref{prop:transcendent-extension-morphism}. Thus, $F=\C$ and the proof is complete.
    
\end{proof}

\needspace{2\baselineskip}
\begin{lem}
    Let $X$ be an infinite set. Then $|\Sym(X)|=2^{|X|}$.
\end{lem}

\begin{proof}
    Fix two elements $x,y\in X$ and introduce $X'=X\setminus\set{x,y}$. The core of the proof relies on showing that the map
    \begin{align*}
        \Sym(X)&\to\mathcal P(X')\\
        \sigma&\mapsto\set{a\in X'\mid\sigma(a)=a}
    \end{align*}
    is surjective.

    To see this, take $A\subseteq X'$. Let's first consider the case where $X\setminus A$ is infinite. Divide $X\setminus A$ in two equipotent disjoint sets $X\setminus A=U\cup V$. Let $\phi\colon U\to V$ be a bijection. Define
    \begin{align*}
        \sigma\colon X\setminus A&\to X\setminus A\\
            z&\mapsto\begin{cases}
                \phi(z) &\text{if }z\in U,\\
                \phi^{-1}(z)    &\text{if } z\in V.
            \end{cases}
    \end{align*}
    note that $\sigma\in\Sym(X\setminus A)$ and $\sigma(z)\ne z$ for all $z\in X\setminus A$. Extend $\sigma$ to $\Sym(X)$ by setting $\sigma(a)=a$ for $a\in A$. Then $A$ is the set of fixed points of~$f$.

    In the case where $X\setminus A$ is finite, we can repeat the same partition as above if its cardinal is even. If it is odd, pick $z\in X\setminus A$, $z\notin\set{x,y}$ and consider the cyclic permutation $(x, y, z)$. We can now partition $X\setminus\set{x,y,z}\setminus A$ into two disjoint sets with the same number of elements and proceed as before.
    
    The surjective map shows that $2^{|X'|}\le|\Sym(X)|$. Since $\Sym(X)\subseteq\mathcal P(X\times X)$, we get
    $$
        2^{|X|} = 2^{|X'|} \le |\Sym(X)|\le2^{|X\times X|}=2^{|X|}.
    $$
\end{proof}

\begin{cor}
    The cardinal of\/ $\Aut(\C)$ is\/ $2^{2^{\aleph_0}}$.
\end{cor}

\begin{proof}{[Pablo Armas]}
    Let's say that a subset $T$ of $\C$ is \textsl{transcendent\/} over $\Q$ when, for every $t\in T$, it holds that $t$ is transcendent over $\Q(T\setminus\set t)$.
    
    Let $\mathcal T$ be the family of subsets of $\C$ that are transcendent over $\Q$. If $\mathcal F$ is a filtrant subfamily of $\mathcal T$, we claim that $U=\bigcup\mathcal F$ is transcendent over $\Q$. Suppose otherwise. Then there exists $u\in U$, algebraic over $K=\Q(U\setminus\set u)$. Let $f=\min_{K,u}$. Since $f$ has finitely many coefficients, We can pick $V\in\mathcal F$ such that $f\in\Q(V)[x]$. This leads to a contradiction because we can pick $W\in\mathcal F$ such that $\set u\cup V\subseteq W$.
    
    By Zorn's lemma, $\mathcal T$ has a maximal element $\bar T$. Since $\C$ is algebraic over $\Q(\bar T)$ we must have $|\bar T|=|\C|$. By the theorem, every automorphism of $\Q(\bar T)$ can be extended to an automorphism of $\C$. Therefore,
    $$
        |\Aut(\C)|\ge|\Aut(\Q(\bar T))|\ge|\Sym(\bar T)|
           = \Sym(\C)\ge|\Aut(C)|,
    $$
    i.e., $|\Aut(\C)|=|\Sym(\C)|$, which according to the lemma equals $2^{|\C|} = 2^{2^{\aleph_0}}$.
\end{proof}

\section{Problems}

\begin{probl}${}$
    Let\/ $D$ be an integral domain and $\sigma\colon D \to D$ is a ring automorphism. If\/ $F$ is the quotient field of\/ $D$, show that\/ $\sigma$ induces a ring automorphism $\sigma^*\colon F \to F$ defined by\/ $\sigma^*(a/b) = \sigma(a)/\sigma(b)$ if\/ $a, b \in D$ with\/ $b \neq 0$.
\end{probl}

\begin{solution}
    Given that $a/b=c/d$ if, and only if, $ad=bc$, the quotient $\sigma(a)/\sigma(b)$ only depends on $a/b$ and not on the particular numerator and denominator used to represent it. Since $\sigma\subseteq\sigma^*$, it follows that $\sigma^*(1)=1$. Moreover,
    $$
        \sigma^*\Big(\frac ab\frac cd\Big)
            = \frac{\sigma(ac)}{\sigma(bd)}
            = \frac{\sigma(a)\sigma(c)}{\sigma(b)\sigma(d)}
            = \sigma^*\Big(\frac ab\Big)\sigma^*\Big(\frac cd\Big).
    $$
    Finally,
    $$
        \sigma^*\Big(\frac ab+\frac cd\Big)
            %= \sigma^*\Big(\frac{ad+bc}{bd}\Big)
            = \frac{\sigma(ad+bc)}{\sigma(bd)}
            = \frac{\sigma(a)\sigma(d)+\sigma(b)\sigma(c)}{\sigma(b)\sigma(d)}
            = \sigma^*\Big(\frac ab\Big)+\sigma^*\Big(\frac cd\Big).
    $$
\end{solution}

\begin{probl}
    Let\/ $K = \kappa(x_1, \dots, x_n)$ be the field of rational functions in\/ $n$ variables over a field\/ $\kappa$. Show that the definition 
    $$
        \sigma\bigg(\frac{f(x_1,\dots,x_n)}{g(x_1,\dots,x_n)}\bigg)
            = \frac{f(x_{\sigma(1)},\dots,x_{\sigma(n)})}
                {g(x_{\sigma(1)},\dots,x_{\sigma(n)})}
    $$
    makes a permutation\/ $\sigma \in S_n$ into a field automorphism of\/ $K$.
\end{probl}

\begin{solution}
    This is a direct consequence of the previous problem and the fact that $\sigma$ induces a $\kappa$-algebra isomorphism between $\kappa[x_1,\dots,x_n]$ and $\kappa[x_{\sigma(1)},\dots,x_{\sigma(n)}]$.
\end{solution}

\begin{probl}
    Let\/ $\kappa$ be a field of characteristic not\/ $2$, and let\/ $K$ be an extension of\/ $\kappa$ with\/ $[K : \kappa] = 2$. Show that\/ $K = \kappa(\sqrt{a})$ for some\/ $a \in \kappa$; that is, show that\/ $K = \kappa(\alpha)$ with\/ $\alpha^2 = a \in \kappa$. Moreover, show that\/ $K$ is Galois over\/~$\kappa$.
\end{probl}

\begin{solution}
    Let $\set{1,\beta}$ be a basis of $K$ as $\kappa$-vector space. There must exist $b,c\in\kappa$ such that $\beta^2=b+c\beta$. Then, $\alpha=\beta-c/2$ satisfies
    $$
        \alpha^2 = \beta^2-c\beta+c^2/4 = b + c^2/4,
    $$
    i.e., $\alpha=\sqrt{b+c^2/4}$. Since $\set{1,\alpha}$ is also a basis of~$K$ and $\kappa(\alpha)\subseteq K$, we deduce that $K=\kappa(\alpha)$. The extension is Galois by Corollary~\ref{cor:galois-group-order-and-min-deg} because $\min_{\kappa,\alpha}=x^2-a$, where $a=b+c^2/4$, with $a\ne-a$ because $2\ne0$ and $a=\alpha^2\ne0$ either.
\end{solution}

\begin{probl}
    Let\/ $K = \F_2(\alpha)$, where\/ $\alpha$ is a root of\/ $1 + x + x^2$. Show that the function\/ $\sigma : K \rightarrow K$ given by\/ $\sigma(a + b\alpha) = a + b + b\alpha$ for $a, b \in\F_2$ is an\/ $\F_2$-automorphism of\/ $K$.
\end{probl}

\begin{solution}
    Since $1+x+x^2$ has no roots in $\F_2$, it is irreducible. Therefore, $[K:\F_2]=2$ and so $\sigma$ is well-defined. Moreover,
    \begin{align*}
        \sigma(a+b\alpha+c+d\alpha) &= a+b+c+d+(b+d)\alpha\\
            &=\sigma(a+b\alpha)+\sigma(c+d\alpha).\\
        \sigma((a+b\alpha)(c+d\alpha)) &= \sigma(ac+bd(\alpha+1)+(ad+bc)\alpha)\\
            &= \sigma(ac+bd+(bd+ad+bc)\alpha)\\
            &= ac+\cancel{bd}+\cancel{bd}+ad+bc+(bd+ad+bc)\alpha.\\
        \sigma(a+b\alpha)\sigma(c+d\alpha) &= (a+b+b\alpha)(c+d+d\alpha)\\
            &= (a+b)(c+d)+bd(\alpha+1)+((a+b)d+(c+d)b)\alpha\\
            &= ac + ad + bc +(bd+ad+\cancel{bd}+cb+\cancel{db})\alpha\phantom{\qedhere}
    \end{align*}
\end{solution}

\begin{probl}
    Show that the complex numbers\/ $i\sqrt{3}$ and\/ $1+i\sqrt{3}$ are roots of\/ $f(x)=x^4 - 2x^3 + 7x^2 - 6x + 12$. Let\/ $K$ be the field generated by\/ $\Q$ and the roots of\/ $f$. Is there an automorphism\/ $\sigma$ of\/ $K$ with\/ $\sigma(i\sqrt{3}) = 1 + i\sqrt{3}$?
\end{probl}

\begin{solution}
    Firstly observe that
    \begin{align*}
        f(x) &= x^4 - 2x^3 + 7x^2 - 6x + 12\\
            &= x^4 - 2x^3 + 6x^2 + x^2 - 6x + 12\\
            &= x^2(x^2-2x + 6) + x^2 - 2x + 6 - 4x + 6\\
            &= (x^2+1)(x^2-2x+6) - 4x+6.
    \end{align*}
    Therefore,
    \begin{align*}
        f(i\sqrt3) &= \overbrace{(-3+1)}^{-2\hphantom-}(-3-2i\sqrt3+6)- 4i\sqrt3 + 6\\
            &= 6+4i\sqrt3-12-4i\sqrt3 + 6\\
            &= 0.
    \end{align*}
    Moreover, using that $(1+i\sqrt3)^2=1-3+2i\sqrt3=-2+2i\sqrt3$, we get
    \begin{align*}
        f(1+i\sqrt3) &= (-1+2i\sqrt3)
                \overbrace{(-2+\cancel{2i\sqrt3}-2-\cancel{2i\sqrt3}+6)}^2-4-4i\sqrt3 + 6\\
            &= -2 + 4i\sqrt3 - 4i\sqrt3+2\\
            &= 0.
    \end{align*}
    It follows that the other roots of $f$ are $-i\sqrt3$ and $1-i\sqrt3$. But the minimal of $i\sqrt3$ is $x^2+3$ and the minimal of $1-i\sqrt3$ is $(x-1)^2+3$. Since these two polynomials are different, by Proposition~\ref{prop:algebraic-extension-morphism}, no isomorphism of $K$ maps $i\sqrt3$ to $1+i\sqrt3$.
\end{solution}

\begin{probl}
    Determine whether the following fields are Galois over $\Q$.
    \begin{enumerate}[a), font=\upshape]
        \item $\Q(\omega)$, where $\omega = e^{2i\pi/3}$.
        \item $\Q(\sqrt[4]2)$.
        \item $\Q(\sqrt{5}, \sqrt{7})$.
    \end{enumerate}
\end{probl}

\begin{solution}${}$
    \begin{enumerate}[a), font=\upshape]
        \item The minimal is $\min_{\Q,\omega}(x)=x^3-1$ and the other two roots are $1$ and $e^{4i\pi/3}$. Since $e^{4i\pi/3}=\omega^2$, all three roots belong in $\Q(\omega)$. Hence, $\Q(\omega)/\Q$ is Galois by Corollary~\ref{cor:galois-group-order-and-min-deg}.

        \item The minimal is $\min_{\Q,\sqrt[4]2}(x)=x^4-2$, which is irreducible by Eisenstein criterion Lemma~\ref{lem:eisenstein}. However, since the roots $\pm i\sqrt[4]2$ are not in $\Q(\sqrt[4]2)$, Corollary~\ref{cor:galois-group-order-and-min-deg} shows that this extension is not Galois.

        \item Problem~\ref{probl:sqrt5,sqrt7=sqrt5+sqrt7} shows that $\Q(\sqrt5,\sqrt7)=\Q(\sqrt5+\sqrt7)$ is a $\Q$-vector space of dimension $4$. And since $\sqrt5+\sqrt7$ is root of $f=(x^2-12)^2-4\cdot35$, it follows that $f=\min_{\Q,\sqrt5+\sqrt7}$. But
        $$
            f = (x^2-12-2\sqrt{35})(x^2-12+2\sqrt{35}),
        $$
        which shows that the four roots of $f$ are real, namely $\pm(\sqrt7+\sqrt5)$ and $\pm(\sqrt7-\sqrt5)$, and belong in $\Q(\sqrt5,\sqrt7)$. Hence, the extension is Galois by Corollary~\ref{cor:galois-group-order-and-min-deg}.\qedhere
    \end{enumerate}
\end{solution}

\begin{probl}
    Prove or disprove the following assertion and its converse: If\/ $\kappa \subseteq F \subseteq K$ are fields with\/ $K/F$ and\/ $F/\kappa$ Galois, then\/ $K/\kappa$ is Galois.
\end{probl}

\begin{solution}
    Galois transitivity doesn't hold true. For instance, consider
    $$
        \begin{tikzcd}
            {\Q(\sqrt[4]2)}
                \arrow[dd,bend left=50,"\cancel{\text{G}}" {pos=0.47},no head]\\
            \Q(\sqrt2)
                \arrow[u,"\text{G}",no head]\\
            \Q
                \arrow[u,"\text{G}", no head]
        \end{tikzcd}
    $$
    Both subextensions are Galois because their minimal polynomials are $x^2-2$ and $x^2-\sqrt2$. However, the total extension is not.
    
    The converse is not true either. The idea behind this insight is to consider a simple extension that is not Galois because it doesn't contain all the roots of the minimal and then extend again to include them.
    
    Take $\Q\subseteq\Q(\sqrt[4]2)\subseteq\Q(\sqrt[4]2,i)$. Put $\alpha=\sqrt[4]2$. Then, $\min_{\Q,\alpha}(x)=x^4-2$ with roots $\pm\alpha$ and $\pm i\alpha$. From the previous problem we know that $\Q(\alpha)/\Q$ is not Galois. On the other hand, $\Q(\alpha,i)/\Q(\alpha)$ is Galois because the minimal of~$i$ over $\Q(\alpha)$ is~$x^2+1$. However, $\Q(\alpha,i)/\Q$ is Galois. To see this, take $\sigma\in\Gal(\Q(\alpha,i))$. First note that $\sigma(i)^2=\sigma(-1)=-1$. Then, $\sigma$ is determined by $\sigma(i)\in\set{i,-i}$ and $\sigma(\alpha)\in\set{\pm\alpha,\pm i\alpha}$, which implies that $|\Gal(\Q(\alpha,i),\Q)|=8$. Therefore,
    $$
        8 = |\Gal(\Q(\alpha,i)/\Q)|\le[\Q(\alpha,i):\Q]=8.
    $$
    By Corollary~\ref{cor:galois-iff-order=dim}, $\Q(\alpha,i)$ is Galois.

    Note however that
    $$
        \begin{tikzcd}[column sep=tiny]
                &{\Q(i,\alpha)}
                    \arrow[dd,"\text{G}", no head]\\
            \Q(i)
                    \arrow[ru,"\text{G}",no head]
                &&\Q(\alpha)
                    \arrow[lu,"\text{G}"',no head]\\
                &\Q
                    \arrow[lu,"\text{G}" {pos=0.47},no head]
                    \arrow[ru,"\cancel{\text{G}}"' {pos=0.54}, no head]
        \end{tikzcd}
    $$
\end{solution}

\begin{probl} {\rm[Galois connections]}
    If\/ $S$ is a set, a relation\/ $\le$ on\/ $S$ is called a \textsl{partial order} on\/ $S$, provided that
    \begin{enumerate}[-]
        \item {\rm reflexivity: $a\le a$ for all\/ $a\in S$;}
        \item {\rm antisymmetry: $a\le b$ and\/ $b\le a \implies a = b$, and}
        \item {\rm transitivity: $a\le b$ and\/ $b\le c\implies a\le c$.}
    \end{enumerate} 
    Let\/ $S$ and\/ $T$ be partial ordered sets. Suppose that there are functions\/ $f\colon S\to T$ and\/ $g\colon T\to S$ such that
    \begin{enumerate}[\rm 1)]
        \item $s_1 \le s_2\implies f(s_2)\le f(s_1)$,
        \item {\rm$t_1\le t_2\implies g(t_2) \le g(t_1)$, and}
        \item {\rm$s\le g(f(s))$ and\/ $t\le f(g(t))$ for all\/ $s\in S$ and\/ $t\in T$.}
    \end{enumerate}
    Prove that there is a one-to-one order reversing correspondence between the image of\/ $g$ and the image of\/ $f$, given by\/ $s\mapsto f(s)$, whose inverse is\/ $t\mapsto g(t)$.
\end{probl}

\begin{solution}
    Let's see that $f|_{\im(g)}$ is an injection. Take $s_1=g(t_1)$ and $s_2=g(t_2)$. Then,
    \begin{align*}
        f(s_1)=f(s_2) &\implies t_1\stackrel{3)}{\le} f(g(t_1)) = f(g(t_2))\\
            &\implies s_2\stackrel{3)}{\le} g(f(s_2))
                =g(f(s_1))\stackrel{2)}{\le} g(t_1)=s_1\\
            &\implies s_2\le s_1
                &&\text{; transitivity},
    \end{align*}
    By symmetry between $s_1$ and $s_2$, $s_1\le s_2$. Then by antisymmetry, $s_1=s_2$. Thus, $f|_{\im(g)}$ is injective. By symmetry between $f$ and $g$, $g|_{\im(f)}$ is injective.

    It remains to be seen that $f_{\im(g)}\colon\im(g)\to\im(f)$, $f_{\im(g)}\subseteq f$, is surjective. Take $f(s)\in \im(f)$. We have to show that $f(s)\in f(\im(g))$. But
    $$
        s\stackrel{3)}{\le}g(f(s))\quad\text{and so}\quad
            f(g(f(s))\stackrel{2)}{\le}f(s)\stackrel{3)}{\le} f(g(f(s))),
    $$
    which implies $f(s)=f(g(f(s))\in f(\im(g))$. By symmetry between $f$ and $g$, $g|_{\im(f)}\colon\im(f)\to\im(g)$ is surjective.
\end{solution}

\begin{probl}
    Let\/ $\kappa$ be a field, and let\/ $\kappa(x)$ be the rational function field in one variable over\/ $\kappa$. Let\/ $\sigma$ and\/ $\tau$ be the automorphisms of\/ $\kappa(x)$ defined by
    \begin{align*}
        \sigma\Big(\frac{f(x)}{g(x)}\Big) &= \frac{f(1/x)}{g(1/x)},\\
        \tau\Big(\frac{f(x)}{g(x)}\Big) &= \frac{f(1-x)}{g(1-x)}.
    \end{align*}
    Determine the fixed field\/ $F$ of\/ $\set{\sigma,\tau}$ and\/ $\Gal(\kappa(x)/F)$. Find\/ $h\in F$ such that\/ $F=\kappa(h)$.
\end{probl}

\begin{solution}
    Let $\omega=\sigma\circ\tau$. Then
    \begin{align*}
        \omega(x) = 1-\frac1x,\quad
        \omega^2(x) = \sigma\Big(1-\frac1{1-x}\Big) = \frac1{1-x}
        \quad\text{and}\quad \omega^3(x)=x,
    \end{align*}
    i.e., $\ord(\omega)=3$. Hence, $\sigma$ and $\tau$ generate the dihedral group $D_6$ of~$6$ elements:
    $$
        \langle\sigma,\tau\rangle = \set{\id, \tau,\omega,\omega^2,\tau\omega=\omega^2\tau,\tau\omega^2=\omega\tau}.
    $$
    Applying these automorphisms to $x$ we get
    \begin{align*}
        a_0(x) &= x, &a_1(x) &= 1 - x, &a_2(x) &= \frac{x-1}x,\\
        a_3(x) &= \frac1{1-x}=a_1^{-1}, &a_4(x) &= \frac x{x-1} = a_2^{-1}, &a_5(x) &= \frac1x=a_0^{-1}.
    \end{align*}
    Let $K$ be the field fixed by the subgroup $\set{\id,\omega,\omega^2}$ generated by $\omega$. The sum
    $$
        (\id+\omega+\omega^2)(x) = x + \frac{x-1}x + \frac1{1-x}
    $$
    belongs to $K$. Moreover, if $\zeta = (\id+\omega+\omega^2)(x)$ then
    $$
        x(1-x)\zeta=x^2(1-x)-(x-1)^2+x = x^2 - x^3 - x^2 + 2x - 1 + x 
            = -x^3+3x-1,
    $$
    i.e., $x$ is a root of $g(t)\in K[t]$ defined as
    $$
        g(t) = t^3-3t+1+\zeta t(1-t) = t^3 - \zeta t^2 + (\zeta - 3)t + 1.
    $$
    Thus, $K=\kappa(\zeta)$ and $[\kappa(x),\kappa(\zeta)]=3$.

    Now observe that
    $$
        \zeta + \tau(\zeta) = x + \frac{x-1}x + \frac1{1-x}
                + 1 - x - \frac x{1-x} + \frac1x
            = 1,
    $$
    i.e., $\tau(\zeta)=1-\zeta$. Therefore,
    $$
        \zeta(1-\zeta)\in F.
    $$
    It follows that $h(z)=z^2-z+\zeta(1-\zeta)\in F[z]$ is the minimal of $\zeta$ over $F$. In particular, $[\kappa(\zeta):F]=2$. Hence, 
    $$
        [\kappa(x):F]=[\kappa(x):\kappa(\zeta)][\kappa(\zeta):F]=6.
    $$
    The polynomial
    \begin{align*}
        f(t) &= (t-a_0)(t-a_1)(t-a_2)(t-a_3)(t-a_4)(t-a_5)\\
            &= (t-a_0)(t-a_1)(t-a_2)(t-a_0^{-1})(t-a_1^{-1})(t-a_2^{-1})
    \end{align*}
    has coefficients in $F$, has degree $6$ and satisfies $f(x)=f(a_0)=0$. It is therefore the minimal of $x$ over $F$. Since the minimal has $6$ different roots, all in $\kappa(x)$, by Corollary~\ref{cor:galois-group-order-and-min-deg}, $\kappa(x)/F$ is Galois and so $|\Gal(\kappa(x)/F)|=6$, which implies that $\Gal(\kappa(x)/F)$ is the dihedral group of order $6$ generated by $\sigma$ and~$\tau$. If $\eta = \zeta(1-\zeta)$ then $\eta\in F$ and we have $\kappa(\eta)\subseteq F$. Equality is attained because $\min_{\kappa,\eta}(z)=z^2-z-\eta$ and so $[\kappa(\zeta):\kappa(\eta)]=2$.
\end{solution}

\begin{probl}
    Let $\kappa$ be a field, and let\/ $K = \kappa(x)$ be the rational function field in one variable over\/ $\kappa$. If\/ $u\in K$, show that\/ $K = \kappa(u)$ if, and only if,\/ $u = (ax + b)/(cx + d)$ for some\/ $a, b, c, d\in \kappa$ with
    \begin{equation}\label{eq:rational-det}
        \det \begin{pmatrix}
            a   &b\\
            c   &d
        \end{pmatrix}\ne0.
    \end{equation}
    %\textrm{\rm Hint: See the example before Proposition 1.15.}
\end{probl}

\begin{solution}
    Firstly recall that the map $x\mapsto(ax+b)/(cx+d)$ is not constant if, and only if, equation \eqref{eq:rational-det} does hold. Indeed,
    $$
        (ax_1+b)/(cx_1+d) = (ax_2+b)/(cx_2+d) \iff (ad-bc)(x_1-x_2)=0.
    $$
    \begin{description}
        \item[\rm\textit{if\/ }part:] Solving for $x$ we first get $cux+du=ax+b$, and then
        $$
            x = \frac{du-b}{-cu+a} \in \kappa(u).
        $$

        \item[\rm\textit{only if\/}:] Suppose that $x\in \kappa(u)$. Put $u=f(x)/g(x)$ with $f\perp g$. From Example~\ref{xmpl:k(t)/k(u)} we know that $[K:\kappa(u)]=\max\set{\deg f,\deg g}$. Therefore, both polynomials have degree bounded by $1$ and we must have
        $$
            u = \frac{ax+b}{cx+d}.
        $$
        Condition \eqref{eq:rational-det} must hold because $x\in\kappa(u)$ and so $u\notin\kappa$.
    \end{description}
\end{solution}

\begin{probl}
    Use the previous problem to show that any invertible\/ $2\times2$ matrix\/ $A=(a_{ij})$ determines an element of\/ $\Gal(\kappa(x)/\kappa)$ with
    $$
        x\mapsto(a_{11}x + a_{12})/(a_{21}x+a_{22}).
    $$
    Moreover, show that every element of\/ $\Gal(\kappa(x)/\kappa)$ is given by such a formula. Show that the map from the set of invertible\/ $2\times2$ matrices over\/ $\kappa$ to\/ $\Gal(\kappa(x)/\kappa)$ given by\/ $A\mapsto\varphi_A$, where\/ $\varphi_A(x) = (a_{11}x + a_{12})/(a_{21}x + a_{22})$, is a group morphism. Determine the  kernel to show that\/ $\Gal(\kappa(x)/\kappa)\cong\op{PGL}_2(\kappa)$, the group of invertible\/ $2\times2$ matrices over\/ $\kappa$ modulo the scalar matrices. 

    \textrm{\rm(This group is the \textsl{projective general linear group} over $\kappa$ of $2\times2$ matrices.)}
\end{probl}

\begin{solution}
    By definition,
    $$
        \varphi_A(1)=1,\quad\text{and}\quad
        \varphi_A(x)=\rho\circ A\circ\iota(x),
    $$
    where $\iota(x)=(x,1)\in\kappa_2$ and $\rho(s,t)=s/t$. Since $\rho\circ\iota=\id$, we get $\varphi_I=\id$, where $I$ is the identity matrix, and $\varphi_{AB}=\varphi_A\circ\varphi_B$. Thus, $\varphi_A\in\Gal(\kappa(x)/\kappa)$.

    To see that $A\mapsto\varphi_A$ is an epimorphism, take $\phi\in\Gal(\kappa(x)/\kappa)$ and let's verify that $\phi=\varphi_A$ for some invertible matrix $A$. Let $u=\phi(x)$. Since every element of $\kappa(x)$ is a quotient of two polynomials in $x$ and $\kappa(x)=\im\phi$, every element of $\kappa(x)$ is a quotient of two polynomials in $u$, i.e., $\kappa(x)=\kappa(u)$. By the previous problem, we deduce that $\phi(x)=u=\varphi_A(x)$ for some matrix $A$. Therefore, $\phi=\varphi_A$, as wanted.

    Finally, $A\in\ker(\varphi)$ if, and only if, $\rho\circ A\circ\iota(x)=x$ or
    \begin{align*}
        a_{11}x+a_{12} = x(a_{21}x+a_{22}),
    \end{align*}
    i.e., $a_{11}=a_{22}$ and $a_{12}=a_{21}=0$, which means that $A=a_{11}I$.
    
\end{solution}

\begin{probl}
    Let\/ $\kappa = \R$, and let\/ $A$ be the matrix
    $$
        \begin{pmatrix}
            -1/2    &\sqrt3/2\\
            -\sqrt3/2    &-1/2
        \end{pmatrix}
    $$
    given by rotating the plane around the origin by\/ $120^\circ$. Using the previous problem, show that\/ $A$ determines a subgroup of\/ $\Gal(\kappa(x)/\kappa)$ of order\/ $3$. Let\/ $F$ be the fixed field. Show that\/ $\kappa(x)/F$ is Galois, find a\/ $u$ so that\/ $F=\kappa(u)$, find the minimal polynomial\/ $\min_{F,x}$, and find all the roots of this polynomial.
\end{probl}

\begin{solution}
    With the notations of the preceding problem, we have $\ord(\varphi_A)=3$. Let $G=\langle\varphi_A\rangle$ be the subgroup of $\Gal(\kappa(x)/\kappa)$ generated by $\varphi_A$. By definition, $F=\mathcal F(G)$. From Corollary~\ref{cor:galois-group=dim} we obtain that $[\kappa(x):F]=3$ and $\Gal(\kappa(x)/F)=G$. In particular $\mathcal F(\Gal(\kappa(x)/F))=\mathcal F(G)=F$, i.e., $\kappa(x)/F$ is Galois.

    Since $A^3=I$, we know that $\min_A(x)\mid x^3-1=(x-1)(x^2+x+1)$. Hence, $\min_A(x)=x^2+x+1$. Consider the map
    $$
        u = (\varphi_{A^2} + \varphi_A + \id)(x).
    $$
    It satisfies
    $$
        \varphi_A(u)=(\varphi_{A^3}+\varphi_{A^2}+\varphi_A)(x)
            = u,
    $$
    i.e., $u\in F$. Since $A^2=A^T$, we get
    \begin{align*}
        u &= \frac{-x/2-\sqrt3/2}{\sqrt3x/2-1/2} +
            \frac{-x/2+\sqrt3/2}{-\sqrt3x/2-1/2} + x\\
            &= -\frac{x+\sqrt3}{\sqrt3x-1}
            +\frac{x-\sqrt3}{\sqrt3x+1}+x\\
            &= \frac{(\sqrt3x+1)(-x-\sqrt3)+(\sqrt3x-1)(x-\sqrt3)+3x^3-x}{3x^2-1}\\
            &= \frac{-\cancel{\sqrt3x^2}-(\sqrt3+1)x-\bcancel{\sqrt3}
                +\cancel{\sqrt3x^2}-4x+\bcancel{\sqrt3}
                +3x^3-x}{3x^2-1}\\
            &= \frac{3x^3-(\sqrt3+6)x}{3x^2-1}.
    \end{align*}
    It follows that $x$ is root of the polynomial
    $$
        f(t) = 3t^3-(\sqrt3+6)t-u(3t^2-1)\in F[t].
    $$
    Since $\deg(f)=3$, we deduce that $f(t)=\min_{F,x}(t)$. Moreover, given that $u\in F$ and $[\kappa(x):F]=[\kappa(x):\kappa(u)]$, we deduce that $F=\kappa(u)$. Finally, the other two roots of the minimal $f$ are $\varphi_A(x)$ and $\varphi_{A^2}(x)$.
\end{solution}

\begin{probl}\label{probl:Fp(u)}
    Define\/ $\varphi\colon\F_p(x)\to\F_p(x)$ by\/ $\varphi(x)=x+1$. Show that\/ $\varphi$ has finite order in\/ $\Gal(\F_p(x)/\F_p)$. Determine this order, find a\/ $u$ so that\/ $\F_p(u)$ is the fixed field of\/ $\varphi$, determine the minimal polynomial over\/ $\F_p(u)$ of\/ $x$, and find all the roots of this minimal polynomial. 
\end{probl}

\begin{solution}
    Induction on $n$ shows that $\varphi^n(x)=x+n$. In particular, $\varphi^p=\id$ and $\ord(\varphi)=p$. 
    Consider
    \begin{align*}
        u &= \prod_{i=0}^{p-1}x-i.
    \intertext{We have}
        \varphi(u) &= \prod_{i=0}^{p-1}x-(i-1)\\
            &= \prod_{j=-1}^{p-2}x-j  &&;\ j=i-1\\
            &= \prod_{j=0}^{p-1}x-j  &&;\ x-(-1) = x-(p-1)\\
            &= u,
    \end{align*}
    which shows that $u$ belongs to the fixed field of $\varphi$. Moreover, $x$ is a root of
    $$
        f(t)=u - \prod_{i=0}^{p-1}t-i\in\F_p(u)[t].
    $$
    Since this polynomial is primitive in $\F_p[t][u]$ and irreducible in $\F_p(t)[u]$, by Proposition~\ref{prop:primitive+irreducible=irreducible}, which can be applied because $u$ is transcendent over $\F_p$, it is irreducible in $\F_p(u)[t]$. In consequence, $\min_{\F_p(u),x}(t)=f(t)$. The other roots are $x+1,\dots, x+p-1$.

    The next problem generalizes this one.
\end{solution}

\begin{probl}\label{probl:k(x):k(f)}
    Let\/ $\kappa$ be a field of characteristic\/ $p>0$, and let\/ $a \in \kappa$. Let\/ $f(x)= x^p - a^{p-1}x$. Show that\/ $f$ is fixed by the automorphism\/ $\varphi$ of\/ $\kappa(x)$ defined by\/ $\varphi(f(x)/g(x)) = f(x+a)/g(x+a)$ for any\/ $f(x),g(x) \in \kappa[x]$. Show that\/ $\kappa(f)$ is the fixed field of\/~$\varphi$.
\end{probl}

\begin{solution}
    A straightforward verification shows that $f$ is fixed by $\varphi$
    \begin{align*}
        \varphi(f(x)) &= (x+a)^p-a^{p-1}(x+a)\\
            &= x^p + a^p - a^{p-1}(x+a)\\
            &= x^p - a^{p-1}x\\
            &= f(x).
    \end{align*}
    Consider the polynomial in $\kappa(f)[t]$
    $$
        h(t) = t^p - a^{p-1}t - f.
    $$
    Since $f$ is transcendent over $\kappa$, we can use Proposition~\ref{prop:primitive+irreducible=irreducible} to deduce that $h(t)$ is irreducible in $\kappa(f)[t]$. Indeed, $h$ is primitive in $\kappa[t][f]$ and irreducible in $\kappa(t)[f]$, as needed. It follows that $[\kappa(x):\kappa(f)]=p$. Let $F$ be the fixed field of $\varphi$. Then, $\kappa(f)\subseteq F$ and the equation 
    $$
        p=[\kappa(x):\kappa(f)]=[\kappa(x):F][F:\kappa(f)],
    $$
    implies that $[F:\kappa(f)]=1$.
\end{solution}

\section{Normal Extensions}

\begin{defn}
    Let $K$ be an extension of the field $\kappa$ and $f\in\kappa[x]$ a nonconstant polynomial. We say that $f$ \textsl{splits over} $K$ if $f$ decomposes as product of linear factors in $K[x]$.
\end{defn}

\begin{thm}
    Let\/ $f(x)\in\kappa[x]$ have degree\/ $n$. There is an extension field\/ $K$ of\/ $\kappa$ with\/ $[K:\kappa]\le n$ such that\/ $K$ contains a root of\/ $f$. In addition, there is a field\/ $L$ containing\/ $\kappa$ with\/ $[L:\kappa]\le n!$ such that\/ $f$ splits over\/~$L$. 
\end{thm}

\begin{proof}
    For the first part take $K=\kappa[x]/\gen{q(x)}$ where $q$ is an irreducible factor of~$f(x)$.

    The proof of the second part works by induction on $n$. It is trivial for $n=1$. For $n>1$ write $f(x)=(x-a)g(x)$, where $a\in K$ is a root whose existence we just proved. By induction, there is an extension $L$ of $K$ where $g$ splits that satisfies $[L:K]\le(n-1)!$. Then, $f$ splits over $L$ and
    $$
        [L:\kappa]=[L:K][K:\kappa]\le(n-1)!n=n!
    $$
\end{proof}

\begin{defn}
    Let\/ $S\subseteq\kappa[x]$ be a set of nonconstant polynomials. An extension\/ $K$ of\/ $\kappa$ is an \textsl{splitting field of} $S$ if\/ $K=\kappa(R)$, where\/ $R$ is the set of all roots of polynomials in\/~$S$. In a diagram, this will be indicated as follows
    $$
        \begin{tikzcd}
            K\\
            \kappa
                    \arrow[u,"\sfld_S"',no head,pos=0.8]
        \end{tikzcd}
    $$
    If\/ $f\in\kappa[x]$ is nonconstant, a \textsl{splitting field of}~$f$ is a splitting field of\/~$\set f$.
\end{defn}

\begin{cor}
    If\/ $S\subseteq\kappa[x]$ is finite, there is a splitting field of\/ $S$ over\/ $\kappa$.
\end{cor}

\begin{proof}
    Define $f=\prod S$. By the theorem, there is a splitting field of $f$, which is clearly a splitting field of $S$.
\end{proof}

\begin{xmpls}${}$
    \begin{enumerate}[a), font=\upshape]
        \item $\Q(e^{2\pi i/3},\sqrt[3]2)$ is a splitting field of $x^3-2$ over $\Q$ [cf.~Example~\ref{xmpls:minimal}~a)].

        \item $\C$ is a splitting field over $\R$ of $x^2+1$.

        \item In general, for any $a\in\kappa$, $\kappa(\sqrt a)$ is a splitting field of $x^2-a$.

        \item $\F_2[x]/\gen{x^2+x+1}$ is a splitting field of $x^2+x+1$ over $\F_2$ because if $\alpha$ is one of its roots, $\alpha+1$ is the other.
    \end{enumerate}
\end{xmpls}

\begin{prop}\label{prop:splitting-field-extension}
    Let\/ $f$ and\/ $g$ be polynomials in\/ $\kappa[x]$, with\/ $g\mid f$. Given a splitting field\/ $L$ of\/ $g$, there is an extension\/ $K$ of\/ $L$ such that\/ $K$ is an splitting field of\/ $f$. Conversely, if\/ $K$ is a splitting field of\/ $f$, there exists a subfield\/ $L\subseteq K$ such that\/ $L$ is a splitting field of\/ $g$.
\end{prop}

\begin{proof}
    Write $f(x)=g(x)q(x)$. For the first part, take an algebraic closure $F$ of $\kappa$ such that $L\subseteq F$. If $c_1,\dots,c_m$ are the roots of $q$ in $F$, then $L(c_1,\dots,c_m)$ is a splitting field of~$f$. Conversely, if $F$ is an algebraic closure of $K$ and $c_1,\dots, c_m, c_{m+1},\dots,c_n$ are the roots of $f$ in $F$, with $c_{m+1},\dots,c_n$ the roots of $g$, then $K=\kappa(c_1,\dots,c_n)$ and $\kappa(c_{m+1},\dots,c_n)$ is a splitting field of~$L$.
    
\end{proof}

\begin{prop}\label{prop:splitting-field-n!}
    Let\/ $f(x) \in \kappa[x]$ be a polynomial of degree\/ $n$. If\/ $K$ is a splitting field of\/ $f$ over\/ $\kappa$, then\/ $[K : \kappa] \leq n!$.
\end{prop}

\begin{proof}
    By induction on $n$. The case $n=1$ is trivial. If $n>1$, take a root $c$ of $f$ in $K$. Write $f(x)=(x-c)g(x)$ with $g(x)\in\kappa(c)[x]$. Then, 
    $$
        [\kappa(c):\kappa]\le\deg(f)=n\quad\text{and}\quad\deg(g)=n-1.
    $$
    Since $K$ is a splitting field of $g$ over $\kappa(c)$, the inductive hypothesis implies that $[K:\kappa(c)]\le (n-1)!$. In consequence, $[K:\kappa]\le(n-1)!n=n!$.
    
\end{proof}

\begin{xmpl}\label{xmpl:symmetric-polynomials}
    Let $K=\kappa(x_1,\dots,x_n)$ be the field of rational functions over $\kappa$. The permutation group $S_n$ can be seen as a subgroup of $\Gal(K/\kappa)$ via the inclusion $\sigma\mapsto \ev\sigma$, where $\ev\sigma(x_i)=\ev{\sigma(i)}$ for $1\le i\le n$.

    Let $F=\mathcal F(S_n)$. By Corollary~\ref{cor:galois-group=dim}, $S_n=\Gal(K/F)$ and $[K:F]=n!$. We claim that $F=\kappa(s_1,\dots,s_n)$, where $s_i$ is the $i$th elementary symmetric polynomial in $x_1,\dots,x_n$.

    Clearly $\kappa(s_1,\dots,s_n)\subseteq F$. The identity
    $$
        t^n-s_1t^{n-1}+\cdots+(-1)^ns_n=\prod_{i=1}^nt-x_i
    $$
    shows that $K$ is a splitting field of this polynomial over $\kappa(s_1,\dots,s_n)$. By Proposition~\ref{prop:splitting-field-n!}, $[K:\kappa(s_1,\dots,s_n)]\le n!=[K:F]$, proving the claim.
\end{xmpl}

\subsection{Algebraic Closures}

\begin{lem}\label{lem:closure-equivalences}
    If\/ $\kappa$ is a field, then the following statements are equivalent: 
    \begin{enumerate}[a), font=\upshape]
        \item There are no algebraic extensions of\/ $\kappa$ other than\/ $\kappa$ itself. 

        \item There are no finite extensions of\/ $\kappa$ other than\/ $\kappa$ itself. 

        \item If\/ $K$ is a field extension of\/ $\kappa$, then\/ $\kappa = \set{a\in K\mid a \text{ is algebraic over }\kappa}$. 

        \item Every\/ $f(x)\in \kappa[x]\setminus\kappa$ splits over\/ $\kappa$. 

        \item Every\/ $f(x)\in \kappa[x]\setminus\kappa$ has a root in\/ $\kappa$. 

        \item Every irreducible polynomial over\/ $\kappa$ has degree\/ $1$.
    \end{enumerate}
\end{lem}

\begin{proof}${}$
    \begin{enumerate}[a), font=\upshape]
        \item $\Rightarrow$ b) By Proposition~\ref{prop:algebraic-iff-finite} every finite extension is algebraic.

        \item $\Rightarrow$ c) If $a\in K$ is algebraic over $\kappa$, then $\kappa(a)$ is finite over $\kappa$, hence equal to~$\kappa$.

        \item $\Rightarrow$ d) Every root of $f$ is algebraic over $\kappa$, hence an elemente of $\kappa$.

        \item $\Rightarrow$ e) Trivial.

        \item $\Rightarrow$ f) Trivial.

        \item $\Rightarrow$ a) Trivial.
    \end{enumerate}
\end{proof}

\begin{defn}
    A field is \textsl{algebraically closed} if it satisfies any (and therefore all) of the conditions of the lemma. If $K$ is algebraically closed and algebraic over $\kappa$, we say that $K$ is \textsl{an algebraic closure} of~$\kappa$.
\end{defn}

\begin{lem}
    If\/ $K/\kappa$ is algebraic, then\/ $|K|\le\max\set{|\kappa|,\aleph_0}$. 
\end{lem}

\begin{proof}
    Let $\kappa[x]_n$ be the set of all polynomials of degree $n$. Then
    $$
        \kappa[x]\setminus\set0
            =\bigcup_{n=0}^\infty\kappa[x]_n.
    $$
    Hence,
    $$
        |\kappa[x]|
            =\Big|\bigcup_{n=1}^\infty\kappa^n\Big|
            \le\max\set{\aleph_0,|\kappa|}.
    $$
    Since every element of $K$ is a root of its minimal polynomial over $\kappa$, and every polynomial of degree $n$ has $n$ roots in $K$ at most, we deduce that $|K|\le |\kappa[x]\times\N|$ by mapping $a\mapsto(\min_{\kappa,a},i_a)$, where $i_a$ is an integer index identifying $a$ among the roots of the minimal polynomial in $K$, which can be achieved by giving $K$ a linear order. Then
    $$
        |K|\le |\kappa[x]\times\N|
            \le\max\set{\aleph_0,|\kappa|}\cdot\aleph_0
            =\max\set{\aleph_0,|\kappa|}.
    $$
\end{proof}

\begin{thm}
    Let\/ $\kappa$ be a field. Then\/ $\kappa$ has an algebraic closure. 
\end{thm}

\begin{proof}
    Let $S$ be a set whose cardinal is greater than $\max\set{\aleph_0,|\kappa|}$. We may assume that $\kappa\subseteq S$. Let $\mathcal K$ be the collection of algebraic extension of $\kappa$ whose underlying set is a part of $S$. Note that $\mathcal K$ is a set because the sum and product of each of these field structures are parts of $S\times S\times S$.

    Consider the partial order in $\mathcal K$ given by $K\preceq K'$ if, and only if, $K'/K$ is a field extension. Let $\mathcal F$ be a filtrant subfamily of $\mathcal K$. It's union $\bigcup\mathcal F$ is naturally an element of $\mathcal K$. By Zorn's lemma, there exists a maximal algebraic extension~$F$ of~$\kappa$ in $S$. We claim that $F$ is algebraically closed. Suppose that $L/F$ is an algebraic extension. By the lemma, $|L|\le\max\set{\aleph_0,|F|}$ and also $|F|\le\max\set{\aleph_0,|\kappa|}$. Therefore, $|L|\le\max\set{\aleph_0,|\kappa|}<|S|$. In particular, $|L\setminus F|\le|S\setminus F|$. Let $\phi\colon L\setminus F\to S\setminus F$ be an injective map. Define the injection
    \begin{align*}
        \varphi\colon L&\to S\\
        a&\mapsto\begin{cases}
            a   &\text{if }a\in F,\\
            \phi(a) &\text{otherwise}.
        \end{cases}
    \end{align*}
    Then $\varphi(L)\in\mathcal K$ with the structure of field transferred by $\varphi$. Since $F$ is maximal and $F\subseteq\varphi(L)$, we deduce that $\varphi(F)=F=\varphi(L)$. Therefore, $F=L$, proving the claim.
\end{proof}

\begin{cor}
    Every set of nonconstant polynomials over\/ $\kappa$ has a splitting field over\/ $\kappa$. 
\end{cor}

\begin{proof}
    Let $S$ be a set of nonconstant polynomials in $\kappa[x]$. Let $F$ be an algebraic closure of $\kappa$. Let $R\subset F$ be the set of all roots of polynomials in $S$. Then $\kappa(R)$ is a splitting field of $S$ over~$\kappa$.
\end{proof}

\begin{cor}
    If\/ $\kappa$ is a field, then the splitting field of the set of all nonconstant polynomials over\/ $\kappa$ is an algebraic closure of\/~$\kappa$.
\end{cor}

\begin{proof}
     Let $K$ be the splitting field of $\kappa[x]\setminus\kappa$ and $F$ an algebraic closure of $K$. If $c\in F$, $c$ is algebraic over $K$ and therefore over $\kappa$. Thus, $\pmin_{\kappa,c}[x]\in\kappa[x]$ splits over $K$. In particular, $c\in K$. Hence, $F\subseteq K$ and equality is attained.
\end{proof}

\begin{ntn}
    If\/ $\sigma\colon K\to K'$ is a morphism of fields, given a polynomial\/ $f\in K[x]$, we will denote by\/ $\sigma(f)$ the polynomial in\/ $K'[x]$ whose coefficients are the images by\/ $\sigma$ of the coefficients of\/~$f$. Note that $\sigma(fg)=\sigma(f)\sigma(g)$.
\end{ntn}

\begin{lem}\label{lem:algebraic-extension-1}
    Let\/ $\sigma\colon K\to K'$ be a field isomorphism. Let\/ $f(x)\in K[x]$ irreducible, $c$ a root of\/ $f$ in an extension of\/ $K$ and\/ $c'$ a root of\/ $\sigma(f)$ in an extension of\/ $K'$. Then, there exists a field isomorphism\/ $\bar\sigma\colon K(c)\to K'(c')$ such that\/~$\sigma\subseteq\bar\sigma$ and $\bar\sigma(c)=c'$.
\end{lem}

\begin{proof}
    Consider the following diagram
    $$
        \begin{tikzcd}
            K(c)
                    \arrow[d,"\cong"']
                    \arrow[r,"\bar\sigma",dashed]
                &K'(c')\\
            {K[x]/\gen{\,f\,}}
                    \arrow[r]
                &{K'[x]/\gen{\,\sigma(f)\,},}
                    \arrow[u,"\cong"']\\
            K\arrow[u]
                    \arrow[r,"\sigma"]
                &K'
                    \arrow[u]                          
        \end{tikzcd}
    $$
    where the horizontal arrow in the middle is induced by passing $\sigma$ to the quotient, and both vertical isomorphisms are respectively induced by mapping $x$ to $c$ and~$c'$.
\end{proof}

\begin{lem}\label{lem:algebraic-extension-2}
    Let\/ $\sigma\colon K \to K'$ be a field isomorphism,  $F$ a field extension of\/ $K$, and\/ $F'$ a field extension of\/ $K'$. Suppose that\/ $F$ is a splitting field of\/ $S\subseteq K[x]$ over\/ $K$ and that\/ $\omega\colon F\to F'$ is a morphism with\/ $\sigma\subseteq\omega$. If\/ $\sigma(S)=\set{\sigma(f)\mid f\in S}$, then\/ $\omega(F)$ is a splitting field of\/ $\sigma(S)$ over\/~$K'$. In a diagram
    $$
        \begin{tikzcd}
                &F'\\[-0.3cm]
            F
                    \arrow[ru,"\omega"]
                    \arrow[r]
                &\omega(F)
                    \arrow[u,no head]\\
            K
                    \arrow[u,"\sfld_S",no head,pos=0.8]
                    \arrow[r,"\sigma"]
                &K'
                    \arrow[u,"\sfld_{\sigma(S)}"',no head,pos=0.8]
        \end{tikzcd}
            $$
        \end{lem}

\begin{proof}
    Let $R\subseteq F$ be the set of roots of members of $S$. Take $f\in S$. Then
    $$
        f(x) = a(x-c_1)\cdots(x-c_n)
    $$
    for $a\in K$ and $c_1,\dots,c_n\in F$. It follows that
    $$
        \sigma(f) = \sigma(a)(x-\omega(c_1))\cdots(x-\omega(c_n)),
    $$
    with $\omega(c_1),\dots,\omega(c_n)\in\omega(R)$.  By hypothesis, $F=K(R)$. Therefore,
    $$
        \omega(F) = \omega(K)(\omega(R))
            = \sigma(K)(\omega(R))
            = K'(\omega(R)),
    $$
    which is the splitting field of $\sigma(S)$. 
\end{proof}

\begin{thm}
    Let\/ $\sigma\colon K\to K'$ be a field isomorphism, $f(x)\in K[x]$, and\/ $\sigma(f)\in K'[x]$ the corresponding polynomial over\/ $K'$. Let\/ $F$ be the splitting field of\/ $f$ over\/ $K$, and\/ $F'$ the splitting field of\/ $\sigma(f)$ over\/ $K'$. Then there is an isomorphism\/ $\omega\colon F\to F'$ with\/ $\sigma\subseteq\omega$. In a diagram
    $$
        \begin{tikzcd}
            F
                    \arrow[r,"\omega",dashed]
                &F'\\
            K
                    \arrow[r,"\sigma"]
                    \arrow[u,"\sfld_f",no head,pos=0.8]
                &K'
                    \arrow[u,"\sfld_{\sigma(f)}"',no head,pos=0.8]
        \end{tikzcd}
    $$
    Furthermore, if\/ $c\in F$ and if\/ $c'\in F'$ is any root of\/ $\sigma(\min_{K,c})$ in\/ $F'$, then\/ $\omega$ can be chosen so that\/ $\omega(c)=c'$.
\end{thm}

\begin{proof}
    First observe that if $f$ splits over $K$ then $\omega=\sigma$ fulfills the requirements of the theorem because $F=K$, $F'=K'$ and if $c$ is a root of $f$, then $\sigma(c)$ is the only root of $x-\sigma(c)=\sigma(\min_{K,c})$ in $K'$.
    
    By induction on $n=[F:K]$. In the case $n=1$ is trivial because, in that case, $f$ splits over $K$.
    
    In the case $n>1$ take a root $c$ of $f(x)$ in $F\setminus K$. Pick a nonlinear irreducible factor $q(x)\in K[x]$ of $f$ such that $q(c)=0$. Then $q(x)=\min_{K,c}(x)$. Let $c'$ be a root of $\sigma(q)$. Since $[K(c):K]>1$, we see that $[F:K(c)]<n$. By Lemma~\ref{lem:algebraic-extension-1}, there exists $\bar\sigma\colon K(c)\to K'(c')$ such that $\sigma\subseteq\bar\sigma$ and $\bar\sigma(c)=c'$. Now, the inductive hypothesis implies the existence of an isomorphism $\omega\colon F\to F'$ with $\bar\sigma\subseteq\omega$. Since $\sigma\subseteq\bar\sigma$, we see that $\sigma\subseteq\omega$ and $\omega(c)=c'$.
    
\end{proof}

\begin{thm}\label{thm:isomorphism-extension} {\rm [Isomorphism Extension Theorem]}
    Let\/ $\sigma \colon K \to K'$ be a field isomorphism. Let\/ $S$ be a set of polynomials over\/ $K$, and\/ $S' = \sigma(S)$ the corresponding set over\/ $K'$. Let\/ $F$ be a splitting field of\/ $S$ over\/ $K$, and\/ $F'$ a splitting field of\/ $S'$ over\/ $K'$. Then there is an isomorphism\/ $\omega \colon F \to F'$ with\/ $\sigma\subseteq\omega$.
    $$
        \begin{tikzcd}
            F
                    \arrow[r,"\omega",dashed]
                &F'\\
            K
                    \arrow[r,"\sigma"]
                    \arrow[u,"\sfld_S",no head,pos=0.8]
                &K'
                    \arrow[u,"\sfld_{S'}"',no head,pos=0.8]
        \end{tikzcd}
    $$
    Furthermore, if\/ $c \in F$ and\/ $c'$ is any root of\/ $\sigma(\min_{K,c})$ in\/ $F'$, then\/ $\omega$ can be chosen so that\/ $\omega(c) = c'$.
\end{thm}

\begin{proof}
    By Zorn's lemma. Fix $c\in F$ and a root $c'\in F'$ of $\sigma(\min_{K,c})$. Let $\Phi$ be the family of morphisms of fields $\phi\colon L\to F'$ such that
    \begin{enumerate}[\rm i)]
        \item $L=\dom\phi$ is a subfield of $F$,
        
        \item $c\in L$,
        
        \item $\phi(c)=c'$ and
        
        \item $\sigma\subseteq\phi$. 
    \end{enumerate}
    Since the elements of $\Phi$ are parts of $F\times F'$, $\Phi$ is a set. By Lemma~\ref{lem:algebraic-extension-1} we know that $\Phi$ is not empty because $\sigma$ can be extended to an isomorphism between $K(c)$ and $K'(c')$ that maps $c$ to $c'$.
    
    If $\mathcal F\subseteq\Phi$ is a filtrant part, then its union $\bigcup\mathcal F$ is a well defined map, whose domain is a subfield of $F$ and its image is included in $F'$. Since $\bigcup\mathcal F$ satisfies conditions i) to iv), we can pick a maximal element $\omega\colon L\to F'$ of $\Phi$. By hypothesis, $L=\dom\omega$ is a subfield of $F$. Suppose that $L\ne F$. Then, there would exist $f\in S$ such that $f$ doesn't split over $L$. Thus, the previous theorem would imply that the isomorphism between $L$ and $\omega(L)\subseteq F'$ induced by~$\omega$ could be extended to the splitting fields of $f$ and $\omega(f)$, which contradicts the maximality of $\omega$ because any such extension would also be a member of~$\Phi$.
    
\end{proof}

\begin{cor}\label{cor:interior-morphism-extension}
    Let\/ $F$ be the algebraic closure of\/ $\kappa$ and\/ $K\subseteq F$ an algebraic extension of\/ $\kappa$. Then, every\/ $\kappa$-morphism\/ $\sigma\colon K\to F$ can be extended to a\/ $\kappa$-morphism\/ $\omega\colon F\to F$. In a diagram
    $$
        \begin{tikzcd}
            F
                    \arrow[r,"\omega",dashed]
                &F\\
            K
                    \arrow[ru,"\sigma"]
                    \arrow[u,no head]\\
                &\kappa
                    \arrow[uu,"\text{alg.~closure}"',no head]
                    \arrow[lu,"\text{alg.}",no head]
        \end{tikzcd}
    $$
\end{cor}

\begin{proof}
    Since $F$ is also the algebraic closure of $K$, it is the splitting field over $K$ of $S=K[x]\setminus K$. The result is then a direct consequence of the theorem applied to the diagram, where the isomorphism is the coastriction of $\sigma$,
    $$
        \begin{tikzcd}
            F
                    \arrow[r,"\omega",dashed]
                &F\hphantom.\\
            K
                    \arrow[r,"\cong"']
                    \arrow[u,"\sfld_S",no head,pos=0.8]
                    \arrow[ur,"\sigma",color=gray]
                &\sigma(K).
                    \arrow[u,"\sfld_{\sigma(S)}"',no head,pos=0.8]
        \end{tikzcd}
    $$
    because $F$ is also the splitting field of $\sigma(S)$ over $\sigma(K)$.
\end{proof}

\begin{cor}\label{cor:isomorphic-split-fields}
    Let\/ $\kappa$ be a field, and let\/ $S$ be a subset of\/ $\kappa[x]$. Any two splitting fields of\/ $S$ over\/ $\kappa$ are\/ $\kappa$-isomorphic. In particular, any two algebraic closures of\/ $\kappa$ are\/ $\kappa$-isomorphic. In a diagram,
    $$
\begin{tikzcd}[column sep=tiny]
    K
            \arrow[rr,"{\exists\,\cong}",dashed,bend left]
        &&K'\\
        &\kappa
            \arrow[lu,"\sfld_S",no head,pos=0.9]
            \arrow[ru,"\sfld_S"',no head,pos=0.9]
\end{tikzcd}
    $$
\end{cor}

\begin{proof}
    Let $K$  and $K'$ be two splitting fields of $S$ over $\kappa$. By the theorem, $\id_\kappa\colon\kappa\to\kappa$ can be extended to an isomorphism $K\to K'$. The second statement is trivial because algebraic closures are splitting fields of $\kappa[x]\setminus\kappa$.
\end{proof}

\begin{cor}\label{cor:alg.closures-swalows-alg.}
    Let\/ $\kappa$ be a field and\/ $F$ an algebraic closure of\/ $\kappa$. If\/ $K$ is an algebraic extension of\/ $\kappa$, then\/ $K$ is isomorphic to a subfield of\/~$F$. In a diagram,
    $$
        \begin{tikzcd}
                &&F\\
            K
                    \arrow[r,"\exists\,\cong",dashed,bend left]
                &\textcolor{gray}{K'}
                    \arrow[ru,no head,dashed]\\
                &\kappa
                    \arrow[lu,"\text{\rm alg.}",no head]
                    \arrow[u,no head,dashed]
                    \arrow[ruu,"\text{\rm alg.~closure}"',no head]
        \end{tikzcd}
    $$
\end{cor}

\begin{proof}
    Let $F'$ be an algebraic closure of $K$. Since $F'$ is algebraically closed and algebraic over $\kappa$, we deduce that $F'$ is an algebraic closure of $\kappa$. By the previous corollary, there is a $\kappa$-isomorphism from $F'$ onto $F$. In particular, $K$ is $\kappa$-isomorphic to its image in $F$, as shown in the following diagram
        $$
        \begin{tikzcd}
            F'
                    \arrow[rr,"\cong"]
                &&F\\
            K
                    \arrow[r,"\cong",bend left]
                    \arrow[u,no head,"\text{\rm alg.~closure}"]
                &K'
                    \arrow[ru,no head]\\
                &\kappa
                    \arrow[lu,"\text{\rm alg.}",no head]
                    \arrow[u,no head,dashed]
                    \arrow[ruu,"\text{\rm alg.~closure}"',no head]
        \end{tikzcd}
    $$
\end{proof}


\subsection{Back to Normal}

\begin{defn}
    If\/ $K$ is a field extension of\/ $\kappa$, then $K$ is \textsl{normal} over\/ $\kappa$ if\/ $K$ is a splitting field of a set of polynomials over\/ $\kappa$. 
\end{defn}

\needspace{2\baselineskip}
\begin{xmpls}\label{xmpls:normal}${}$
\begin{enumerate}[a), font=\upshape]\label{xmpl1:normal}
    \item \textit{If\/ $[K:\kappa]=2$, then\/ $K$ is normal over\/ $\kappa$.}
    
    Indeed, given\/ $c\in K\setminus\kappa$, we must have\/ $K=\kappa(c)$. Thus, the minimal\/ $f(x)= \min_{\kappa,c}(x)$ is a polynomial of degree $2$ with one root in\/ $K$. Hence, $f$ factors over\/ $K$ and so\/ $K$ is a splitting field of\/ $f$ over\/~$\kappa$.

    \item \textit{If\/ $\kappa\subseteq K\subseteq F$ are fields such that\/ $F/\kappa$ is normal, then\/ $F/K$ is normal.}
    
    In fact, if $F=\kappa(S)$ is the splitting field of $S\subseteq\kappa[x]$ over $\kappa$ then $F=K(S)$ is also the splitting field of $S$ over~$K$.

    \item \textit{The field\/ $\Q(\omega,\sqrt[3]2)$, where $\omega=e^{2\pi i/3}$, is normal over\/ $\Q$}
    
    This is because $\Q(\omega,\sqrt[3]2)$ is the splitting field of $x^3-2$.

    \item \textit{Let\/ $\kappa$ be a field of characteristic\/ $p>0$, and suppose that\/ $K= \kappa(c_1,\dots,c_n)$ with\/ $c_i^p\in\kappa$ for each\/ $1\le i\le n$. Then\/ $K$ is normal over\/ $\kappa$.}
    
    To see this note that $K=\kappa(S)$ is the splitting field of $S=\set{\min_{\kappa,c_i}(x)\mid1\le i\le n}$ over $\kappa$ because $\min_{\kappa,c_i}(x)\mid x^p-c_i^p=(x-c_i)^p$ splits over $K$. Since $\min_{\kappa,c_i}(x)$ has only $c_i$ as a root, we deduce that $\Gal(K/\kappa)=\set\id$. A particular case of this example happens for $K=\kappa(x_1,\dots,x_n)/\gen{x_1^p,\dots,x_n^p}$, which is a normal extension over $\kappa$.
\end{enumerate}
\end{xmpls}

\begin{defn}
    Given a field extension $K/\kappa$, the \textsl{algebraic closure of} $\kappa$ in $K$ is the subfield of $K$ whose underlying set is
    $$
        F = \set{a\in K \mid a\text{\rm\ is algebraic over }\kappa}.
    $$
\end{defn}

\begin{thm}\label{thm:normal-equivalences}
    If\/ $K$ is algebraic over\/ $\kappa$, then the following statements are equivalent:
    \begin{enumerate}[a), font=\upshape]
        \item The field\/ $K$ is normal over\/ $\kappa$.
        
        \item If\/ $F$ is an algebraic closure of\/ $K$ and if\/ $\sigma \colon K \to F$ is a\/ $\kappa$-morphism, then\/ $\sigma(K) = K$.

        \item If\/ $\kappa \subseteq L \subseteq K \subseteq L'$ are fields and if\/ $\sigma \colon L \to L'$ is an\/ $\kappa$-morphism, then\/ $\sigma(L) \subseteq K$, and there is\/ $\omega \in \Gal(K/\kappa)$ with\/ $\sigma\subseteq\omega$.
        
        \item For any irreducible\/ $f(x) \in \kappa[x]$, if\/ $f$ has a root in\/ $K$, then\/ $f$ splits over\/~$K$.
    \end{enumerate}
\end{thm}

\begin{proof}${}$

    \textbf{Note:} For a direct proof of b) $\Rightarrow$ d) see Problems~\ref{probl:normal-equivalence-prep} and~\ref{probl:normal-equivalence}, where part~b) is incarnated in Corollary~\ref{cor:extension-to-normal}.
    
    \begin{enumerate}[a), font=\upshape]
        \item $\Rightarrow$~b) By Lemma~\ref{lem:algebraic-extension-2}, if $K$ is the splitting field of $S\subseteq\kappa[x]$ over $\kappa$, then $\sigma(K)$ is the splitting field of $\sigma(S)$ over $\kappa$. The conclusion follows because $\sigma(f)=f$ for all $f\in S$ (actually, for all $f\in\kappa[x]$).

        \item $\Rightarrow$~c) Consider the diagram
        $$
            \begin{tikzcd}
                F\\
                    &&L'\\
                \bar L
                        \arrow[uu,"\text{alg.~closure}",no head]
                    &&K
                        \arrow[u,no head]
                        \arrow[lluu,"\omega"',dashed,bend right=90]\\
                    &&L
                        \arrow[u,no head]
                        \arrow[u,no head]
                        \arrow[uu,"\sigma",bend left,pos=0.7]
                        \arrow[llu,no head]
                        \arrow[lluuu,"\text{alg.~closure}"',no head,pos=0.8]
                        \arrow[lluuu,"\bar\sigma",dashed,bend left=20]\\
                    &&\kappa
                        \arrow[u,no head]
                        \arrow[uu,"\text{alg.}"',no head,bend right]
            \end{tikzcd}
        $$
        where
        \begin{enumerate}[-]
            \item $\bar L$ is the algebraic closure of $L$ inside $L'$,
            \item $F$ is the algebraic closure of $\bar L$, hence of $K$ and $\kappa$,
            \item $\bar\sigma$ is the coastriction of $\sigma$ to $F$, which is valid because $\sigma(L)\subseteq L'$ is algebraic over $\kappa$
            \item $\omega$ is the extension of $\bar\sigma$ to $K$, whose existence is consequence of Corollary~\ref{cor:interior-morphism-extension}. Indeed, by the corollary, $\bar\sigma$ can be extended to $F$ and then restricted to $K$. Thus, part~b) implies that $\omega(K)=K$. In consequence,
            $$
                \sigma(L)=\bar\sigma(L)=\omega(L)\subseteq\omega(K)=K.
            $$
        \end{enumerate}
        Finally, the coastriction of $\omega$ to $K$ is an element of $\Gal(K/\kappa)$ whose restriction to $L$ coincides with~$\sigma$.

        \item $\Rightarrow$~d) Let $c$ be a root of $f$ in $K$ and take any other root $c'$ of $f$ in a algebraic closure $F$ of $K$. Consider the extensions $\kappa\subseteq\kappa(c)\subseteq K\subseteq L$, where $L$ is the splitting field of $f$ in $F$. By Lemma~\ref{lem:algebraic-extension-1} there exists a $\kappa$-isomorphism $\sigma\colon\kappa(c)\to\kappa(c')$ such that $\sigma(c)=c'$. The composition $\bar\sigma=\iota_{\kappa(c'),L}\circ\sigma$ with the inclusion $\kappa(c')\subseteq L$ puts us in the hypotheses of part~c). Therefore, $\bar\sigma(\kappa(c))\subseteq K$, which shows that $c'\in K$.

        \item $\Rightarrow$~a) According to part~d), if $c\in K$, then $\min_{\kappa,c}(x)$ splits over $K$. Therefore, $K$ is the splitting field of $\set{\min_{\kappa,c}\mid c\in K}$, which shows that $K$ is normal over~$\kappa$. \qedhere
    \end{enumerate}

\end{proof}

\begin{rem}
    As explained in the proof of d)~$\Rightarrow$~a) if $K$ is normal over $\kappa$, then $K$ is the splitting field over $\kappa$ of $\set{\min_{\kappa,c}\mid c\in K}$.
\end{rem}

\begin{cor}\label{cor:extension-to-normal}
    Let $K$ be a normal extension of $\kappa$. If $\kappa\subseteq L\subseteq K$ is an intermediate field, and $F$ is an algebraic closure of $K$, then every $\kappa$-morphism $\sigma\colon L\to L'$, where $L'\subseteq F$, can be extended to an isomorphism $\omega\colon K\to K$. In a diagram,
    \small
    $$
        \begin{tikzcd}[row sep=0.5cm]
                &F\\
            K
                    \arrow[r,dashed, "\omega"]
                &K
                    \arrow[u,no head]\\
            L
                    \arrow[r,"\sigma"]
                    \arrow[u,no head]
                &L'
                    \arrow[u,no head,dashed]
                    \arrow[uu,no head,bend right]\\
            \kappa
                    \arrow[u,no head]
                    \arrow[ur,no head]
        \end{tikzcd}    
    $$
    \normalsize
\end{cor}

\begin{proof}
    Corollary~\ref{cor:interior-morphism-extension} implies that the composition $L\stackrel\sigma\to L'\hookrightarrow F$ can be extended to a $\kappa$-morphism $\bar\sigma\colon F\to F$. By part~c) of the theorem, the restriction of $\bar\sigma$ to $K$ can be coastricted to a $\kappa$-morphism $\omega\colon K\to K$. Clearly, $\sigma\subseteq\omega$. In a diagram,
    \small
    $$
        \begin{tikzcd}[row sep=0.5cm]
            F
                    \arrow[r,dashed,"\bar\sigma"]
                &F\\
            K
                    \arrow[r,dashed, "\omega"]
                    \arrow[u,no head]
                &K
                    \arrow[u,no head]\\
            L
                    \arrow[r,"\sigma"]
                    \arrow[u,no head]
                &L'
                    \arrow[u,no head,dashed]
                    \arrow[uu,no head,bend right]\\
            \kappa
                    \arrow[u,no head]
                    \arrow[ur,no head]
        \end{tikzcd}    
    $$
    \normalsize
\end{proof}


\section{Problems}

\begin{probl}
    Show that\/ $K$ is a splitting field over\/ $\kappa$ of a set\/ $\{f_1,\dots,f_n\}$ of polynomials in\/ $\kappa[x]$ if, and only if, $K$ is a splitting field over\/ $\kappa$ of the single polynomial\/ $f_1\cdots f_n$.
\end{probl}

\begin{solution}
    The set of all roots of $\set{f_1,\dots,f_n}$ equals the set of roots of the product $f_1\cdots f_n$. Moreover, since the linear factorization of a polynomial implies the linear factorization of all its divisors, the linear factorization of the product is equivalent to the linear factorization of each of the~$f_i$.
\end{solution}


\begin{probl}
    Let\/ $K$ be a splitting field of a set\/ $S$ of polynomials over\/ $\kappa$. If\/ $L$ is a subfield of\/ $K$ containing\/ $\kappa$ for which each\/ $f \in S$ splits over\/ $L$, show that\/ $L = K$.
\end{probl}

\begin{solution}
    By definition, $K$ is generated by all the roots of polynomials in $S$. Since each of these polynomials splits over $L$, then all the roots are in $L$, hence $K\subseteq L$.
\end{solution}

\begin{probl}
    If\/ $\kappa \subseteq L \subseteq K$ are fields, and if\/ $K$ is a splitting field of\/ $S \subseteq \kappa[x]$ over $\kappa$, show that\/ $K$ is also a splitting field of\/ $S$ over\/ $L$.
\end{probl}

\begin{solution}
    By definition $K$ is generated by all the roots of polynomials in $S$, so we only need to show that these are the roots of the same polynomials when considered in $L[x]$. But they are, because a complete linear factorization in $K$ doesn't depend on the base field we are using to see the polynomial coefficients.
    
\end{solution}

\begin{probl}${}$
    \begin{enumerate}[a), font=\upshape]
        \item Let\/ $K$ be an algebraically closed field extension of\/ $\kappa$. Show that the algebraic closure of\/ $\kappa$ in\/ $K$ is an algebraic closure of\/ $\kappa$.
    
        \item If\/ $\A = \set{a \in \C \mid a \text{\rm\ is algebraic over } \Q}$ then, assuming that\/ $\C$ is algebraically closed, show that\/ $A$ is an algebraic closure of\/ $\Q$.
    \end{enumerate}
\end{probl}

\begin{solution}${}$
    \begin{enumerate}[a), font=\upshape]
        \item Let $F\subseteq K$ be the algebraic closure of $\kappa$ in $K$. In particular, $F$ is algebraic over $\kappa$. Take an irreducible polynomial $f(x)=a_nx^n+\cdots+a_0\in F[x]$. If $c\in K$ is a root of $f$, then $\kappa(a_0,\dots,a_n,c)$ is an algebraic extension of $\kappa$ included in $K$, hence in $F$.

        \item This is a direct consequence of part a).
    \end{enumerate}
\end{solution}

\begin{probl}
    Give an example of fields\/ $\kappa \subseteq K \subseteq F$ where $F/K$ and\/ $K/\kappa$ are normal but\/ $F/\kappa$ is not normal.
\end{probl}

\begin{solution}
    Take $\kappa=\Q$, $K=\Q(\sqrt2)$ and $F=\Q(\sqrt[4]2)=\Q[x]/\gen{x^4-2}$. Then both $F/K$ and $K/\kappa$ are normal because $[F:K]=[K:\kappa]=2$ [cf.~Example~\ref{xmpl1:normal}~a)]. However, $x^4-2=(x^2-\sqrt2)(x^2+\sqrt2)$, which has a root in $F$, doesn't include any of the (complex) roots of $x^2+\sqrt2$ [cf.~Theorem~\ref{thm:normal-equivalences}].
\end{solution}

\begin{probl}
    Let\/ $f(x)\in\kappa[x]$ be an irreducible polynomial 
    of degree\/ $n$, and let\/ $K$ be a field extension of\/ $\kappa$ with\/ $[K \colon \kappa] = m$. If\/ $n\perp m = 1$, show that\/ $f$ is irreducible over\/ $K$.
\end{probl}

\begin{solution}
    Let $q(x)\in K[x]$ be an irreducible factor of $f(x)$. Take an algebraic closure $F$ of $\kappa[x]/\gen{\,f\,}$. Since $K[x]/\gen{\,q\,}$ is algebraic over $\kappa$, by Corollary~\ref{cor:alg.closures-swalows-alg.}, there is an isomorphism of $K[x]/\gen{\,q\,}$ onto a subfield $K'$ of $F$. We have the following diagram
    $$
        \begin{tikzcd}
                &F\\
            K[x]/\gen{\,q\,}
                    \arrow[ur,no head]
                    \arrow[r,"\cong"]
                &K'
                    \arrow[u,no head]\\
            K
                \arrow[u,no head,"\deg(q)"]
                &\kappa[x]/\gen{\,f\,}
                    \arrow[u,no head,"d'"']\\
                &\kappa
                    \arrow[ul,no head,"m"]
                    \arrow[u,no head,"n"']
        \end{tikzcd}
    $$
    It follows that $\deg(q)m = d'n$, which implies that $n\mid\deg(q)$. In consequence, $\deg(f)=n\le\deg(q)\le\deg(f)$ because $q$ is a factor of $f$. Thus, $\deg(q)=\deg(f)$.
\end{solution}


\begin{probl}
    Show that\/ $x^5 - 9x^3 + 15x + 6$ is irreducible over\/ $\Q(\sqrt2, \sqrt3)$.
\end{probl}

\begin{solution}
    Let $f$ denote the given polynomial. Then $f$ is irreducible in $\Z[x]$ by Eisenstein criterion~\ref{lem:eisenstein}. Since it is monic, it is also irreducible in $\Q[x]$ (Gauss~\ref{lem:gauss-irreducibility}). The diagram
    $$
        \begin{tikzcd}[column sep=tiny]
            \Q(\sqrt2,\sqrt3)\\
            \Q(\sqrt2)
                    \arrow[u,no head,"2"']\\
            \Q
                    \arrow[u,no head,"2"']
        \end{tikzcd}
    $$
    shows that $[\Q(\sqrt2,\sqrt3):\Q]=4\perp5=\deg f$. The result is a direct consequence of the previous problem.
\end{solution}

\begin{probl}
    Find the degree of the splitting field of\/ $x^6 + 1$ over
    \begin{enumerate}[a), font=\upshape]
        \item $\Q$,
        \item $\mathbb{F}_2$.
    \end{enumerate}
\end{probl}

\begin{solution}
    \begin{enumerate}[a), font=\upshape]
        \item Firstly note that $i^6=(i^2)^3=-1$. Therefore, $x^2+1\mid x^6+1$. Since
        $$
            (y+1)(y^2-y+1)=y^3-\cancel{y^2}+\bcancel y+\cancel{y^2}-\bcancel y+1=y^3+1,
        $$
        To factor $x^4-x^2+1$ over $\Q(i)$ write
        \begin{align*}
            x^4-x^2+1 &= (x^2+ax+b)(x^2+cx+d)\\
                &= x^4 + (c+a)x^3 + (ac+d+b)x^2+(ad+cb)x+bd.
        \end{align*}
        Then
        \begin{align*}
            c+a &= 0,   &ac+d+b&=-1, &ad+cb&=0,  &bd&=1
        \end{align*}
        Therefore, if $a\ne0$, we have $d=b$, $b^2=1$ and $a^2-2b=1$. Hence, $a^2=\pm2 +1$ or $a^2\in\set{3,-1}$. Since $\sqrt3\notin\Q(i)$, we obtain $d=b=-1$ and $-c=a=\pm i$. Thus,
        $$
            x^4-x^2+1 = (x^2+ix-1)(x^2-ix-1),
        $$
        with roots
        $$
            \frac{-i\pm\sqrt3}2\quad\text{and}\quad\frac{i\pm\sqrt3}2.
        $$
        In conclusion, the splitting field of $x^6+1$ is $\Q(i,\sqrt3)$ with 
        $$
            [\Q(i,\sqrt3),\Q]=[\Q(i,\sqrt3):\Q(i)][\Q(i):\Q]=4.
        $$

        \item In this case $x^6+1=(x^3+1)^2$ and so, the question reduces to the polynomial $x^3+1=(x+1)(x^2+x+1)$, hence to $x^2+x+1$. Since this polynomial has degree $2$, Lemma~\ref{xmpls:normal}~a) implies that $\F_2[x]/\gen{x^2+x+1}$ is normal, i.e., a splitting field. In consequence, the degree of the splitting field of $x^6+1$ is~$2$. 
    \end{enumerate}
\end{solution}

\begin{probl}
    Determine the splitting field of $x^4 - 7$ over
    \begin{enumerate}[a), font=\upshape]
        \item $\Q$,
        \item $\F_5$,
        \item $F_{11}$.
    \end{enumerate}
\end{probl}

\begin{solution}
    \begin{enumerate}[a), font=\upshape]
        \item We have
        $$
            x^4-7 = (x^2-\sqrt7)(x^2+\sqrt7)
                =(x-\sqrt[4]7)(x+\sqrt[4]7)(x-\sqrt[4]7i)(x+\sqrt[4]7i).
        $$
        Therefore, the splitting field is $\Q(i,\sqrt[4]7)$, whose degree is
        $$
            [\Q(i,\sqrt[4]7):\Q]=[\Q(\sqrt[4]7)(i):\Q(\sqrt[4]7)][\Q(\sqrt[4]7):\Q]=2\cdot4=8.
        $$

        \item In $\F_5$ the polynomial becomes $x^4-2$ and we have $c^4=1$ for $c\ne0$. The factorization is
        $$
            x^4-2=(x-\sqrt[4]2)(x+\sqrt[4]2)(x-\sqrt[4]2i)(x+\sqrt[4]2i),
        $$
        where $i$ is one of the roots of $x^2+1$. Therefore, the splitting field is $\F_5(\sqrt[4]2,i)$. To see that this field has degree $8$ it is enough to show that $x^4-2$ is irreducible over $\F_5(i)$. Firstly observe that there is no element in $a+bi\in\F_5(i)$ satisfying $(a+bi)^4=2$. Indeed. Suppose otherwise and that squared norm to get
        $$
            (a^2+b^2)^4=2^2=-1,
        $$
        which is impossible in $\F_5$. Now assume that
        $$
            x^4-2=(x^2+ax+b)(x^2+cx+d)
        $$
        in $\F_5(i)[x]$. Then,
        $$
            a+c=0,\quad ac+d+b=0,\quad ad+bc=0,\quad bd=-2.
        $$
        If $a\ne0$, then $d=b$ and so $b^2=-2$. If $a=0$, then $c=0$, $d=-b$ and $b^2=2$. However, there is no element $b$ in $\F_5(i)$ with $b^2=\pm2$, because, taking squared norm that would yield $(N^2(b))^2=-1$ in $\F_5$, which is impossible.

        \item In $\F_{11}$ the polynomial becomes $x^4+4$. Moreover, we have
        $$
            \begin{tabular}{l|rrrr}
                 $x$&$2$ &$3$ &$4$ &$5$\\
                 $x^2$ &$4$ &$-2$ &$5$ &$3$\\
                 $x^4$ &$5$ &$4$ &$3$ &$-2$
            \end{tabular}
        $$
        So, there is no root in $\F_{11}$. Let's try a factorization
        $$
            x^4+4 = (x^2+ax+b)(x^2+cx+d).
        $$
        As above, we have
        $$
            a+c=0,\quad ac+d+b=0,\quad ad+bc=0,\quad bd=-2.
        $$
        Let's try the case $a\ne0$. Then $b=d$ and $b^2=-2$, which leads to $d=b=\pm3$. Hence, $-a^2+2b=0$ and so $a^2=\pm6$. The only possibility left is $a=\pm4$, $b=d=-3$, i.e.,
        $$
            x^4+4=(x^2+4x-3)(x^2-4x-3).
        $$
        So, the roots are
        $$
            \frac{-4\pm\sqrt{5+1}}2\quad\text{and}\quad
            \frac{4\pm\sqrt{5+1}}2,
        $$
        i.e.,
        $$
            -2\pm6\sqrt6\quad\text{and}\quad2\pm6\sqrt6.
        $$
        In consequence, to get all $4$ roots it is enough to inject $\sqrt6$. So, the splitting field is $\F_{11}(\sqrt6)=\F_{11}[x]/\gen{x^2+5}$, which has degree~$2$.
    \end{enumerate}
\end{solution}

\begin{probl}
    Let\/ $\kappa$ be a field, and let\/ $f(x) \in \kappa[x]$ be a polynomial of prime degree. Suppose for every field extension\/ $K$ of\/ $\kappa$ that if\/ $f$ has a root in\/ $K$, then\/ $f$ splits over\/ $K$. Prove that either\/ $f$ is irreducible over\/ $\kappa$ or\/ $f$ has a root (and hence splits) in\/ $\kappa$.
\end{probl}

\begin{solution}
    Consider the case where $f$ has no root in $\kappa$. Suppose that $f$ is reducible over $\kappa$, and let $q(x)\mid f(x)$ be a prime factor over $\kappa[x]$. By hypothesis $\kappa[x]/\gen{q(x)}$ is a splitting field of $f$ over $\kappa$. Take any other irreducible polynomial $r(x)\mid f(x)$, Since $\kappa[x]/\gen{r(x)}$ is also an splitting field of $f$ over $\kappa$, by Corollary~\ref{cor:isomorphic-split-fields} there is an isomorphism between $\kappa[x]/\gen{q(x)}$ and $\kappa[x]/\gen{r(x)}$. In particular $\deg(q)=\deg(r)$, i.e., all prime factors of $f$ have the same degree. Therefore, from an irreducible factorization
    $$
        f =\prod_{i=1}^mq_i^{e_i}
    $$
    we obtain $\deg(f)=\sum_{i=1}^me_id=kd$, where $d$ is the common degree of all $q_i$. Since $d>1$ and $\deg(f)$ is prime, we deduce that and $m=1$ and $e_1=1$.
    
\end{solution}

\begin{probl}\label{probl:irreducibility-criterion}
    Show that the hypotheses of the previous problem hold for 
    \begin{enumerate}[a), font=\upshape]
        \item $f(x) = x^p - a$, where\/ $\fchar(\kappa) = p$ and\/ $a \in \kappa$.
        \item $f(x) = x^p - x - a$, where\/ $\fchar(\kappa) = p$ and\/ $a \in \kappa$.
        \item $f(x) = x^p - a$, where\/ $\fchar(\kappa) \ne p$ and\/ $\kappa$ contains an element\/ $\omega$ 
        with\/ $\omega^p = 1$ and\/ $\omega \ne 1$.
    \end{enumerate}
\end{probl}

\begin{solution}
    \begin{enumerate}[a), font=\upshape]
        \item If $K$ is an extension of $\kappa$ and $c\in K$ satisfies $c^p=a$, then $x^p-a=x^p-c^p=(x-c)^p$, which shows that $f$ splits over $K$.

        \item Suppose that $c\in K$ satisfies $c^p-c-a=0$. Then
        $$
            f(x)=x^p-x - c^p + c = (x-c)^p - (x-c). 
        $$
        Therefore, any other root $\alpha$ of $f$ satisfies
        $$
            (\alpha-c)^p - (\alpha-c)=0,
        $$
        i.e., $\alpha-c$ is a root of $x^p-x$. But the $p$ roots of this polynomial are the $p$ elements of $\F_p\subseteq\kappa$. Therefore, $\alpha-c\in\kappa\subseteq K$, and in particular $\alpha\in K$.

        \item Let $c\in K$ be a root of $x^p-a$. Note that $\omega^i\ne1$ for $1\le i<p$ because the group $\set{1,\omega,\dots,\omega^{p-1}}$ of $p$ elements has, necessarily, order $p$. It follows that $c,c\omega,\dots,c\omega^{p-1}$ are the $p$ roots of the polynomial, all in $K$.
    \end{enumerate}
\end{solution}

\begin{probl}
    Let\/ $K$ be a field, and suppose that\/ $\sigma \in \operatorname{Aut}(K)$ has infinite order. Let\/ $\kappa$ be the fixed field of\/ $\sigma$. If\/ $K / \kappa$ is algebraic, show that\/ $K$ is normal over\/ $\kappa$.
\end{probl}

\begin{solution}
    Take an irreducible polynomial $f(x)\in\kappa[x]$ with a root $c\in K$. Note that $f=\min_{\kappa,c}$. By Theorem~\ref{thm:normal-equivalences}, it suffices to show that $f$ splits over $K$. Since $\sigma(f)=f$, we see that $f(\sigma(c))=\sigma(f(c))=0$, i.e., $\sigma(c)$ is a root of $f$. Iterating we deduce that $\set{\sigma^i(c)\mid i\ge0}\subseteq K$ are roots of $f$. Therefore, there exists $j$, $0<j<\deg(f)$, such that $\sigma^j(c)=c$. Consider the polynomial
    $$
        q(x) = (x-c)(x-\sigma(c))\cdots(x-\sigma^{j-1}(c)).
    $$
    Clearly, $\sigma(q)=q$. Thus, $q\in\kappa[x]$. But $q(c)=0$, and so $f(x)=\min_{\kappa,c}(x)=q(x)$, which visibly splits over $K$.

    \textbf{Note:} The solution makes no use of the hypothesis $\ord(\sigma)=\infty$.
\end{solution}

\begin{probl}\label{probl:normal-equivalence-prep}
    Let\/ $K$ be a normal extension of\/ $\kappa$, and let\/ $f(x) \in \kappa[x]$ be an irreducible polynomial over\/ $\kappa$. Let\/ $q_1(x)$ and\/ $q_2(x)$ be monic irreducible factors of\/ $f(x)$ in\/ $K[x]$. Prove that there exists\/ $\sigma \in \Gal(K / \kappa)$ with\/ $\sigma(q_1) = q_2$.
\end{probl}

\begin{solution}
    Take an algebraic closure $\bar K$ of $K$. Put $F=\mathcal F(\Gal(K/k))$. Write a factorization $f(x)=q_1(x)q_2(x)\cdots q_m(x)$ of $f(x)$ in $K[x]$, with possible repetitions. Given $\sigma\in\Gal(K/k)$, $\sigma(q_1)$ is a monic irreducible factor of $f$ in $K[x]$, i.e., there exists $j$ such that $\sigma(q_1)=q_j$. Suppose that such $j$ is never $2$. Then the product $q(x)$ of all $q_j(x)\ne q_2(x)$ would remain fixed by $\Gal(K/\kappa)$, i.e., $q(x)\in F[x]$, with $f(x)=q(x)q_2^e(x)$. Since $q_2(x)\perp q(x)$, if $c\in\bar K$ is a root of $q_2(x)$, then $q(c)\ne0$. In particular, for $\sigma\in\Gal(K/\kappa)$, we have
    $$
        q(\sigma(c)) = \sigma(q)(\sigma(c))=\sigma(q(c))\ne0.
    $$
    Given that $q(\sigma(c))q_2^ e(\sigma(c))=f(\sigma(c))=0$, we get $q_2(\sigma(c))=0$, i.e., $\sigma$ permutes the roots of $q_2(x)$. Thus, $\sigma(q_2)=q_2$. As a result, $q_2(x)\in F[x]$.

    Let $c'$ be a root of $q(x)$ in $\bar K$. Since both $c$ and $c'$ are roots of $f(x)$, by Lemma~\ref{lem:algebraic-extension-1}, there exists a morphism $\theta\colon\kappa(c')\to\kappa(c)$ satisfying $\theta(c')=c$. Corollary~\ref{cor:extension-to-normal} implies that $\theta$ can be extended to $\sigma\colon K\to K$. However, we cannot have $\sigma(q)=q$, because that would imply $q(c)=q(\sigma(c'))=\sigma(q(c'))=0$.
    
\end{solution}

\begin{probl}\label{probl:normal-equivalence}
    Let\/ $K$ be a normal extension of\/ $\kappa$, and let\/ $f(x)$ be an irreducible polynomial in\/ $\kappa[x]$. If\/ $f$ is not irreducible over\/ $K$, show that\/ $f$ factors over\/ $K$ into a product of irreducible polynomials of the same degree. In particular, if\/ $f$ has a root in\/ $K$, then\/ $f$ splits over\/~$K$. 
\end{probl}

\begin{solution}
    This is a direct consequence of the previous problem because the elements of $\Gal(K/\kappa)$ permute the factors of $f$ in $K[x]$. The second statement is, in turn, a direct consequence of the first because a root in $K$ corresponds to a factor of degree~$1$.
\end{solution}

\begin{probl}
    Let\/ $K$ and\/ $L$ be extensions of\/ $\kappa$. Show that\/ $KL$ is normal over\/ $\kappa$ if both\/ $K$ and\/ $L$ are normal over\/ $\kappa$. Is the converse true? 
\end{probl}

\begin{solution}
    According to part~b) of Theorem~\ref{thm:normal-equivalences} it is enough to show that every $\kappa$-morphism $\sigma\colon KL\to F$, where $F$ is an algebraic closure of $\kappa$, satisfies $\sigma(KL)=KL$. But this is a direct consequence of the same theorem applied to $K$ and $L$ and the restrictions of $\sigma$ to these fields. The converse is not true as shown by the example $\Q(\omega,\sqrt[3]2)=\Q(\omega)\Q(\sqrt[3]2)$.
\end{solution}

\begin{probl}
    Let\/ $K$ be a normal extension of\/ $\kappa$. Suppose that\/ $c, c' \in K$ are roots of\/ $\min_{\kappa,c}$ and that\/ $b, b'$ are roots of\/ $\min_{\kappa,b}$. Determine whether or not there is an automorphism\/ $\sigma\in\Gal(K/\kappa)$ with\/ $\sigma(c) = c'$ and\/ $\sigma(b) = b'$. 
\end{probl}

\begin{solution}
    First note that, in case such an isomorphism does exist, both $b$ and $b'$ must belong to $K$. Therefore, membership in $K$ is clearly required.

    Both minimal polynomials are (1)~equal, or (2)~coprime.
    
    From Lemma~\ref{lem:algebraic-extension-1} we know that there exists $\theta\colon\kappa(c)\to\kappa(c')$ with $\theta(c)=c'$. By Corollary~\ref{cor:extension-to-normal}, there exists $\sigma\colon K\to K$ such that $\theta\subseteq\sigma$. Let $T$ be the set of all extensions of $\theta$ to $K$, and let $F=\mathcal F(T)$ its fixed field. If no element of $T$ maps $b$ to $b'$, then $\min_{F,b}(x)$ satisfies $\min_{F,b}(b')\ne0$. In particular, $\min_{F,b}(x)$ is a proper factor of $\min_{\kappa,b}(x)$.

    ???
    
\end{solution}

\section{Separable and Inseparable Extensions}

\begin{defns}${}$
    \begin{enumerate}[-]
        \item Let $f(x) \in \kappa[x]$. A root $\alpha$ of $f$ has \textsl{multiplicity} $m$ if $(x - \alpha)^m$ divides $f(x)$ but $(x - \alpha)^{m+1}$ does not divide $f(x)$. If $m > 1$, then $\alpha$ is called a \textsl{repeated root} of $f(x)$.
    
        \item An irreducible polynomial $f(x) \in \kappa[x]$ is \textsl{separable} over $\kappa$ if $f$ has no repeated roots in any splitting field. A polynomial $g(x) \in \kappa[x]$ is \textsl{separable} over $\kappa$ if all irreducible factors of $g(x)$ are separable over $\kappa$.
    \end{enumerate}
\end{defns}

\begin{rem}
    By Corollary~\ref{cor:isomorphic-split-fields}, if $f$ has no repeated roots in a given splitting field, then it has no repeated roots in any splitting field, i.e., the separability (or not) of a polynomial doesn't depend on the splitting field of choice.
\end{rem}

\begin{xmpls}${}$
    \begin{enumerate}[a), font=\upshape]
        \item The polynomial $x^2 - 2$ is separable over $\Q$, as is $(x - 1)^9$.
        
        \item The polynomial $x^2 + x + 1$ is separable over $\F_2$, since we saw in Example~\ref{xmpls:minimal}~c) that if $\alpha$ is a root, then so is $\alpha + 1$.
        
        \item Suppose that $\fchar(\kappa) = p$ and $\alpha \in \kappa\setminus\kappa^p$. Then $x^p - \alpha$ is irreducible over $\kappa$ [cf.~Problem~\ref{probl:irreducibility-criterion}], but it is not separable over $\kappa$, since it has at most one root in any extension field of $\kappa$. Note that if $\alpha$ is a root of $x^p-a$, then $x^p-a$ is separable over $\kappa(\alpha)$.
    \end{enumerate}
\end{xmpls}

\begin{lem}\label{lem:separable-polynomials}
    Let\/ $f(x)$ and\/ $g(x)$ be polynomials over a field\/ $\kappa$.
    \begin{enumerate}[a), font=\upshape]
        \item If\/ $f$ has no repeated roots in any splitting field, then\/ $f$ is separable over\/~$\kappa$.
        \item If\/ $g$ divides\/ $f$ and if\/ $f$ is separable over\/ $\kappa$, then\/ $g$ is separable over\/ $\kappa$.
        \item If\/ $f_1, \dots, f_n$ are separable polynomials over\/ $\kappa$, then the product\/ $f_1 \cdots f_n$ is separable over\/ $\kappa$.
        \item If\/ $f$ is separable over\/ $\kappa$, then\/ $f$ is separable over any extension field of\/ $\kappa$.
    \end{enumerate}
\end{lem}

\needspace{2\baselineskip}
\begin{proof}${}$
    \begin{enumerate}[a), font=\upshape]
        \item In the case where $f$ is irreducible, this is exactly the definition of separable polynomial. Otherwise, if $q(x)$ is an irreducible factor with a root $c$ in a splitting field $K$ of $f$, then $c$ is also a root of $f$ which cannot be repeated in $f$, hence in $q$.

        \item Every irreducible factor $q$ of $g$ is an irreducible factor of $f$. Therefore, there is no repeated root in any splitting field of~$q$.

        \item Every irreducible factor of $f_1\cdots f_n$ is separable because it is an irreducible factor of some $f_i$.

        \item Let $K$ be an extension of $\kappa$. If $r(x)\in K[x]$ is an irreducible factor of $f$ over $K$, then $r$ is an irreducible factor of some irreducible factor $q(x)\in\kappa[x]$. Since $f$ is separable, $q$ has no repeated roots. Therefore, $r$ has no repeated roots either and so it is separable by part~a). %\qedhere
    \end{enumerate}    
\end{proof}

\begin{defn}
    If\/ $f(x) = a_0 + a_1x + \dots + a_nx^n \in\kappa[x]$, then the \textsl{formal derivative\/} $f'(x)$\/ is\/ defined\/ by\/ $f'(x) = a_1 + 2a_2x + \dots + na_nx^{n-1}$.
\end{defn}

\begin{prop}
    If\/ $f$ and\/ $g$ are polynomials in $\kappa[x]$ and $a,b\in\kappa$, then
    \begin{enumerate}[a), font=\upshape]
        \item $(af(x) + bg(x))' = af'(x) + bg'(x)$;
        \item $(f(x)g(x))' = f'(x)g(x) + f(x)g'(x)$;
        \item $(f\circ g)'(x) = f'(g(x))g'(x)$.
    \end{enumerate}
\end{prop}

\needspace{2\baselineskip}
\begin{proof}${}$
    \begin{enumerate}[a), font=\upshape]
        \item By definition, the formal derivative is $\kappa$-linear.
        \item By part a) it suffices to verify the identity for $f(x)=x^n$ and $g(x)=x^m$,
        %\small
        \begin{align*}
            (fg)'(x) &= (n+m)x^{n+m-1}\\
                &= nx^{n-1}x^m+mx^nx^{m-1}\\
                &= f'(x)g(x)+f(x)g'(x).
        \end{align*}
        \normalsize
        
        \item By part a) it suffices to consider the case $f=x^n$ where $(f\circ g)(x)=g(x)^n$. By induction on $n$, the case $n=0$ is trivial because $f'=0$. The case $n=1$ holds because $f\circ g=g$ and $f'=1$. If $n>1$, we have
        %\small
        \begin{align*}
            g^n(x)' &= (g^{n-1}g)'(x)\\
                &= (g^{n-1}(x))'g(x)+g^{n-1}(x)g'(x)\\
                &= (n-1)g^{n-2}(x)g'(x)g(x) + g^{n-1}(x)g'(x)
                    &&\text{; IH}\\
                &= ((n-1)g^{n-1}(x)+g^{n-1}(x))g'(x)\\
                &= ng^{n-1}(x)g'(x)\\
                &= f'(g(x))g'(x).
        \end{align*}
        \normalsize
    \end{enumerate}
\end{proof}

\begin{prop}\label{prop:f-perp-f'}
    Let\/ $f(x) \in\kappa[x]$ be a nonconstant polynomial. Then\/ $f$ has no repeated roots in a splitting field if, and only if, $f\perp f'$ in\/ $\kappa[x]$.
\end{prop}

\needspace{2\baselineskip}
\begin{proof}${}$
    \begin{description}
        \item[\rm\textit{only if\/}:] Suppose, toward a contradiction, that $f$ and $f'$ have a common irreducible factor $q(x)$ in $\kappa[x]$. Let $K$ be a splitting field of $\set{f,f'}$. Take a root $c$ of $q(x)$ in $K$. Then, $f(c)=f'(c)=0$. Put $c_1=c$ and let $c_2,\dots,c_n$ the remaining roots of $f$ in $K$. We have
        $$
            f(x) = (x-c_1)\cdots(x-c_n).
        $$
        Therefore,
        $$
            f'(x) = \sum_{i=1}^n\prod_{j\ne i}x-c_j.
        $$
        Evaluating at $c_1$,
        $$
            0 = f'(c) = f'(c_1)= \prod_{j\ne1}c_1-c_j,
        $$
        which implies that $c_j=c$ for some $j\ne1$, i.e., $c$ is a repeated root of $f$, in contradiction with the hypothesis on $f$.
        
        \item[\rm\textit{if\/} part:] If $f$ had a repeated root $c$ in any splitting field $K$ of $f$, we could write $f(x)=(x-c)^2q(x)$ for some $q\in K[x]$. Therefore,
        $$
            f'(x) = 2(x-c)q(x) + (x-c)^2q'(x)
        $$
        would satisfy $f'(c)=0$. However, this is not possible if $f\perp f'$ because in that case there exist polynomials $s(x)$ and $t(x)$ in $\kappa[x]$ such that $1=s(x)f(x)+t(x)f'(x)$, which prevents the existence of any common root between $f$ and $f'$.
        
    \end{description}
    
\end{proof}

\begin{prop}\label{prop:irreducible-separable-char-p}
    Let\/ $f(x) \in \kappa[x]$ be an irreducible polynomial.
    \begin{enumerate}[a), font=\upshape]
        \item If\/ $\fchar(\kappa) = 0$, then\/ $f$ is separable over\/ $\kappa$. If\/ $\fchar(\kappa) = p > 0$, then\/ $f$ is separable over\/ $\kappa$ if, and only if, $f'(x) \neq 0$, and this occurs if, and only if, $f(x) \not\in \kappa[x^p]$.
        
        \item If\/ $\fchar(\kappa) = p$, then\/ $f(x) = g(x^{p^m})$ for some integer\/ $m \ge 0$ and some\/ $g(x) \in \kappa[x]$ that is irreducible and separable over\/ $\kappa$.
    \end{enumerate}
\end{prop}

\needspace{2\baselineskip}
\begin{proof}${}$
    \begin{enumerate}[a), font=\upshape]
        \item Suppose that $\fchar(\kappa)=0$. Then $f'\ne0$ and since $\deg(f')<\deg(f)$, we deduce that $f\perp f'$. By the previous proposition, $f$ has no repeated roots in any splitting field, i.e., it is separable. In the case $\fchar(\kappa)=p>0$, the same argument shows that $f$ is separable if, and only if, $f'\ne0$, which is clearly equivalent to $f\notin\kappa[x^p]$.

        \item If $f$ is separable, the conclusion holds for $m=0$ and $g=f$. Otherwise, $f\in\kappa[x^p]$. Let $m$ be the maximum integer such that $f\in\kappa[x^{p^m}]$. Write $f=g(x^{p^m})$ and let's see that $g$ is irreducible. If $g(x)=u(x)v(x)$, then $f(x)=u(x^{p^m})v(x^{p^m})$. From the irreducibility of $f$ we deduced that say, $u(x^{p^m})\in\kappa$, which can only happen if $u\in\kappa$. Note finally, that the maximality of $m$ ensures that $g\notin\kappa[x^p]$, which means that $g$ is separable by part~a).
    \end{enumerate}
\end{proof}

\begin{defn}
    Let\/ $K$ be an extension field of\/ $\kappa$ and let\/ $c \in K$. Then\/ $c$ is \textsl{separable} over\/ $\kappa$ if\/ $\min_{\kappa,c}$ is separable over\/ $\kappa$. If every\/ $c \in K$ is separable over\/ $\kappa$, then\/ $K$ is \textsl{separable} over\/ $\kappa$.
\end{defn}

\begin{xmpls}${}$
    \begin{enumerate}[a), font=\upshape]
        \item If $\kappa$ is a field of characteristic 0, then any algebraic extension of $\kappa$ is separable over $\kappa$, since every polynomial in $\kappa[x]$ is separable over $\kappa$.

        \item If $\kappa$ is a field of characteristic $p > 0$ and if $\kappa(x)$ is the rational function field in one variable over $\kappa$, then the extension $\kappa(x)/\kappa(x^p)$ is not separable because $\min_{\kappa(x^p), x}(t) = t^p - x^p$, which has only $x$ as a root.
    \end{enumerate}
\end{xmpls}

\begin{thm}\label{thm:galois-equivalences}
    Let\/ $K$ be an algebraic extension of\/ $\kappa$. Then the following statements are equivalent:
    \begin{enumerate}[a), font=\upshape]
        \item $K$ is Galois over\/ $\kappa$.
        \item $K$ is normal and separable over\/ $\kappa$.
        \item $K$ is a splitting field of a set of separable polynomials over\/ $\kappa$.
    \end{enumerate}
\end{thm}

\begin{proof}${}$
    \begin{enumerate}[a), font=\upshape]
        \item $\Rightarrow$ b) Take $c\in K$ and consider the set
        $$
            Z = \set{\sigma(c)\mid \sigma\in\Gal(K/\kappa)}.
        $$
        Given that the elements of $Z$ are roots of $\min_{\kappa,c}$, we see that $|Z|<\infty$. Therefore, we can form the polynomial
        $$
            f(x) = \prod_{\zeta\in Z}x-\zeta,
        $$
        which belongs to $\kappa[x]$ because $\sigma(f)=f$ and $K/\kappa$ is Galois. In consequence $\min_{\kappa,c}\mid f$, which shows that $\min_{\kappa,c}$ splits over $K$ and is separable. Thus, $K$ is separable by definition and normal because it is the splitting field of $\set{\min_{\kappa,c}\mid c\in K}$.

        \item $\Rightarrow$ c) Since $K$ is normal, Theorem~\ref{thm:normal-equivalences}~c) implies that $K$ is the splitting field of the set $\set{\min_{\kappa,c}\mid c\in K}$.

        \item $\Rightarrow$ a) Let $a\in K$ be fixed by every element of $\Gal(K,\kappa)$. We have to show that~$a\in\kappa$.
        
        Let's first consider the case where $K$ is an algebraic extension of~$\kappa$. Since $K$ is normal, Theorem~\ref{thm:normal-equivalences} implies that all the roots of $\min_{\kappa,a}$ are in $K$. Let $b$ be any other root. Consider the $\kappa$-morphism $\sigma\colon\kappa(a)\to\kappa(b)$ that maps $a$ to $b$. By Corollary~\ref{cor:extension-to-normal}, $\sigma$ can be extended to an automorphism $\omega\colon K\to K$. But $a$ is fixed by $\omega$, so $b=\sigma(a)=\omega(a)=a$. Since $\min_{\kappa,a}$ has no repeated roots, it must be linear, i.e., $a\in\kappa$.

        In the general case, $K$ is the splitting field of a set $S$ of polynomials in $\kappa[x]$. Therefore, there exists a finite subset $T\subseteq S$ such that $a\in L$, where $L$ is the splitting field of $T$. By definition, $L$ is normal. Moreover, if $c\in L$ its minimal $\min_{\kappa,c}$ has no repeated roots because $c\in K$ and $K$ is separable. Thus, to show that $a\in\kappa$ it suffices to show that $a\in\mathcal F(\Gal(L/\kappa))$. Let $\sigma$ be a $\kappa$-automorphism of $L$. By Corollary~\ref{cor:extension-to-normal}, $\sigma$ can be extended to an automorphism $\omega$ of $K$. Hence, $\sigma(a)=\omega(a)=a$, as wanted. \qedhere
    \end{enumerate}
\end{proof}

\begin{cor}\label{cor:separable-iff-in-Galois}
    Let\/ $L$ be a finite extension of\/ $\kappa$. Then
    \begin{enumerate}[a), font=\upshape]
        \item $L$ is separable over\/ $\kappa$ if, and only if, $L$ is contained in a (finite) Galois extension of\/~$\kappa$.
        \item If\/ $L = \kappa(c_1, \dots, c_n)$ with each\/ $c_i$ separable over\/ $\kappa$, then\/ $L$ is separable over\/ $\kappa$.
    \end{enumerate}
\end{cor}

\begin{proof}${}$
    \begin{enumerate}[a), font=\upshape]
        \item If $L$ is contained in a Galois extension $K$ of $\kappa$, then $L$ is separable by definition because $K$ is separable and $L\subseteq K$.

        Conversely, if $L$ is separable over $\kappa$ and $K$ is the splitting field of the set of minimal polynomials of elements in $L$, then $K$ is Galois by the theorem and $L\subseteq K$.

        \item It follows from the theorem that the splitting field $K$ of the set of minimal polynomials $\set{\min_{\kappa,c_i}\mid 1\le i\le n}$ is Galois, hence separable. Since $L\subseteq K$, we see that $L$ is separable too.
    \end{enumerate}
\end{proof}

\begin{defn}
    A field\/ $\kappa$ is \textsl{perfect} if every algebraic extension of\/ $\kappa$ is separable.
\end{defn}

\begin{xmpls}
    Two trivial examples of perfect fields are those with characteristic zero and the algebraically closed ones.
\end{xmpls}

\begin{thm}\label{thm:perfect-field-criterion}
    Let\/ $\kappa$ be a field of characteristic\/ $p$. Then\/ $\kappa$ is perfect if, and only if, $\kappa^p = \kappa$.
\end{thm}

\begin{proof}${}$
    \begin{description}
        \item[\rm\textit{only if\/}:] Take $a\in\kappa$. Pick a root $c$ of the polynomial $x^p-a$ in some algebraic extension $K$ of $\kappa$. Since $\min_{\kappa,c}(x)\mid x^p-a$, we see that $\min_{\kappa,c}(x)\mid(x-c)^p$. By hypothesis $K$ is separable, and so $\min_{\kappa,c}(x)$ is linear, i.e., $c\in\kappa$ and $a=c^p\in\kappa^p$.

        \item[\rm\textit{if\/} part:] Let $K$ be an algebraic extension of $\kappa$. Take $c\in K$ and put $f=\min_{\kappa,c}$. By Proposition~\ref{prop:irreducible-separable-char-p}, there exists $g\in\kappa[x]$ irreducible and separable such that $f(x)=g(x^{p^m})$ for some integer $m\ge0$. Suppose that $m>0$. Put $g(x)=a_0+\cdots+a_nx^n$. By hypothesis, we can write $a_i=b_i^p$ for $0\le i\le n$. Therefore,
        $$
            f(x) = \Big(\sum_{i=0}^nb_ix^{p^{m-1}}\Big)^p,
        $$
        which contradicts the irreducibility of $f$. In consequence, $m=0$ and $f(x)=g(x)$ is separable. \qedhere
    \end{description}
\end{proof}

\begin{cor}\label{cor:finite-implies-perfect}
    Every finite field is perfect.
\end{cor}

\begin{proof}
    If $\kappa$ is finite and $\fchar(\kappa)=p$, the map $\kappa\to\kappa$ defined by $a\mapsto a^p$ is a morphism of fields, hence a monomorphism. In consequence it is an epimorphism, i.e., $\kappa^p=\kappa$. 
\end{proof}

\begin{thm}
    Let\/ $E/F$ be an extension of finite fields of characteristic\/ $p>0$. If\/ $q=|F|$, then\/ $q=p^n$ and\/ $|E|=p^{nm}$ for some\/ $n,m\in\Z$. Moreover, $E/F$ is Galois and\/ $\Gal(E/F)$ is cyclic generated by the Frobenius isomorphism\/ $x\mapsto x^q$.
\end{thm}

\begin{proof}
    Since $F$ is an extension of $\F_p$ we have
    $$
        [E:\F_p] = [E:F][F:\F_p] = [E:F]n.
    $$
    Since $|E|$ and $q$ are powers of $p$, because $E$ and $F$ are $\F_p$-vector spaces, we get
    $$
        |E|=p^{[E:\F_p]} = p^{n[E:F]},
    $$
    which shows that $m=[E:F]$. Consider $f(x)=x^{p^{nm}}-x\in\F_p[x]$. Note that $f(x)$ is separable because $f'(x)=-1$. Moreover, $E$ is the splitting field of $f$ because $E^*$ is cyclic \citep{LC}. Hence, $E/F$ is Galois [cf.~Example~\ref{xmpls:normal}~b) and Corollary~\ref{cor:finite-implies-perfect}].

    By Corollary~\ref{cor:galois-group=dim}, $|\Gal(E/F)|=[E:F]=m$. In addition, the Frobenius morphism $\varphi(x)=x^q$ belongs to $\Gal(E/F)$. Since $\varphi^r(x)=x^{q^r}$, we see that $\ord(\varphi)$ is the minimum $r$ such that $x^{q^r}=x$ for all $x\in E$. But this is precisely $q^m$, because any smaller value would contradict the fact that $[E:\F_p]=nm$.
    
\end{proof}
    

\subsection{Purely inseparable extensions}

\begin{defn}
    Let $K$ be an algebraic field extension of $\kappa$. An element $c \in K$ is \textsl{purely inseparable} over $\kappa$ if $\min_{\kappa, c}$ has only one distinct root. The field $K$ is \textsl{purely inseparable} over $\kappa$ if each of its elements is purely inseparable over~$\kappa$.
\end{defn}

\begin{lem}\label{lem:purely-inseparable-minimal}
    Let\/ $\kappa$ be a field of characteristic\/ $p > 0$. If\/ $a$ is algebraic over\/ $\kappa$, then\/ $a$ is purely inseparable over\/ $\kappa$ if, and only if, $a^{p^m} \in \kappa$ for some\/ $m\ge0$. When this happens, $\min_{\kappa, a}(x) = (x - a)^{p^m}$ for some\/~$m$.
\end{lem}

\begin{proof}
    Let $a$ be purely inseparable. By Proposition~\ref{prop:irreducible-separable-char-p}, $\min_{\kappa,a}(x)=g(x^{p^m})$ for some $g$ irreducible and separable over $\kappa[x]$. Write $g(x)=(x-b_1)\cdots(x-b_r)$. Then
    $$
        \pmin_{\kappa,a}(x)= (x^{p^m}-b_1)\cdots(x^{p^m}-b_r).
    $$
    If $c$ is the only root of $\min_{\kappa,a}(x)$, all the factors must have the same root, namely $c^{p^m}$. Thus, $b_i=c^{p^m}$ for all $1\le i\le r$. Since $g$ is separable this means that $r=1$ and so $\pmin_{\kappa,a}(x)=(x-a)^{p^m}$.

    Conversely, if $a^{p^m}\in\kappa$ for some $m\ge0$, then $\pmin_{\kappa,a}(x)\mid(x^{p^m}-a^{p^m})=(x-a)^{p^m}$. In particular, $a$ is the only root of $\pmin_{\kappa,a}(x)$, i.e., $a$ is purely inseparable.
    
\end{proof}

\begin{prop}\label{prop:purely-inseparable-properties}
    Let\/ $K$ be an algebraic extension of\/ $\kappa$. Then the following properties hold true:
    \begin{enumerate}[a), font=\upshape]
        \item If\/ $a \in K$ is separable and purely inseparable over\/ $\kappa$, then\/ $a \in \kappa$.

        \item If\/ $\kappa \subseteq L \subseteq K$ are fields, then\/ $K / \kappa$ is purely inseparable if, and only if, $K / L$ and\/ $L / \kappa$ are purely inseparable.
        
        \item If\/ $K / \kappa$ is purely inseparable, then\/ $K / \kappa$ is normal and\/ $\Gal(K / \kappa) = \set{\id}$. Moreover, if\/ $[K : \kappa] < \infty$, and if\/ $p = \fchar(\kappa)$, then\/ $[K : \kappa] = p^n$ for some\/~$n$.
        
        \item If\/ $K = \kappa(S)$ with each\/ $a \in S$ purely inseparable over\/ $\kappa$, then\/ $K$ is purely inseparable over\/ $\kappa$.
    \end{enumerate}
\end{prop}

\begin{proof}${}$
    \begin{enumerate}[a), font=\upshape]
        \item If $a$ is purely inseparable and separable over $\kappa$, its minimal polynomial has only one root and this root is not repeated. Therefore, it must be linear, i.e., $a\in\kappa$.

        \item If $K/L$ and $L\kappa$ are purely inseparable, given $a\in K$, by Lemma~\ref{lem:purely-inseparable-minimal}, there exists $m\ge0$ such that $b=a^{p^m}\in L$. By the same proposition, there exists $r\ge0$ such that $b^{p^r}\in\kappa$. In consequence, $a^{p^{m+r}}\in\kappa$. The converse is a direct consequence of the same proposition.

        \item If $K$ is purely inseparable, it is normal because every minimal polynomial $\pmin_{\kappa,c}$ for $c\in K$ has all its roots, namely $c$, in $K$. Take $\sigma\in \Gal(K/\kappa)$. Given $c\in K$, $\sigma(c)$ is a root of $\pmin_{\kappa,c}(x)$, i.e., $\sigma(c)=c$.

        If $[K:\kappa]<\infty$ there exist $c_1,\dots,c_n$ such that $K$ is the splitting field of $\set{c_1,\dots,c_n}$. Consider the tower of extensions
        $$
            \kappa\subseteq\kappa(c_1)\subseteq\cdots\subseteq
                \kappa(c_1,\dots,c_n)=K.
        $$
        By part~b) $\kappa(c_1,\dots,c_i)$ is purely inseparable over $\kappa(c_1,\dots,c_{i-1})$. Therefore, to prove that $[K:\kappa]=p^n$ for some $n$, it suffices to consider the case where $K=\kappa(c)$. But in this case, the result is a direct consequence of Lemma~\ref{lem:purely-inseparable-minimal}.


        \item Take $c\in K$. There exists $\set{a_1,\dots,a_n}\subseteq S$ such that $c\in\kappa(a_1,\dots,a_n)$, say
        $$
            c = \frac{f(a_1,\dots,a_n)}{g(a_1,\dots,a_n)},
        $$
        for some $f,g\in\kappa[x_1,\dots,x_n]$. By Lemma~\ref{lem:purely-inseparable-minimal}, there is $m\ge0$ such that $a_i^{p^m}\in \kappa$ for $1\le i\le n$. Therefore, if $\bar{f}$ and $\bar{g}$ denote the polynomials obtained from $f$ and $g$ by replacing each coefficient with its $p^m$-th power, we have
        $$
            c^{p^m}=\frac{\bar f(a_1^{p^m},\dots,a_n^{p^m})}
                {\bar g(a_1^{p^m},\dots,a_n^{p^m})}\in\kappa.
        $$
    \end{enumerate}
\end{proof}

\begin{xmpl}\label{xmpl:counterexample-sep-p.insep}
    Let $\kappa=\F_2(x)$ and $K=\kappa(\sqrt[6]x)$. Then, $K=\kappa(\sqrt x,\sqrt[3]x)$. Indeed. Put $L=\kappa(\sqrt x,\sqrt[3]x)$. Clearly, $L\subseteq K$. Since $[L:\kappa(\sqrt[3]x)]=2$, we see that $2\mid[L:\kappa]$. Similarly, $3\mid[L:\kappa]$ because $[L:\kappa(\sqrt x)]=3$ as shown in Example~\ref{xmpl:sqrt3 x-has-degree-3}. It follows that $[K:\kappa]=6\mid[L:\kappa]$.
    $$
        \begin{tikzcd}[column sep=-1cm]
                &{K=\kappa(\sqrt x,\sqrt[3]x)}\\
            \kappa(\sqrt x)
                    \arrow[ru,no head]
                &&{\kappa(\sqrt[3]x)}
                    \arrow[lu,no head]\\
                &\kappa=\F_2(x)
                    \arrow[lu,"\text{p.~insep.}",no head]
                    \arrow[ru,"\text{separable}"',no head]
        \end{tikzcd}
    $$
    Since $\fchar(\kappa)=2$ and $(\sqrt x)^2\in\kappa$, we see that $\kappa(\sqrt x)$ is purely inseparable over $\kappa$. On the other hand, $\kappa(\sqrt[3]x)$ is separable over $\kappa$ because $\pmin_{\kappa,\sqrt[3]x}(t)=t^3-x$, whose derivative is $t^2$, which is not zero [cf.~Proposition~\ref{prop:irreducible-separable-char-p}].
\end{xmpl}

\begin{defn}
    Let $K$ be a field extension of $\kappa$. Then the \textsl{separable closure} of $\kappa$ in $K$ is the set
    $$
        \kappa_s = \set{a\in K\mid a\text{\rm\ is separable over }\kappa}.
    $$
    The \textsl{purely inseparable closure} of $\kappa$ in $K$ is the set
    $$
        \kappa_i = \set{a\in K\mid a\text{\rm\ is purely inseparable over }\kappa}.
    $$
\end{defn}

\begin{thm}\label{thm:sep-and-p.insep-closures}
    Let\/ $K$ be a field extension of\/ $\kappa$. If\/ $\kappa_s$ and\/ $\kappa_i$ denote the separable and purely inseparable closures of\/ $\kappa$ in\/ $K$, respectively, then\/ $\kappa_s$ and\/ $\kappa_i$ are field extensions of\/ $\kappa$ with\/ $\kappa_s/\kappa$ separable, $\kappa_i / \kappa$ purely inseparable, and\/ $\kappa_s \cap \kappa_i = \kappa$. If\/ $K / \kappa$ is algebraic, then\/ $K / \kappa_s$ is purely inseparable.
\end{thm}

\begin{proof}
    Take $a,b\in\kappa_s$. By Corollary~\ref{cor:separable-iff-in-Galois}, $\kappa(a,b)$ is separable over $\kappa$. In particular, $ab$ and $a+b$ are separable over $\kappa$.

    If $a,b\in\kappa_i$, there exists $m\ge0$ such that $a^{p^m},b^{p^m}\in\kappa$. Thus, $ab$ and $a+b$ are purely inseparable over $\kappa$ because $(ab)^{p^m}$ and $(a+b)^{p^m}$ are elements of $\kappa$.

    Proposition~\ref{prop:purely-inseparable-properties}~a) implies that $\kappa_s\cap \kappa_i=\kappa$.

    Now suppose that $K/\kappa$ is algebraic and take $c\in K$. By Proposition~\ref{prop:irreducible-separable-char-p}~b), $\pmin_{\kappa,c}(x)=g(x^{p^m})$, where $g\in\kappa[x]$ is irreducible and separable and $m\ge0$. Therefore, $b = c^{p^m}$ satisfies $g(b)=0$, i.e, $g=\min_{\kappa,b}$. By definition $b\in\kappa_s$. Then $c$ is purely inseparable over $\kappa_s$ because $c^{p^m}=b\in\kappa_s$.
\end{proof}

\begin{cor}\label{cor:separable-transitivity}
    If\/ $\kappa \subseteq L \subseteq K$ are fields such that\/ $L / \kappa$ and\/ $K / L$ are separable, then\/ $K / \kappa$ is separable.
\end{cor}

\begin{proof}
    First note that $K/\kappa$ is algebraic because every separable extension is algebraic, and this is a transitive property [cf.~Theorem~\ref{thm:algebraic-transitivity}]. Let $\kappa_s$ be the separable closure of $\kappa$ in $K$. Then $L\subseteq\kappa_s$ and so $K$ is separable over $\kappa_s$. But $K$ is purely inseparable over $\kappa_s$ by Theorem~\ref{thm:sep-and-p.insep-closures}, which implies that $K=\kappa_s$ by Proposition~\ref{prop:purely-inseparable-properties}~a).
    
\end{proof}

\begin{cor}
    Let\/ $K$ be a finite extension of\/ $\kappa$. If\/ $\fchar(\kappa)\perp[K:\kappa]$, then\/ $K / \kappa$ is separable.
\end{cor}

\begin{proof}
    Let $p=\fchar(\kappa)$. If $\kappa_s$ is the separable closure of $\kappa$ in $K$, then $K/\kappa_s$ is purely separable. According to Proposition~\ref{prop:purely-inseparable-properties}, $[K:\kappa_s]=p^m$ for some $m\ge0$. Therefore, $p^m\mid[K:\kappa]$, which implies that $m=0$. Thus, $K=\kappa_s$ is separable over $\kappa$.
\end{proof}

\begin{cor}\label{cor:separable-closure-restriction}
    Let\/ $K/\kappa$ be a field extension and\/ $\kappa_s$ the separable closure of\/ $\kappa$ in\/ $K$. If\/ $\sigma\in\Gal(K/\kappa)$, then\/ $\sigma(\kappa_s)=\kappa_s$. In particular, there is a well-defined morphism\/ $\sigma_s\in\Gal(\kappa_s/\kappa)$ such that\/ $\sigma_s\subseteq\sigma$. Moreover the restriction map
    \begin{align*}
        \rho\colon\Gal(K/\kappa)&\to\Gal(\kappa_s/\kappa)\\
        \sigma&\mapsto\sigma_s
    \end{align*}
    is a morphism of groups. If $K/\kappa$ is algebraic, then $\rho$ is a monomorphism.
\end{cor}

\begin{proof}
    Take $\sigma\in\Gal(K/\kappa)$ and $c\in\kappa_s$. By definition $\pmin_{\kappa,c}$ is separable. But $\pmin_{\kappa,\sigma(c)}=\pmin_{\kappa,c}$ and $\sigma(c)\in K$. It follows that $\sigma(c)\in\kappa_s$. Then $\sigma_s$ is well-defined (by restriction and coastriction) and $\rho$ is a morphism of groups. Moreover, $\ker\rho=\Gal(K/\kappa_s)$. If $K/\kappa$ is algebraic then $K/\kappa_s$ is purely inseparable, $\ker\rho$ is trivial by Proposition~\ref{prop:purely-inseparable-properties}~c).
\end{proof}

\begin{thm}\label{thm:sep-and-p.insep-galois}
    Let\/ $K$ be a normal extension of\/ $\kappa$, and\/ $\kappa_s$ and\/ $\kappa_i$ the separable and purely inseparable closures. Then\/ $\kappa_s / \kappa$ is Galois, $\kappa_i = \mathcal F(\Gal(K / \kappa))$, and\/ $\Gal(\kappa_s / \kappa) \cong \Gal(K / \kappa_i)$. Thus, $K / \kappa_i$ is Galois. Moreover, $K = \kappa_s \kappa_i$.
\end{thm}

\begin{proof}
    To verify that $\kappa_s/\kappa$ is Galois, take $a\in\kappa_s$ and put $f=\min_{\kappa,a}$. Since $a\in K$, which is normal, $f$ splits over $K$. In addition, $f$ has no repeated roots. Therefore, if $c$ is a root of $f$, since $c\in K$ and $K$ is purely inseparable over $\kappa_s$ by Theorem~\ref{thm:sep-and-p.insep-closures}, $\min_{\kappa_s,c}\mid f$, and Lemma~\ref{lem:purely-inseparable-minimal} implies that $c\in\kappa_s$. Thus, $f$ splits over~$\kappa_s$. In consequence $\kappa_s$ is normal over $\kappa$ and hence Galois by Theorem~\ref{thm:galois-equivalences}.

    According to Corollary~\ref{cor:separable-closure-restriction}, the restriction map
    \begin{align*}
        \rho\colon\Gal(K/\kappa)&\to\Gal(\kappa_s/\kappa)\\
        \sigma&\mapsto\sigma_s,
    \end{align*}
    where $\sigma_s\subseteq\sigma$, is a monomorphism of groups. In this case, it is also an epimorphism by Corollary~\ref{cor:extension-to-normal}.

    To verify that $\kappa_i\subseteq\mathcal F(\Gal(K/\kappa))$, take $a\in\kappa_i$ and $\sigma\in\Gal(K/\kappa)$. By Lemma~\ref{lem:purely-inseparable-minimal}, $a^{p^m}\in\kappa$ for some $m\ge0$. Hence, $\sigma(a)^{p^m}=a^{p^m}$, i.e., $\sigma(a)=a$.

    For the other inclusion, take $b\in\mathcal F(\Gal(K/\kappa))$. Since $K/\kappa_s$ is purely inseparable, there exists $m\ge0$ such that $b^{p^m}\in\kappa_s$. Take $\nu\in\Gal(\kappa_s/\kappa)$. Let $\omega=\rho^{-1}(\nu)\in\Gal(K/\kappa)$. Then, $\nu\subseteq\omega$ and $\nu(b^{p^m})=\omega(b^{p^m})=b^{p^m}$. It follows that $b^{p^m}\in\mathcal F(\Gal(\kappa_s/\kappa))=\kappa$ because, as we have seen above, $\kappa_s/\kappa$ is Galois. In consequence, $b\in\kappa_i$ by Lemma~\ref{lem:purely-inseparable-minimal}.

    Moreover, the equality $\kappa_i=\mathcal F(\Gal(K/\kappa))$ implies that $\Gal(K/\kappa)\cong\Gal(K/\kappa_i)$ [cf.~Lemma~\ref{lem:fixed-field-properties}~f)].

    It remains to be seen that $K=\kappa_s\kappa_i$. But $K$ is separable over $\kappa_s\kappa_i$ because it is separable over $\kappa_i$ (it is Galois) and is purely inseparable over $\kappa_s\kappa_i$ because it is purely inseparable over $\kappa_s$ (Theorem~\ref{thm:sep-and-p.insep-closures}). As a result, $K=\kappa_s\kappa_i$ by Proposition~\ref{prop:purely-inseparable-properties}. 
\end{proof}

\begin{defn}
    Let $K$ be a finite field extension of $\kappa$ with separable and purely inseparable closures $\kappa_s$ and $\kappa_i$. Then $[K:\kappa]_s=[\kappa_s:\kappa]$ is the \textsl{separable degree} of $K/\kappa$, and $[K:\kappa]_i=[K:\kappa_s]$ its \textsl{inseparable degree}.
\end{defn}

\begin{rem}\label{rem:sep-and-p.insep-galois}
    If\/ $K$ is a finite extension of\/ $\kappa$, then
    $$
        [K:\kappa]=[K:\kappa]_s[K:\kappa]_i.
    $$
    Note that the inseparable degree is defined as $[K:\kappa_s]$ and not as $[\kappa_i:\kappa]$ because $[K:\kappa_s]$ is a better measure of how far is $K/\kappa$ from being separable. In the normal case, however, $[\kappa_i:\kappa]=[K:\kappa_s]$, as shown by the following diagram,
    $$
        \begin{tikzcd}[column sep=-0.2cm,row sep=huge]
            {}
                &&K=\kappa_s\kappa_i\\
                &\kappa_s
                    \arrow[ru,"{[K:\kappa]_i}",no head]
                &&\kappa_i
                    \arrow[lu,"\text{Galois}"',no head,pos=0.75]
                    \arrow[lu,"{[K:\kappa_s]}"',no head,pos=0.4]\\
            {}
                    \arrow[uu,"\text{normal}",no head,bend left]
                &&\kappa
                    \arrow[lu,"\text{Galois}",no head,pos=0.75]
                    \arrow[ru,no head]
                    \arrow[lu,"{[K:\kappa]_s}",no head,pos=0.4]
        \end{tikzcd}
    $$
\end{rem}

\begin{xmpl}\label{xmpl:sep-p.insep-counterexample}
    What follows is an an example of a field extension $K/\kappa$ in which $K$ is not separable over the purely inseparable closure $\kappa_i$. It is also an example of a nonseparable field extension $K/\kappa$ in which $\kappa_i=\kappa$.

    Let $k$ be a field of characteristic $2$, let $\kappa$ be the rational function field $\kappa = k(x, y)$, let $F = \kappa(u)$, where $u$ is a root of $t^2 + t + x$, and let $K = F(\sqrt{uy})$. In a diagram,
    %\small
    $$
        \begin{tikzcd}%[row sep=0.7cm]
            K=F(\sqrt{uy})\\
            F=\kappa(u)
                \arrow[u,"t^2-uy=0",no head]
                \arrow[u,no head, bend right=40,"\text{p.~insep.}"']\\
            {\kappa=k(x,y)}.
                \arrow[u,"t^2+t+x=0",no head]
                \arrow[u,no head, bend right=40,"\text{sep.~closure}"']
        \end{tikzcd}
    $$
    %\normalsize
    Note that $K/F$ is purely inseparable (Lemma~\ref{lem:purely-inseparable-minimal}) and $F/\kappa$ is separable (Proposition~\ref{prop:f-perp-f'}), so $F=\kappa_s$.

    A basis for $K/\kappa$ is $\set{1, u, \sqrt{uy}, u\sqrt{uy}}$. Take $\zeta\in K$ such that $\zeta^2\in\kappa$ and write
    $$
        \zeta = a+bu+c\sqrt{uy}+du\sqrt{uy}
    $$
    with $a,b,c,d\in\kappa$. Then
   \begin{align*}
        \zeta^2 &= a^2 + b^2(u+x) + c^2uy + d^2(u+x)uy\\
            &= a^2 + b^2(u+x) + c^2uy + d^2(u+x)y + d^2xuy.
   \end{align*}
    The coefficient of $u$ is zero because $\zeta^2\in\kappa$, i.e.,
    $$
        (b+c)^2+(c+d)^2y+d^2xy=0.
    $$
    Suppose, for a contradiction, that $d\ne0$. Then, solving for $x$ yields
    $$
        x = \Big(\frac{c+d}{d}\Big)^2 + \frac1y\Big(\frac{b+c}{d}\Big)^2 \in\kappa^2(y),
    $$
    which is impossible because the LHS has an odd degree in $x$ and the RHS an even one.

    It follows that $d=0$. Then $c=0$ because $y\notin\kappa^2$. But $c=d=0$ implies $b=0$, which means that $\zeta=a\in\kappa$. Since $\zeta$ was arbitrarily chosen, by Lemma~\ref{lem:purely-inseparable-minimal} we may conclude that $\kappa_i=\kappa$.

    In particular, $K/\kappa_i$ is not separable because $K/F$ is purely inseparable.
\end{xmpl}

\section{Problems}

\begin{probl}
    If\/ $\kappa \subseteq L \subseteq K$ are fields such that\/ $K / \kappa$ is separable, show that\/ $L / \kappa$ and\/ $K / L$ are separable.
\end{probl}

\begin{solution}
    Suppose that $K/\kappa$ is separable. Then every element of $c\in K$ satisfies that $\pmin_{\kappa,c}$ is separable over $\kappa$. By part~d) of Lemma~\ref{lem:separable-polynomials} $\pmin_{L,c}$ is separable, and since $\pmin_{L,c}\mid\pmin_{\kappa,c}$, $c$ is separable over $L$ by part~b) of the same lemma. In addition, $L/\kappa$ is separable simply because $L\subseteq K$.

    The converse is Corollary~\ref{cor:separable-transitivity}.
\end{solution}

\begin{probl}
    If\/ $K$ is a field extension of\/ $\kappa$ and if\/ $c \in K$ is not separable over\/ $\kappa$, show that\/ $c^{p^m}$ is separable over\/ $\kappa$ for some\/ $m \ge 0$, where\/ $p = \fchar(\kappa)$.
\end{probl}

\begin{solution}
    Even though this not stated in the problem we will assume that $c$ is algebraic over $\kappa$.
    
    Let $\kappa_s$ be the separable closure of $\kappa$ in $\kappa(c)$. By Theorem~\ref{thm:sep-and-p.insep-closures} $c$ is purely inseparable over $\kappa_s$ and the result follows from Lemma~\ref{lem:purely-inseparable-minimal}.
\end{solution}

\begin{probl}\label{probl:normal+p.insep-is-normal}
    Let\/ $\kappa \subseteq L \subseteq K$ be fields such that\/ $K / L$ is normal and\/ $L / \kappa$ is purely inseparable. Show that\/ $K / \kappa$ is normal.
\end{probl}

\begin{solution}
    Take $f\in\kappa[x]$ irreducible. Suppose that $c\in K$ is a root of $f$. Pick $g\in L[x]$ irreducible, such that $g(c)=0$. Then $g\mid f$. By Lemma~\ref{lem:purely-inseparable-minimal}, there exists $m\ge0$ such that $g^{p^m}\in\kappa[x]$. It follows that $f\mid g^{p^m}$. In consequence $f=g^r$ for some $r\ge0$. Thus, $f$ and $g$ have the very same roots and since all the roots of $g$ are in $K$, all the roots of $f$ are in~$K$.
\end{solution}

\begin{probl}
    Let\/ $\kappa$ be a field of characteristic\/ $p > 0$, and let\/ $a \in \kappa - \kappa^p$. Show that\/ $x^p - a$ is irreducible over\/~$\kappa$.
\end{probl}

\begin{solution}
    Take an irreducible factor $g(x)\in\kappa[x]$ of $x^p-a$. Pick a root $c$ of $g(x)$. Since $c^p-a=0$, it follows from  Proposition~\ref{prop:purely-inseparable-properties} that $\kappa(c)/\kappa$ is purely inseparable over $\kappa$. Then, $g(x)=\pmin_{\kappa,c}(x)=(x-c)^{p^m}$ by Lemma~\ref{lem:purely-inseparable-minimal}. But $g(x)\mid x^p-a$, and so both polynomials are the same.
\end{solution}

\begin{probl}
    Let\/ $\kappa$ be a field of characteristic\/ $p > 0$, and let\/ $K$ be a purely inseparable extension of\/ $\kappa$ with\/ $[K : \kappa] = p^n$. Prove that\/ $a^{p^n} \in \kappa$ for all\/ $a \in K$.
\end{probl}

\begin{solution}
    If $a\in K$, by Lemma~\ref{lem:purely-inseparable-minimal}, $\pmin_{\kappa,a}=(x-a)^{p^m}$ for some $m\ge0$. In particular, $a^{p^m}\in\kappa$. Therefore,
    $$
        p^m=[\kappa(a):\kappa]\mid[K:\kappa]=p^n,
    $$
    and so $m\ge n$. Thus, $a^{p^n}=(a^{p^m})^{p^{n-m}}\in\kappa$ since $a^{p^m}\in\kappa$.
\end{solution}

\begin{probl}\label{probl:separable.separable=separable}
    Let\/ $K$ and\/ $L$ be extensions of\/ $\kappa$. Show that\/ $KL$ is separable over\/ $\kappa$ if both\/ $K$ and\/ $L$ are separable over\/ $\kappa$. Is the converse true?
\end{probl}

\begin{solution}
    Since every element of $K$ is separable over $K$ and the same holds for the elements of $L$, we deduce that $KL=\kappa(K\cup L)$ is separable. To see this take $c\in KL$. Pick $S\subseteq K$ and $T\subseteq L$ finite such that $c\in\kappa(S\cup T)$. Since $\kappa(S\cup T)$ is separable by Corollary~\ref{cor:separable-iff-in-Galois}, $c$ is separable by definition.

    The converse is clearly true by definition because $K\cup L\subseteq KL$.
\end{solution}

\begin{probl}\label{probl:galois.galois=galois}
    Let\/ $K$ and\/ $L$ be extensions of\/ $\kappa$. Show that\/ $KL$ is Galois over\/~$\kappa$ if both\/ $K$ and\/ $L$ are Galois over\/ $\kappa$. Is the converse true?
\end{probl}

\begin{solution}
    By Theorem~\ref{thm:galois-equivalences} both $K$ and $L$ are normal an separable over $\kappa$. Then, according to the preceding problem, $KL$ is also separable over $\kappa$. To see that $KL$ is normal, take a $\kappa$-morphism $\sigma\colon KL\to F$, where $F$ is an algebraic closure of $KL$. By Theorem~\ref{thm:normal-equivalences}~b), $\sigma(K)\subseteq K$ and $\sigma(L)\subseteq L$. It follows that $\sigma(KL)\subseteq KL$, which shows that $KL$ is normal for the very same reasons.

    The converse is not true because not every subfield of a normal field is normal [cf.~Example~\ref{xmpls:normal}~c].
\end{solution}

\begin{probl}\label{probl:KL}
    Let\/ $K$ and\/ $L$ be subfields of a common field $F$, both of which contain a field\/ $\kappa$. Prove the following statements.
    \begin{enumerate}[a), font=\upshape]
        \item If\/ $K = \kappa(S)$ for some set\/ $S \subseteq K$, then\/ $KL = L(S)$.
        \item $[KL: \kappa] \leq [K: \kappa][L : \kappa]$.
        \item If\/ $K$ and\/ $L$ are algebraic over\/ $\kappa$, then\/ $KL$ is algebraic over\/ $\kappa$.
        \item Prove that the previous statement remains true when ``algebraic'' is replaced by ``normal,'' ``separable,'' ``purely inseparable,'' or ``Galois''.
    \end{enumerate}
\end{probl}

\begin{solution}${}$
    \begin{enumerate}[a), font=\upshape]
        \item Let's define $KL$ as the smallest subfield of $F$ that contains $K\cup L$ and $\kappa(S)$ as the smallest subfield that contains $\kappa\cup S$. In the case where $K=\kappa(S)$, since $K$ is the smallest subfield of $F$ that contains $\kappa$ and $S$, we have $K=\kappa(S)\subseteq L(S)$ and so
        $$
            KL \subseteq L(S)\subseteq L(K)\subseteq KL.
        $$

        \item Let $B$ and $C$ bases of the $\kappa$-vector spaces $K$ and $L$. Then 
        $$
            KL=L(B)=\kappa(C)(B)=\kappa(BC),
        $$
        where $BC=\set{bc\mid b\in B,\,c\in C}$. Given that $BC$ spans the vector space $\kappa(BC)$, we have
        $$
            [KL:\kappa]\le |BC|\le |B||C|=[K:\kappa][L:\kappa].
        $$

        \item If $c\in KL$, there are finite subsets $S\subseteq K$ and $T\subseteq L$ such that $c\in\kappa(S)\kappa(T)$. By part~b) this product is finite over $\kappa$ and so $c$ is algebraic over $\kappa$.

        \item The solution of Problem~\ref{probl:galois.galois=galois} shows that the product of two normal (resp.\ Galois) fields is normal (resp.\ Galois). For the separable case see Problem~\ref{probl:separable.separable=separable}. Finally, if $K$ and $L$ are purely inseparable over $\kappa$, by Lemma~\ref{lem:purely-inseparable-minimal}f, or every product $ab$ with $a\in K$ and $b\in L$ there exist $n$ and $m$ such that $a^{p^n}\in\kappa$ and $b^{p^m}\in L$. Therefore, $(ab)^{p^{\max(n,m)}}\in\kappa$, which shows that $ab$ is purely separable over $\kappa$. Then, $KL$ is purely inseparable by Proposition~\ref{prop:purely-inseparable-properties}~d).
    \end{enumerate}
\end{solution}

\begin{probl}
    Let\/ $K$ be the rational function field\/ $\kappa(x)$ over a perfect field\/~$\kappa$ of characteristic\/ $p > 0$. Let\/ $F = \kappa(u)$ for some\/ $u \in K$, and write\/ $u = f(x)/g(x)$ with\/ $f$ and\/ $g$ relatively prime. Show that\/ $K / F$ is a separable extension if, and only if, $u \notin K^p$.
\end{probl}

\begin{solution}${}$ First recall from Example~\ref{xmpl:k(t)/k(u)} that $\pmin_{F,x}(t)=f(t)-ug(t)$.
    
    Let's consider first the case where $K/F$ is separable. Then $\pmin_{F,x}(t)$ must be separable over $F$. Suppose for a contradiction that $u\in K^p$. This means that $f(x)=a(x)^p$ and $g(x)=b(x)^p$. Therefore, $\pmin_{F,x}'(t)=0$, in contradiction with the hypothesis of separability.

    Now assume that $u\notin K^p$. To see that $K/F$ is separable it suffices to check that $\min_{F,x}$ is separable. Suppose it is not. Then $f'(t)-ug'(t)=0$, which can only happen if $f'=g'=0$ (because $u$ is transcendental over $\kappa(t)$), i.e., $f,g\in\kappa[x]^p$, which is impossible because that would imply $u\in K^p$.
\end{solution}

\begin{probl}
    Let\/ $K$ be a finite extension of\/ $\kappa$ with\/ $\fchar(\kappa) = p > 0$ and\/ $K^p \subseteq \kappa$. Thus, $K / \kappa$ is purely inseparable. A set\/ $\set{a_1, \dots, a_n} \subseteq K$ is said to be a \textsl{$p$-basis} for\/ $K / \kappa$ provided that there is a chain of proper extensions 
    $$
        \kappa \subset \kappa(a_1) \subset \dots \subset \kappa(a_1,\dots,a_n) = K.
    $$
    Show that if\/ $\set{a_1, \dots, a_n}$ is a\/ $p$-basis for\/ $K / \kappa$, then 
    $$
        [K : \kappa] = p^n,
    $$
    and conclude that the number of elements in a\/ $p$-basis is uniquely determined by\/ $K / \kappa$. The number\/ $n$ is called the \textsl{$p$-dimension} of\/ $K / \kappa$. Also, show that any finite purely inseparable extension has a\/ $p$-basis.
\end{probl}

\begin{solution}
    Let $\set{a_1,\dots,a_n}$ be a $p$-basis of $K/\kappa$. Given that $a_1^p\in K^p\setminus\kappa$, the minimal of $a_1$ over $\kappa$ is $x^p-a_1^p$ [cf.~Lemma~\ref{lem:purely-inseparable-minimal}]. Thus, $[\kappa(a_1):\kappa]=p$.
    
    Put $a_0=1$ and assume inductively that $[\kappa(a_{i-1}):\kappa(a_i)]=p$ for $1\le i\le m$<n. If $F=\kappa(a_1,\dots,a_m)$, we have 
    $$
        a_{m+1}^p\in\kappa\subseteq F\quad\text{and}\quad
            a_{m+1}\notin F,
    $$
    which shows that $\pmin_{F,a_{m+1}}(x)=x^p-a_{m+1}^p$. Thus, $[F(a_{m+1}):F]=p$ and $[K:\kappa]=p^n$. In particular, $n$ doesn't depend on the $p$-basis chosen to compute it.

    Let's now assume that $K/\kappa$ is finite and purely inseparable. If $K=\kappa$, then the empty set is a $p$-basis. For the other case, where $K\ne\kappa$, pick $c\in K\setminus\kappa$ and let $m$ be the smallest integer for which $c^{p^m}\in\kappa$ [cf.~Lemma~\ref{lem:purely-inseparable-minimal}]. Put $a=c^{p^{m-1}}$. Then $a\notin\kappa$ and $a_1^p\in\kappa$. Thus $[\kappa(a):\kappa]=p$ and since $[K:\kappa(a)]<[K:\kappa]$, we may assume by induction in $[K:\kappa]$, that there is a $p$-basis $\set{a_1,\dots,a_n}$ of $K$ over $\kappa(a)$. Then, $\set{a,a_1,\dots,a_n}$ is a $p$-basis of $K$ over $\kappa$.
\end{solution}

\begin{probl}
    Give three examples of a field extension\/ $K/\kappa$ which is neither normal nor separable. 
\end{probl}

\begin{solution}
    In Example~\ref{xmpl:sep-p.insep-counterexample}, the extension $K/\kappa$ is not normal. Otherwise, according to Theorem~\ref{thm:sep-and-p.insep-galois}, $K/\kappa_i$ would have been separable, which is not, and it is also the case that $\kappa_i=\kappa$ [cf.~Remark~\ref{rem:sep-and-p.insep-galois}].

    In Example~\ref{xmpl:counterexample-sep-p.insep}, $\kappa(\sqrt[3]x)$ is the separable closure of $\kappa$ in $K$ and $\kappa(\sqrt x)$ its purely inseparable closure. In particular, $K/\kappa$ is not separable. We claim that it is not normal either. By Theorem~\ref{thm:sep-and-p.insep-galois}, to see this it is enough to verify that $\kappa(\sqrt[3]x)$ is not normal over $\kappa$. Suppose otherwise. Put $\alpha=\sqrt[3]x$. As shown in Example~\ref{xmpl:sqrt3 x-has-degree-3},
    $$
        \pmin_{\kappa,\alpha}(t)= t^3-x = (t-\alpha)(t^2+\alpha t+\alpha^2)
    $$
    would split over $\kappa(\alpha)$. Thus, we are reduced to verify that $t^2+\alpha t+\alpha^2$ has no root in $\kappa(\alpha)$. Suppose for a contradiction that $\zeta$ is such a root. Then $\zeta\alpha\in\kappa(\alpha)$ would be a root of $t^2+t+1$. But this  is impossible because $\zeta\alpha$ would generate an intermediate field of degree~$2$ in an extension of degree~$3$.

    Third example, ???
\end{solution}

\begin{probl}
    Let\/ $\kappa$ be a field of characteristic\/ $p > 0$, let\/ $K = \kappa(x, y)$ be the rational function field over\/ $\kappa$ in two variables, and let\/ $L = \kappa(x^p, y^p)$. Show that\/ $K / L$ is a purely inseparable extension of degree\/ $p^2$. Show that\/ $K \ne L(c)$ for any\/ $c \in K$.
\end{probl}

\begin{solution}
    From Proposition~\ref{prop:purely-inseparable-properties}, we know that $K/L$ is purely inseparable. By considering the intermediate field $F=\kappa(x,y^p)$, the question reduces to verify that $[K:F]=[F:L]=p$, both of which are clear by Lemma~\ref{lem:purely-inseparable-minimal} because $[K:F]=[\kappa(x)(y):\kappa(x)(y^p)]$ and $[F:L]=[\kappa(y^p)(x):\kappa(x^p)(y^p)]$.

    The reason why $K\ne L(c)$ is that given $c\in K$, $c^p\in\kappa$.
    
\end{solution}

\begin{probl}
    Prove the following product formulas for separability and inseparability degree: If\/ $\kappa\subseteq L\subseteq K$ are fields, then show that
    \begin{align*}
        [K : \kappa]_s &= [K : L]_s[L : \kappa]_s\\
        [K : \kappa]_i &= [K : L]_i[L : \kappa]_i.
    \end{align*}
\end{probl}

\begin{solution} The solution requires the following

    \textbf{Lemma.}
        \textit{Let\/ $K/\kappa$ be a finite extension with separable and purely inseparable closures\/ $\kappa_s$ and\/ $\kappa_i$. If\/ $K/\kappa_i$ is separable, then\/ $[K:\kappa]_i=[\kappa_i:\kappa]$, i.e.,}
        \begin{align*}
            [\kappa_s:k] = [K:\kappa_i]\quad\text{and}\quad
            [K:\kappa_s] = [\kappa_i:\kappa].
        \end{align*}
        In a diagram,
        \small
        $$
            \begin{tikzcd}[column sep=0cm]
                    &K\\
                \kappa_s
                        \arrow[ru,no head]
                    &&\kappa_i
                        \arrow[lu,"{\text{sep.}\implies[\kappa_s:\kappa]}"',no head]\\
                    &\kappa
                        \arrow[lu,no head]\arrow[ru,no head]
            \end{tikzcd}
        $$
        \normalsize
    \begin{proof} Consider the diagram
    \small
    $$
        \begin{tikzcd}[row sep=0.8cm, column sep=0.5cm]
                &F\\
            {\kappa_{s,F}}
                    %\arrow[ru,"\text{p.~insep.}",no head,dashed,pos=0.8cm]
                    \arrow[ru,"{[\kappa_i:\kappa]}",no head,dashed]
                &K
                    \arrow[u,no head,dashed]\\
            \kappa_s
                    \arrow[ru,no head]
                    \arrow[u,no head,dashed]
                &&\kappa_i
                    \arrow[lu,"\text{sep.}",no head]
                    \arrow[luu,"\text{\textcolor{black}{Galois}}"',no head,pos=0.8,color=white]
                    \arrow[luu,"{[\kappa_{s,F}:\kappa]}"',no head,dashed]\\
                &\kappa
                    \arrow[lu,no head]
                    \arrow[ru,no head]
                    %\arrow[uuu,"\text{normal}",no head,bend right=74]
        \end{tikzcd}
    $$
    \normalsize
    where $F/\kappa_i$ is Galois [cf.~Corollary~\ref{cor:separable-iff-in-Galois}]. By Problem~\ref{probl:normal+p.insep-is-normal}, $F/\kappa$ is normal. If $\kappa_{s,F}$ is the separable closure of $\kappa$ in $F$, then Remark~\ref{rem:sep-and-p.insep-galois} lets us conclude that $[F:\kappa_{s,F}]=[\kappa_i:\kappa]$. Similarly, $F/\kappa_s$ is normal, and $\kappa_{s,F}$ is the separable closure of $F/\kappa_s$ because $F/\kappa_{s,F}$ is purely inseparable. Therefore, we can invoke Remark~\ref{rem:sep-and-p.insep-galois} to conclude that $[\kappa_i:\kappa]=[K:\kappa_s]$, as wanted.
    \end{proof}

    Back to the problem, let $\kappa_{s,L}$ be the separable closure of $\kappa$ in $L$. If $\kappa_{s,L}$ and $L_s$ re the separable closure of $\kappa$ in $L$ and $K$, we have a tower of fields
    \small
    $$
        \begin{tikzcd}[column sep=0.5cm]
            K\\
            L_s
                    \arrow[u,"{[K:L]_i}",no head]\\
            L
                    \arrow[u,"{[K:L]_s}",no head]
                &\kappa_s
                    \arrow[lu,no head]
                    \arrow[luu,"{[K:\kappa]_i}"',no head]\\
            {\kappa_{s,L}}
                    \arrow[u,"{[L:\kappa]_i}",no head]
                    \arrow[ru,no head]\\
            \kappa
                    \arrow[u,"{[L:\kappa]_s}",no head]
                    \arrow[ruu,"{[K:\kappa]_s}"',no head]
        \end{tikzcd}
    $$
    \normalsize
    where $\kappa_s\subseteq L_s$ because $K$ is separable over $\kappa$, hence over $L$. Moreover, $L_s/\kappa_s$ is purely inseparable by Theorem~\ref{thm:sep-and-p.insep-closures}. This implies that $\kappa_s$ is the separable closure of $\kappa$ in $L_s$. And since $L$ is the purely inseparable closure of $\kappa_{s,L}$ in $L_s$, the previous lemma allows us to conclude that $[L:\kappa]_i=[L_s:\kappa_s]$. Therefore, $[K:\kappa]_i=[K:L]_i[L:\kappa]_i$. The other equality is a direct consequence of this one.
\end{solution}

\section{Galois Theorem}

\begin{thm}\label{thm:galois-fundamental-thm}
    {\rm[Galois Fundamental Theorem]}
    Let\/ $K/\kappa$ be a finite Galois extension, and\/ $G = \Gal(K/\kappa)$. Then there is a\/ $1$-$1$ inclusion-reversing correspondence between intermediate fields of\/ $K / \kappa$ and subgroups of\/ $G$, given by\/ $F \mapsto \Gal(K/F)$ and\/ $H \mapsto \mathcal F(H)$. Furthermore, if\/ $F \mapsto H$, then\/ $[K: F] = |H|$ and\/ $[F : \kappa] = [G: H]$. Moreover, $H$ is normal in\/ $G$ if, and only if, $F$ is Galois over\/ $\kappa$. When this occurs, $\Gal( F / \kappa) \cong G / H$.
\end{thm}

\begin{proof}
    According to Theorem~\ref{thm:inclusion-reversing-correspondence}, there is an inclusion-reversing bijective correspondence $F\mapsto\Gal(K/F)$ between intermediate fields $F$ and certain subgroups of $G$. In this case, moreover, if $F$ is such a field, $K/F$ is also Galois, which means that $F=\mathcal F(\Gal(K/F))$. In particular, every intermediate field is a fixed field. Conversely, if $H$ is a subgroup of $G$, Corollary~\ref{cor:galois-group=dim}, then $H=\Gal(K/F)$, with $F=\mathcal F(H))$ and $|H|=|K:F|$. Since $\mathcal F(G)=\kappa$, it follows that $|G|=[K:\kappa]$. Then,
    $$
        |G:H| = |G|/|H|=[K:\kappa]/[K:F]=[F:\kappa].
    $$
    Take $a\in F$. If $c$ is a root of $\pmin_{\kappa,a}$, by Proposition~\ref{prop:algebraic-minimal} there exists a (unique) morphism $\sigma\colon\kappa(a)\to\kappa(c)$ such that $\sigma(a)=c$. According to Corollary~\ref{cor:extension-to-normal}, there exists $\omega\in G$ such that $\sigma\subseteq\omega$. Thus, given $\chi\in H$, we have
    $$
        \omega^{-1}\circ\chi(c)=\omega^{-1}\circ\chi\circ\omega(a).
    $$
    If $H$ is normal, then $\omega^{-1}\circ\chi\circ\omega\in H$ and so $\omega^{-1}\circ\chi(c)=a$, hence $\chi(c)=c$. It follows that $c\in\mathcal F(H)=F$, which implies that $F/\kappa$ is Galois.

    Conversely, if $F/\kappa$ is Galois, the restriction map $\rho\colon G\to H$, defined by $\omega\mapsto\omega_F$, where $\omega_F\subseteq\omega$, is well-defined by Theorem~\ref{thm:normal-equivalences}~b). Since $\rho$ is a morphism of groups, we have $F=\mathcal F(H)=\ker\rho$, which is normal.
    
\end{proof}

\begin{xmpls}\label{xmpls:galois-thm}${}$
    \begin{enumerate}[a), font=\upshape]
        \item Let $G$ be a group of order $6$. If $G$ has some element of order $6$, then $G=\Z_6$. Otherwise, take an element $\tau$ of order $2$ and another $\sigma$ of order $3$ (take, for instance, generators of the corresponding Sylow subgroups). If $\tau\sigma=\sigma\tau$, then $(\tau\sigma)^2=\sigma^2$, which has order $3$, i.e., $\ord(\tau\sigma)=6$, in contradiction with our assumption. Thus, $\tau\sigma\ne \sigma\tau$ and we have $G=\set{1, \tau, \sigma, \sigma^2, \tau\sigma, \sigma\tau}\cong S_3$ under $\tau\mapsto (1,2)$ and $\sigma\mapsto(1,2,3)$.

        Now consider $\Q(\sqrt[3]2,\omega)$, where $\omega=e^{2\pi i/3}$. As we have seen, this is an extension of $\Q$ of degree $6$. Then its Galois group $G$ is $S_3$. It cannot be $\Z_6$ because the intermediate field $\Q(\sqrt[3]2)$ is not normal. To realize the isomorphism put $\alpha=\sqrt[3]2$ and consider $\tau$ and $\sigma$, defined by
        \begin{align*}
            \tau\colon\omega&\mapsto\omega^2,
                \quad\alpha\mapsto\alpha\\
            \sigma\colon\omega&\mapsto\omega,\phantom{{}^2}
                \quad\alpha\mapsto\alpha\,\omega.
        \end{align*}
        The subgroups of $G$ are easily identified, so we can use them to find the intermediate fields:
        \small
        $$
            \begin{array}{c|c|c|c|c|c}
            \vphantom{\langle_{|_{|_|}}}\langle 1 \rangle
                & \langle \tau \rangle
                & \langle \sigma \rangle
                & \langle \tau\sigma \rangle
                & \langle \sigma\tau \rangle
                & G \\
            \hline
            \vphantom{|^{|^{|^|}}}\Q(\sqrt[3]{2}, \omega)
                & \Q(\sqrt[3]{2})
                & \Q(\omega)
                & \Q(\sqrt[3]{2}\,\omega)
                & \Q(\sqrt[3]{2}\,\omega^2)
                & \Q
            \end{array}
        $$
        \normalsize
        For instance, the equation
        $$
            \tau\sigma(\alpha\,\omega)
                = \tau(\alpha\,\omega^2)
                = \alpha\,\omega
        $$
        shows that $\Q(\alpha\,\omega)\subseteq\mathcal F\langle\tau\sigma\rangle$. On the other hand [cf.~Example~\ref{xmpl:galois-extensions}],
        $$
            \pmin_{\Q,\alpha}(x)=x^3-2=(x-\alpha)(x-\alpha\omega)(x-\alpha\omega^2),
        $$
        which shows that $\pmin_{\Q(\alpha\omega),\alpha}(x)=(x-\alpha)(x-\alpha\omega^2)$. Thus, equality is attained because
        $$
            [\Q(\alpha,\omega):\Q(\alpha\,\omega)]
            = 2 = |\langle\tau\sigma\rangle|.
        $$
        The other cases can be derived in a similar way.

        \item Let $K = \Q(\sqrt2,\sqrt3)$. Then $K$ is the splitting field of $\{x^2 - 2, x^2 - 3\}$ over $\Q$, hence Galois. Since $[K:\Q]=4$, the Galois group $G=\Gal(K/\Q)$, must be $\Z_4$ or $\Z_2\oplus\Z_2$ (there aren't other groups of order $4$). To see that $G=\Z_2\oplus\Z_2$, consider the following automorphisms:
        \begin{align*}
            \sigma(\sqrt2)&=-\sqrt2,\quad\sigma(\sqrt3)=\sqrt3\\
            \tau(\sqrt3)&=-\sqrt3,\quad\tau(\sqrt2)=\sqrt2.
        \end{align*}
        Then $G=\set{\id,\sigma,\tau,\sigma\tau}$, which is not cyclic. Since $G$ has $4$ subgroups, $K/\Q$ has $4$ intermediate fields, the proper ones being $\Q(\sqrt2)$ and $\Q(\sqrt3)$.

        \item Let $\kappa=\C(t)$ and $f(x)=x^n-t$, which is irreducible in $\kappa[x]$ by Eisenstein Criterion~\ref{lem:eisenstein}. Let $\alpha$ be a root of $f$ and $\omega=e^{2\pi i/n}$ a primitive $n$th root of unit in $\C$. Then, the $n$ roots of $f$ are $\set{\alpha\omega^i\mid0\le i<n}\subseteq\kappa(\alpha)$. Then, $\kappa(\alpha)$ is Galois and $G=\Gal(\kappa(\alpha)/\kappa)=\Z_n$. By Corollary~\ref{cor:isomorphic-split-fields} there exists $\sigma\in G$ such that $\sigma(\alpha)=\alpha\omega$. Since $\sigma^i(\alpha)=\alpha\omega^i$, it follows that $\ord(\sigma)=n$, i.e., $G=\langle\sigma\rangle$. The subgroups of $G$ are of the form $\langle\omega^m\rangle$, with $m\mid n$. Take one of these subgroups and put $n=mk$. Then 
        $$
            \sigma^m(\alpha^k)=\sigma^m(\alpha)^k=(\alpha\omega^m)^k
                =\alpha^k,
        $$
        which shows that $\kappa(\alpha^k)\subseteq\mathcal F\langle\sigma^m\rangle$. Equality is attained because there must exist $m'$ such that $\kappa(\alpha^k)=\mathcal F\langle\sigma^{m'}\rangle$, with $m'\mid n$, and this implies that $\alpha^k=\sigma^{m'}(\alpha^k)=\alpha^k\omega^{m'k}$, i.e., $n=m'k$ or $m=m'$, as claimed.
    \end{enumerate}
\end{xmpls}

\begin{thm} {\rm[Natural Irrationalities]}
     Let\/ $K$ be a finite Galois extension of\/ $\kappa$, and let\/ $L$ be an arbitrary extension of\/ $\kappa$. Then\/ $K L / L$ is Galois and\/ $\Gal(KL / L) \cong \Gal(K / K \cap L)$. Moreover, $[KL : L] = [K : K \cap L]$.
     \small
     $$
        \begin{tikzcd}[column sep=-0.2cm]
                &KL\\
            K
                    \arrow[ru,no head]
                &&L
                    \arrow[lu,"\text{\rm Galois}"',no head,pos=0.75]
                    \arrow[lu,"{[K:K\cap L]}"',no head,pos=0.1]\\
                &K\cap L
                    \arrow[lu,no head]
                    \arrow[ru,no head]\\
                &\kappa
                    \arrow[u,no head]
                    \arrow[luu,"\text{\rm Galois}",no head,bend left]&
        \end{tikzcd}
     $$
     \normalsize
\end{thm}

\begin{proof}
    Since $K$ is normal, there exists $S\subseteq\kappa[x]$ such that $K$ is the splitting field of~$S$. Therefore, $KL$ is the splitting field of $S$ as a subset of $L[x]$. On the other hand, since $KL=L(K)$  [cf.~Problem~\ref{probl:KL}], to verify that $KL$ is separable over $L$ it is enough to note that every irreducible and separable polynomial $f(x)\in\kappa[x]$ is also separable over $L$. It follows that $KL/L$ is Galois.

    Define
    \begin{align*}
        \rho\colon\Gal(KL/L)&\to\Gal(K/K\cap L)\\
        \sigma&\mapsto\sigma_K,
    \end{align*}
    where $\sigma_K\subseteq\sigma$ is well-defined because $K$ is normal over $\kappa$, hence over $K\cap L$ [cf.~Theorem~\ref{thm:normal-equivalences}]. Thus, $\rho$ is a morphism of groups. It is injective because $\sigma_K=\id_K$ implies that both $K$ and $L$ are fixed by $\sigma$, which means that $\sigma=\id_{KL}$.

    Consider the fixed field $F=\mathcal F(\im\rho)$. By Theorem~\ref{thm:galois-fundamental-thm}, $\Gal(K/F)=\im\rho$. Therefore, to verify that $\rho$ is an epimorphism, we have to show that $F=K\cap L$. The inclusion $K\cap L\subseteq F$ is trivial. For the other inclusion, take $c\in F$. By definition, $\sigma(c)=c$ for all $\sigma\in\Gal(KL/L)$. It follows that $c\in L$ and so $c\in K\cap L$ because $F\subseteq K$.

    The equality $[KL:L]=[K:K\cap L]$ is a direct consequence of the isomorphism of groups.
\end{proof}

\begin{defn}
    A field extension $K / \kappa$ is called \textsl{simple} if $K = \kappa(\alpha)$ for some $\alpha \in K$.
\end{defn}

\begin{thm}
    {\rm[Primitive Element Theorem]} A finite extension\/ $K / \kappa$ is simple if, and only if, there are only finitely many fields\/ $F$ with\/ $\kappa \subseteq F \subseteq K$.
\end{thm}

\begin{proof}
    If $\kappa$ is finite, then $K$ is finite and so $K^*$ is a cyclic group. Therefore, if $\alpha$ is a generator of $K^*$, clearly $K=\kappa(\alpha)$.

    Let's assume that $\kappa$ is infinite and that there are finitely many intermediate fields in the $K/\kappa$ extension. In particular, $K=\kappa(c_1,\dots,c_n)$ and we can proceed by induction on $n$. The case $n=1$ is self-fulfilled. If $n>1$, we can apply the induction hypothesis to $\kappa(a_1,\dots,a_{n-1})$ and reduce ourselves to the case $n=2$.
    
    Given $a\in\kappa$, let's introduce $F_a=\kappa(c_1+ac_2)$. By hypothesis, there are $a\ne b$ such that $F_a=F_b$. In particular,
    $$
        c_2 = \frac{(c_1+bc_2)-(c_1+ac_2)}{b-a}\in F_a
    $$
    and so $c_1=(c_1+ac_2)-ac_2\in F_a$, which implies that $K=\kappa(c_1,c_2)=F_a$ is simple.

    For the converse, suppose that $K=\kappa(\alpha)$. Given an intermediate field $F$, take $g(x)=\pmin_{F,\alpha}(x)$ and write $g(x)=x^n+c_{n-1}x^{n-1}+\cdots+c_0\in F[x]$. Then, defining $F_0=\kappa(c_0,\dots,c_{n-1})$, we get $\pmin_{F_0,\alpha}\mid g$. Therefore,
    $$
        [K:F] = n \ge [K:F_0]=[K:F][F:F_0],
    $$
    which implies that $[F:F_0]=1$, i.e., $F=F_0$. In particular, $F$ is determined by the minimal $g$. Since $g\mid\pmin_{\kappa,\alpha}$ there are only finitely many possibilities for $g$, hence for $F$.
    
\end{proof}

\begin{cor}
    If\/ $K / \kappa$ is finite and separable, then\/ $K = \kappa(\alpha)$ for some\/ $\alpha \in K$.
\end{cor}

\begin{proof}
    By Corollary~\ref{cor:separable-iff-in-Galois}, $K$ is contained in a finite Galois extension $L$ of~$\kappa$. Since there are finitely many subgroups of $\Gal(L/\kappa)$, there are only finitely many subfields of $L$, hence of $K$. The conclusion is a direct consequence of the theorem.
\end{proof}

\subsection{Normal Closure}

\begin{defn}
    Let $K$ be an algebraic extension of $\kappa$. The \textsl{normal closure} of $K / \kappa$ is the splitting field over $\kappa$ of the set $\{\pmin_{\kappa, \alpha} \mid \alpha \in K\}$ of minimal polynomials of elements of $K$.
\end{defn}

\begin{prop}\label{prop:normal-closure-properties}
    Let\/ $K$ be an algebraic extension of\/ $\kappa$, and let\/ $N$ be the normal closure of\/ $K / \kappa$. 
    \begin{enumerate}[a), font=\upshape]
        \item The field\/ $N$ is a normal extension of\/ $\kappa$ containing\/ $K$. Moreover, if\/ $L$ is a normal extension of\/ $\kappa$ with\/ $K \subseteq L \subseteq N$, then\/ $L = N$. 
        \item If\/ $K = \kappa(\alpha_1, \dots, \alpha_n)$, then\/ $N$ is the splitting field of the polynomials\/ $\pmin(\kappa, \alpha_1), \dots, \pmin(\kappa, \alpha_n)$ over\/ $\kappa$.
        \item If\/ $K / \kappa$ is a finite extension, then so is\/ $N / \kappa$. 
        \item If\/ $K / \kappa$ is separable, then\/ $N / \kappa$ is Galois.
    \end{enumerate}
\end{prop}

\needspace{2\baselineskip}
\begin{proof}${}$
    \begin{enumerate}[a), font=\upshape]
        \item By definition, $N$ is normal over $\kappa$. Take an intermediate field $L$ in $N/K$ that is normal over $\kappa$. If $c\in K$, then $\min_{\kappa,c}$ splits over $L$. In particular, $c\in L$. It follows that $K\subseteq L$. Equality is attained because the other inclusion is given.

        \item Let $S=\set{\pmin_{\kappa,\alpha_i}\mid 1\le i\le n}$. If $L$ is the splitting field of $S$ over $\kappa$, then $L\subseteq K$ and so $L=N$ by part ~a).

        \item This follows from Proposition~\ref{prop:splitting-field-n!} by part~b) and induction on~$n$.

        \item This is a consequence of Corollary~\ref{cor:separable-iff-in-Galois} and part~a).
    \end{enumerate}
    
\end{proof}

\begin{cor}
    Let\/ $K$ be an algebraic extension of\/ $\kappa$, and let\/ $N$ be the normal closure of\/ $K / \kappa$. If\/ $N'$ is any normal extension of\/ $\kappa$ containing\/ $K$, then there is a\/ $\kappa$-morphism from\/ $N$ to\/ $N'$. In consequence, if\/ $N'$ does not contain any proper subfield normal over\/ $\kappa$ that contains\/ $K$, then\/ $N$ and\/ $N'$ are\/ $\kappa$-isomorphic.
\end{cor}

\begin{proof}
    Let $N'$ be as stated. If $S=\set{\pmin_{\kappa,\alpha}\mid\alpha\in K}$, by definition, $N$ is the splitting field of $S$. Since the elements of $S$ split over $N'$, there is a subfield of $N'$, that is also a splitting field of $S$. Therefore, we can invoke the Isomorphism Extension Theorem~\ref{thm:isomorphism-extension} to conclude that there is an extension $\omega\colon N\to N'$ of the identity $\id_\kappa\colon\kappa\to\kappa$. The final consequence follows from considering the field $\omega(N)\subseteq N'$.
\end{proof}

\begin{xmpl}
    The normal closure of $\Q(\sqrt[3]2)$ is the splitting field of $x^3-2$, i.e., $\Q(\sqrt[3]2,\omega)$, where $\omega=e^{2\pi i/3}$ is a primitive root of unit in $\C$.
\end{xmpl}

\begin{rem}
    If $K$ is an extension of $\kappa$, and if $\alpha \in K$ has a minimal polynomial $f(x)$ over $\kappa$, then the normal closure of $\kappa(\alpha) / \kappa$ is the field $\kappa(\alpha_1, \alpha_2, \dots, \alpha_n)$, where the $\alpha_i$ are the roots of $f(x)$.
\end{rem}

\begin{prop}
    Suppose that\/ $K / \kappa$ is a finite separable but not normal extension with normal closure\/ $N$. Let\/ $G = \Gal(N / \kappa)$ and\/ $H = \Gal(N / K)$. Then\/ $\op{Core}_G(H)=\bigcap_{\sigma \in G} H^\sigma = \set\id$.
\end{prop}

\begin{proof}
    Since $K$ is not Galois over $\kappa$, $H$ is not normal in $G$. The minimality of $N$ as a normal extension of $\kappa$ containing $K$ translates via the fundamental theorem into the following between $G$ and $H$: The largest normal subgroup of $G$ contained in $H$ is $\set{\id}$. Indeed, if $H' \subseteq H$ is a normal subgroup of $G$, then $L = \mathcal F(H')$ is an extension of $K$ that is normal over $\kappa$. But, as $L \subseteq N$, minimality of $N$ implies that $L = N$, so $H' = \set\id$.
\end{proof}


\subsection{The Fundamental Theorem of Algebra}

\begin{lem}${}$
    \begin{enumerate}[a), font=\upshape]
        \item Every complex number has a real square root.
        \item The only odd degree extension of\/ $\R$ is\/ $\R$ itself.
        \item There is no extension of $\C$ with degree~$2$.
    \end{enumerate}
\end{lem}

\begin{proof}${}$
    \begin{enumerate}[a), font=\upshape]
        \item If $z=\rho e^{i\theta}$, then $w=\sqrt\rho e^{i\theta/2}$ satisfies $w^2=z$.

        \item Let $K$ be an extension of $\R$ with $[K:\R]$ odd. Take $c\in K\setminus\R$. Then $\pmin_{\R,c}$ has odd degree. In particular it has a root in $\R$. And since it is irreducible, it must be linear, i.e., $c\in\R$.

        \item Suppose that $K$ is an extension of $\C$ with $[K:\C]=2$. Pick $c\in K\setminus\C$. Then $\deg\pmin_{\C,c}=2$, which is impossible by part~b). \qedhere
    \end{enumerate}
\end{proof}

    \begin{thm} {\rm[Fundamental Theorem of Algebra]} The field\/ $\C$ is algebraically closed. 
\end{thm}

\begin{proof}
    Let $L$ be a proper finite extension of $\C$. Then $L$ is also a finite extension of $\R$ and we can consider a normal closure $N$ of $L/\R$ [cf.~Proposition~\ref{prop:normal-closure-properties}], which is Galois over $\R$. Let $G=\Gal(N/\R)$. Given that $[\C:\R]\mid[N:\R]$, we see that $[N:\R]$ is even. It follows that $|G|$ is even. Let $P$ be a Sylow $2$-subgroup of $G$ and let $F=\mathcal F(P)$ denote its fixed field. Since $[G:P]$ is odd, $[F:\R]=[G:P]$ is odd. It follows from the lemma that $F=\R$, i.e., $G=P$. In consequence, $G$ and $\Gal(N/\C)$ are Sylow $2$-groups. Thus, if $M$ is a maximal subgroup of $\Gal(N/\C)$, we deduce that $[\Gal(N/\C):M]=2$. But this is impossible by the lemma because $\mathcal F(M)$ would be a quadratic extension of $\C$.
\end{proof}

\subsection{Integral Extensions}

Recall that an \textsl{integral domain} is a domain (i.e., a commutative ring) with no zero divisors.

\begin{defns}${}$
    \begin{enumerate}[-]
        \item Let $A$ be a domain. An $A$-module $M$ is \textsl{faithful} if $aM=0\implies a=0$. 

        \item Let $A\subseteq B$ be a domain extension. An element $\alpha\in B$ is \textsl{integral over $A$} if there exists a monic polynomial $f(x)\in A[x]$ such that $f(\alpha)=0$.
    \end{enumerate}
\end{defns}

\begin{thm}\label{thm:integral-equivalences}
    Let\/ $A\subseteq B$ be an extension of integral domains. The following conditions are equivalent
    \begin{enumerate}[a), font=\upshape]
        \item $\alpha\in B$ is integral over\/ $A$.
        \item The\/ $A$-module\/ $A[\alpha]$ is finitely generated.
        \item There exists a faithful finitely generated\/ $A[\alpha]$-module.
    \end{enumerate}
\end{thm}

\begin{proof}${}$
    \begin{enumerate}[a), font=\upshape]
        \item $\Rightarrow$ b)By hypothesis there exist $a_0,\dots,a_{n-1}\in A$ such that
        $$
            \alpha^n=a_{n-1}\alpha^{n-1}+\cdots+a_1\alpha+a_0.
        $$
        Therefore, every element in $A[\alpha]$ can be written as a linear combination with coefficients in $A$ of $1,\alpha,\alpha^2,\dots,\alpha^{n-1}$.

        \item $\Rightarrow$ c) $A[\alpha]$ is faithful because $1\in A[\alpha]$.

        \item $\Rightarrow$ a) Let $\zeta_1,\dots,\zeta_n$ be generators of a faithful finitely generated $A[\alpha]$-module $M$. There is a matrix $T\in A^{n\times n}$ such that
        $$
            \alpha\begin{bmatrix}
                \zeta_1\\
                \vdots\\
                \zeta_n
            \end{bmatrix}
            =
            T\begin{bmatrix}
                \zeta_1\\
                \vdots\\
                \zeta_n
            \end{bmatrix}.
        $$
        In consequence $\det(\alpha\op{Id}-T)=0$, i.e., $\alpha$ is a root of the characteristic polynomial $\chi_T(x)=\det(x\op{Id}-T)$, which is monic and its coefficients belong in~$A$. %\qedhere
    \end{enumerate}
\end{proof}

\begin{defn}
    An element $\alpha$ that satisfies the conditions of the theorem is called \textsl{integral over $A$}. If $A\subseteq B$, we say that $B$ is integral over $A$ when every element of $B$ is integral over~$A$.
\end{defn}

\begin{cor}\label{cor:finite-integral}
    Let\/ $B=A[\alpha_1,\dots,\alpha_n]$ be a finitely generated\/ $A$-algebra. If\/ $\alpha_1,\dots,\alpha_n$ are integral over\/ $A$ then\/ $B$ is finitely generated as an\/ $A$-module.
\end{cor}

\begin{proof}
    Since $\alpha_n$ is integral over $A$, it is integral over $C=A[\alpha_1,\dots,\alpha_{n-1}]$. By definition, $B$ is a finitely generated $C$-module. Induction on $n$ implies that $C$ is a finitely generated $A$-module. Hence, $B$ is a finitely generated $A$-module. Indeed, if $\set{\beta_1,\dots,\beta_h}$ generate $B$ as $C$-module and $\set{\gamma_1,\dots,\gamma_k}$ generate $C$ as $A$-module, then $\set{\beta_i\gamma_j\mid1\le i\le h,\;1\le j\le j}$ generate $B$ as $A$-module.
    
\end{proof}

\begin{cor}\label{cor:integral-ring}
    Let\/ $A\subseteq C$ be an extension of integral domains. Then, the set of elements in\/ $C$ that are integral over\/ $A$ form a subring of\/~$C$. In particular, if\/ $\set{\alpha_1,\dots,\alpha_n}$ are elements of\/ $C$ integral over\/ $A$ then\/ $A[\alpha_1,\dots,\alpha_n]$ is integral over\/~$A$.
\end{cor}

\begin{proof}
    Take $\alpha_1$ and $\alpha_2$ in $C$ and suppose that they are integral over $A$. By Corollary~\ref{cor:finite-integral}, $B=A[\alpha_1,\alpha_2]$ is a finitely generated $A$-module. Therefore, $B$ is finitely generated both as an $A[\alpha_1\alpha_2]$-module and as an $A[\alpha_1+\alpha_2]$-module. Since $B$ is faithful, $\alpha_1\alpha_2$ and $\alpha_1+\alpha_2$ are integral over~$A$.
    
\end{proof}

\begin{cor}
    Let\/ $A$, $B$ and\/ $C$ be integral domains with\/ $A, B\subseteq B$. Then the following statements hold true:
    \begin{enumerate}[a), font=\upshape]
        \item If\/ $B\subseteq C$ then\/ $C$ is integral over\/ $A$ if, and only if, $C$ is integral over\/ $B$ and\/ $B$ integral over\/~$A$.

        \item Assume that\/ $B$ and\/ $C$ are subrings of a common ring. If\/ $B$ is integral over\/ $A$, then\/ $C[B]$ is integral over\/~$C$.
    \end{enumerate}
\end{cor}

\begin{proof}${}$
    \begin{enumerate}[a), font=\upshape]
        \item The \textit{only if\/} part is trivial. For the \textit{if\/} part take $\alpha\in C$. By hypothesis, there exist $b_{n-1},\dots,b_0\in B$ such that
        $$
            \alpha^n+b_{n-1}\alpha^{n-1}+\cdots+b_1\alpha
                + b_0=0.
        $$
        Put $\tilde B=A[b_0,\dots,b_{n-1}]$. By Corollary~\ref{cor:finite-integral}, $\tilde B$ is a finitely generated $A$-module and $\tilde B[\alpha]$ a finitely generated $\tilde B$-module. It follows that $\tilde B[\alpha]$ is a finitely generated $A$-module, and so $\alpha$ is integral over~$A$. Given that $\alpha$ was arbitrarily chosen, $C$ is integral over~$A$.

        \item Every element of $B$ is integral over $A$, therefore over $C$. Thus, by Corollary~\ref{cor:integral-ring}, every element of $C[B]$ is integral over $A$.
    \end{enumerate}
\end{proof}

\begin{prop}
    Let\/ $A$ be an integral domain and\/ $K$ its quotient field. Let\/ $\alpha$ be algebraic over\/ $K$. Then there exists an element\/ $c\in A$ such that\/ $c\alpha$ is integral over\/~$A$.
\end{prop}

\begin{proof}
    If $a_n\alpha^n + a_{n-1}\alpha^{n-1} + \cdots + a_1\alpha + a_0 = 0$ for some $a_n,\dots,a_0\in A$, after multiplication by $a_n^{n-1}$ we get 
    $$
        (a_n\alpha)^n + a_{n-1}(a_n\alpha)^{n-1}
            + \cdots + a_n^{n-2}(a_n\alpha)+ a_n^{n-1}a_0
            = 0.
    $$
\end{proof}

\subsection{Discriminants}

\begin{defn}
    Let $f(x)\in\kappa[x]$ be a nonconstant polynomial. If $K$ is a splitting field of $f$ and $\alpha_1,\cdots,\alpha_n$ are the roots of $f$ in $K$ then, the \textsl{Vandermonde product} of $f$ is
    $$
        \Delta(f)=\prod_{1\le i<j\le n}\alpha_j-\alpha_i,
    $$
    which depends on the numbering of the roots. The square of the Vandermonde product, which doesn't depend on the numbering, is the \textsl{discriminant} of $f$
    $$
        D(f)=\Delta(f)^2.
    $$
\end{defn}

\begin{lem}
    With the preceding notations, if\/ $\sigma\in\Gal(K/\kappa)$, we have
    \begin{enumerate}[a), font=\upshape]
        \item $\Delta(\sigma(f))=\sigma(\Delta(f))=\sg(\sigma)\Delta(f)$.

        \item $D(\sigma(f))=\sigma(D(f))=D(f)$.
    \end{enumerate}
\end{lem}

\needspace{2\baselineskip}
\begin{proof}${}$
    \begin{enumerate}[a), font=\upshape]
        \item The first equality is trivial because the roots of $\sigma(f)$ are $\sigma(\alpha_1),\dots,\sigma(\alpha_n)$. The second can be seen by induction on the number of transpositions required to decompose $\sigma$.

        \item This is a direct consequence of part a).
    \end{enumerate}
\end{proof}

\begin{ntn}\label{ntn:lexicographic}
    Let $A$ be a ring and $x_1,\dots,x_n$ indeterminates over $A$. In what follows we will denote the lexicographic order between monomials of the same degree by $<_L$. This means that
    $$
        x_1^{d_1}\cdots x_n^{d_n} <_L x_1^{e_1}\cdots x_n^{e_n}
        \iff (\exists\, k)\; d_i=e_i \text{\rm\ for } 1\le i<k \text{\rm, and }d_k<e_k.
    $$
\end{ntn}

\begin{lem}\label{lem:lexicographic-compatibility}
    Let $M$ and $N$ be monomials of degree $m$ and $Q$ and $R$ monomials of degree $r$. If $M<_L N$ and $Q<_L R$ then $MQ<_L NR$.
\end{lem}

\begin{proof}
    Write $M=x_1^{d_1}\cdots x_n^{d_n}$, $N=x_1^{e_1}\cdots x_n^{e_n}$ and suppose that $M<_LN$.
    
    Firstly observe that it is enough to show that $MQ<_LNQ$. Secondly, that it will follow if we prove that $Mx_h<_LNx_h$ for $1\le h\le n$. Let $k$ be like in Notation~\ref{ntn:lexicographic}. If $h>k$, there is nothing to prove. If $h<k$, then $d_h+1=e_h+1$ and equality remains in the other indexes below $k$. If $h=k$, equality remains in all indexes below $k$ and $d_k+1<e_k+1$ at index $k$.
\end{proof}

\begin{lem}\label{lem:symmetric-lexicographic-head}
    Given\/ $j\ge1$, the highest monomial (for the lexicographic order) of\/ $s_1^{e_1}\cdots s_j^{e_j}$ is
    $$
        x_1^{e_1+\cdots+e_j}x_2^{e_2+\cdots+e_j}\cdots x_j^{e_j}.
    $$
\end{lem}

\begin{proof}
    Induction on $i$ shows that the maximum monomial of $s_i$ is $x_1\cdots x_i$. Therefore, the conclusion is a direct consequence of the previous lemma.
\end{proof}

\begin{thm}\label{thm:symmetric-from-elementary-symmetric}
    Every symmetric polynomial is expressible as a polynomial in the elementary symmetric polynomials\/~$s_1,\dots,s_n$.
\end{thm}

\begin{proof}
    Since every homogeneous component of a symmetric polynomial is symmetric, it is enough to consider the case where the symmetric polynomial is homogeneous. Let $f$ be such a polynomial. Let $M$ be the maximum monomial (for the lexicographic order) occurring in~$f$, say $M=cx_1^{d_1}\cdots x_n^{d_n}$, where $c$ is a constant. The symmetry of $f$ implies that for every permutation $\sigma\in S_n$ the monomial $N=cx_{\sigma(1)}^{d_1}\cdots x_{\sigma(n)}^{d_n}$ also occurs in~$f$. Thus, if we re-arrange the exponents so that $d_1\ge d_2\ge\cdots\ge d_n$, we deduce that this must be the case with~$M$.

    By Lemma~\ref{lem:symmetric-lexicographic-head}, we know that the maximal monomial occurring in
    $$
        s(x_1,\dots,x_n)=s_1^{d_1-d_2}\cdots s_{n-1}^{d_{n-1}-d_n}s_n^{d_n}
    $$
    is $x_1^{d_1}\cdots x_n^{d_n}$. In particular, $f(x_1,\dots,x_n)-cs(x_1,\dots,x_n)$ is zero, or has a maximum monomial that is smaller than~$M$. In the first case we are done. In the second, we proceed with the difference.
\end{proof}

\begin{thm}\label{thm:square-discriminant}
    Let $f\in\kappa[x]$ be a nonconstant separable polynomial of degree $n$ and let $K$ be a splitting field of $f$. If $G=\Gal(K/\kappa)$ then
    \begin{enumerate}[a), font=\upshape]
        \item $D(f)\in\kappa$.
        \item If $\fchar(\kappa)\ne2$, the subfield of $K$ corresponding to $A_n\cap G$ is $\mathcal F(\Delta(f))$. In consequence,
        $$
            G\subseteq A_n\iff \Delta(f)\in\kappa\iff D(f)\text{\rm\ is a square in }\kappa.
        $$
    \end{enumerate}
\end{thm}

\begin{proof}${}$
    \begin{enumerate}[a), font=\upshape]
        \item By the previous lemma, $D(f)\in\mathcal F(G)$, and by the Fundamental Theorem~\ref{thm:galois-fundamental-thm} $\mathcal F(G)=\kappa$.

        \item By the lemma, $\sigma$ fixes $\Delta(f)$ if, and only if, $\sigma\in A_n$, i.e.,
        $$
            A_n\cap G=\Gal(K/\kappa(\Delta(f))).
        $$
        Hence, $G\subseteq A_n\iff\Delta(f)\in\kappa$, i.e., $D(f)$ is a square in~$\kappa$.
    \end{enumerate}
    
\end{proof}

\begin{thm}
    Let $\set{x_1,\dots,x_n}$ be a set of indeterminates over $\Z$. Then the $n\times n$ matrix $V_n=(x_j^{i-1})_{1\le i,j\le n}$ satisfies
    \begin{enumerate}[a), font=\upshape]
        \item $\det V_n=\prod_{i<j}x_j-x_i$,
        \item $V_nV_n^T$ is the symmetric matrix $(u_{i+j-2})_{i,j}$, where $u_k=x_1^k+\cdots+x_n^k$, for $0\le k\le 2n-2$.
    \end{enumerate}
\end{thm}

\begin{proof}${}$
    \begin{enumerate}[a), font=\upshape]
        \item Let's better visualize $V_n$ as
        $$
            \det V_n = \left(
                \begin{array}{lllll}
                    1      & 1      & 1      & \cdots & 1\\
                    x_1    & x_2    & x_3    & \cdots & x_n\\
                    x_1^2  & x_2^2  & x_3^2  & \cdots & x_n^2\\
                    \vdots & \vdots & \vdots & \ddots & \vdots\\
                    x_1^{n-1} & x_2^{n-1} & x_3^{n-1} & \cdots & x_n^{n-1}
                \end{array}
                \right)
        $$
        Since the determinant takes an element in every row, $\Delta=\det V_n$ is a polynomial in $\Z[x_1,\dots,x_n]$ of total degree $1+2+\cdots+n-1=(n-1)n/2$.
        
        Considering $\Delta=\Delta(x_j)$ as a polynomial in the single variable $x_j$, we see that $\Delta(x_i)=0$, i.e., for $i\ne j$, $x_i$ is a root of $\Delta(x_j)$. Thus, $x_j-x_i\mid\Delta$.

        If $\set{h,k}\ne\set{i,j}$, the very same argument shows that $x_h-x_j\mid\Delta$. Thus, $\prod_{i<j}x_j-x_i\mid\Delta$, and since both sides have the same total degree, we deduce that $\Delta=c\prod_{i<j}x_j-x_i$ for some constant~$c$. Now, the coefficient of the monomial $x_n^{n-1}$ in $D$ is $(-1)^{n+n}=1$, as it can be seen from the expansion of the determinant by the $n$th column. Since the coefficient of $x_n^{n-1}$ in the product $\prod_{i<j}x_j-x_i$ is also $1$, we deduce that~$c=1$.

        \item Given $i\ne j$ product of the $i$th and $j$th rows $F_i$ and $F_j$ of $V_n$ is
        $$
            \langle F_i,F_j\rangle
                = \sum_{k=1}^nx_k^{i-1}x_k^{j-1}=u_{i+j-2}.
        $$
    \end{enumerate}
\end{proof}

\begin{cor}\label{cor:discriminant}
    The discriminant\/ $D(f)$ of a polynomial\/ $f(x)\in\kappa[x]$ is a polynomial in the coefficients of\/~$f$.
\end{cor}

\begin{proof}
    The entries of $V_nV_n^T$ are the symmetric polynomials $u_k$ defined in the theorem. By Theorem~\ref{thm:symmetric-from-elementary-symmetric}, eac $u_k$ is a polynomial in the elementary symmetric polynomials in the roots of~$f$, which are constant multiples (by $\pm1$) of the coefficients of~$f$.
\end{proof}

\begin{ntn}
    If $f\in\kappa[x]$ is separable, and $K$ is a splitting field for $f$, then the \textsl{Galois group of $f$} is $G_f=\Gal(K/\kappa)$. 
\end{ntn}

\begin{thm}\label{thm:galois-mod-p}
     Let\/ $f\in\Z[x]$ be a monic polynomial which doesn't have repeated roots and suppose that\/ $p$ is a prime not dividing the discriminant\/ $D(f)$. Let\/ $\bar f$ denote\/ $f\pmod p$. Then the Galois group\/ $G_{\bar f}$ of\/ $\bar f$ over\/ $\F_p$ is isomorphic to a subgroup of the Galois group\/ $G_f$ of\/ $f$ over\/~$\Q$.
\end{thm}

\begin{proof} {[MO, \citeauthor{386280}]} Put $K=\Q(\alpha_1,\dots,\alpha_n)$, where $\set{\alpha_1,\dots,\alpha_n}$ is the set of complex roots of~$f$. Let $A=\Z[\alpha_1,\dots,\alpha_n]$. Since the elements of $G_f$ permute $\set{\alpha_1,\dots,\alpha_n}$, they induce automorphisms in the $\Z$-algebra~$A$, i.e., $G_f\subseteq\Aut_\Z(A)$.

    Note that $p$ is not a unit in $A$. Otherwise, since the elements of~$A$ are integral over $\Z$, $1/p$ would also be integral over $\Z$. But $h(1/p)=0$ for $h\in\Z[x]$ monic, implies that $p$ is a unit in $\Z$. Indeed, if $h(x)=x^k+a_{k-1}x^{k-1}+\cdots+a_1x+a_0$, then 
    $$
        h(1/p)=0\implies (a_{k-1}+\cdots+a_0p^{k-1})p=-1,
    $$
    which is impossible.

    Fix a maximal ideal $\mathfrak m$ of $A$ including $pA$. We will use bars to denote classes modulo $\mathfrak m$. Then $\mathfrak m\cap\Z=p\Z$, and $L=A/\mathfrak m$ is a finite extension of $\F_p$ generated by $\set{\bar\alpha_1,\dots,\bar\alpha_n}$, where $\bar\alpha_i$ is the class of $\alpha_i$ in $A/\mathfrak m$. Note moreover that $\bar\alpha_i\ne\bar\alpha_j$ because $\alpha_i-\alpha_j\in\mathfrak m$ would imply $D(f)\in\mathfrak m\cap\Z=p\Z$, with doesn't happen by hypothesis.

    Since $L=\F_p(\bar\alpha_1,\dots,\bar\alpha_n)$, we see that $L$ is a splitting field of $\bar f$. In consequence, $G_{\bar f}=\Gal(L/\F_p)$.

    Consider the action of $G_f$ on the set $\mathcal M$ of maximal ideals of $A$ given by 
    $$
        \sigma\cdot\mathfrak n=\sigma(\mathfrak n)
    $$
    for $\mathfrak n\in\mathcal M$. Recall that the orbit of $\mathfrak m$ under this action is
    $$
        \mathcal O_{\mathfrak m}
            =\set{\sigma(\mathfrak m)\mid\sigma\in G_f}.
    $$
    For every $\mathfrak n\in\mathcal O_{\mathfrak m}$ chose one element $\tau_{\mathfrak n}$ such that
    $\tau_{\mathfrak n}(\mathfrak m)=\mathfrak n$ and put
    $$
        T = \set{\tau_{\mathfrak n}\mid
            \mathfrak n\in\mathcal O_{\mathfrak m}}.
    $$
    If $\tau,\tau'\in T$ then
    \begin{equation}\label{eq:T-uniqueness}
        \tau(\mathfrak m)=\tau'(\mathfrak m)\implies\tau=\tau'.
    \end{equation}
    Let $D$ be the stabilizer of the action of $G_f$ on $\mathcal M$, i.e.,
    $$
        D = \set{\sigma\in G_f
            \mid \sigma(\mathfrak m)=\mathfrak m}.
    $$
    Then $D$ is a subgroup of $G_f$ that acts on~$A/\mathfrak m=L$ by
    $$
        \sigma\cdot\bar a=\sigma(a)\pmod{\mathfrak m},
    $$
    which coincides with $\bar\sigma(\bar a)$, where $\bar\sigma\colon L\to L$ is the automorphism induced by~$\sigma$. We claim that this action is faithful, i.e., it satisfies
    \begin{equation}\label{eq:faithful-bar}
        \sigma\cdot\bar a=\bar a\text{ for all }\bar a\in L
            \implies\sigma=\id_{G_f}.
    \end{equation}
    To see this observe that $\sigma(\alpha_i)-\alpha_i\in\mathfrak m$ only if $\sigma(\alpha_i)=\alpha_i$ because $\sigma$ permutes $\set{\alpha_1,\dots,\alpha_n}$ and we have already seen that $\bar\alpha_i\ne\bar\alpha_j$ when $i\ne j$.

    Let's define the ideal $I$ of $A$ as
    $$
        I
            = \bigcap\mathcal O_{\mathfrak m}
            = \bigcap_{\tau\in T}\tau(\mathfrak m).
    $$
    According to \eqref{eq:T-uniqueness} $I$ is the intersection of maximal ideals that are pairwise coprime. By the Chinese remainder theorem, the map
    \begin{align*}
        A/I \to \bigtimes_{\tau\in T}A/\tau(\mathfrak m)
    \end{align*}
    is an isomorphism. Therefore, Since $\mathfrak m$ is one of the $\tau(\mathfrak m)$ and $A/\mathfrak m=L$, we can pick $\beta\in A$ such that $\bar\beta$ is a generator of the cyclic group $L^*$, and $\beta\in\tau(\mathfrak m)$ for all $\tau\in T$ such that $\tau(\mathfrak m)\ne\mathfrak m$. Since the morphisms in $T$ produce all possible elements of $\mathcal M$, it follows that
    \begin{equation}\label{eq:membership-to-m}
        \sigma\in G_f\setminus D
            \implies\sigma^{-1}(\beta)\in\mathfrak m.
    \end{equation}
    Consider the polynomial of $A[x]$
    $$
        g(x) = \prod_{\sigma\in G_f}x-\sigma^{-1}(\beta).
    $$
    Since $\sigma(g)=g$ for all $\sigma\in G_f$, it follows that $g\in\Q[x]$. Therefore, the coefficients of $g$ belong in $A\cap\Q$. But $A\cap\Q=\Z$ because $A$ is integral over $\Z$ [cf.~Corollary~\ref{cor:integral-ring}]. Therefore, $g(x)\in\Z[x]$ and $\bar g(x)\in\F_p[x]$. Moreover, by \eqref{eq:membership-to-m}
    $$
        \bar g(x)
        = \prod_{\sigma\in G_f}
            x-\bar\sigma^{-1}(\bar\beta)
        = x^{|G_f\setminus D|}
            \prod_{\sigma\in D}x-\bar\sigma(\bar\beta).
    $$    
    Since $g(\beta)=0$, we see that $\bar g(\bar\beta)=0$. Thus, the minimal $\pmin_{\F_p,\bar\beta}\mid\bar g$. It follows that
    $$
        \pmin_{\F_p,\bar\beta}(x)
        \mid \prod_{\sigma\in D}x-\bar\sigma(\bar\beta).
    $$
    Now, if $\zeta\in G_{\bar f}$, then $\zeta(\bar\beta)$ is also a root of $\pmin_{\F_p,\bar\beta}$. In consequence, $\zeta(\bar\beta)=\bar\sigma(\bar\beta)$ for some $\sigma\in D$. And since $\bar\beta$ is a generator of the multiplicative group $L^*$, we deduce that $\zeta=\bar\sigma$ on~$L$.
    
    By \eqref{eq:faithful-bar}, we can now define a map $\phi\colon G_{\bar f}\to D$ by $\phi(\zeta)=\sigma$ if, and only if, $\zeta=\bar\sigma$. Clearly, $\phi$ is a morphism of groups. It is also a monomorphism because $\phi(\zeta)=\id_D=\id_{G_f}$ if, and only if, $\zeta=\bar\id_{G_f}=\id_{G_{\bar f}}$.
\end{proof}

\section{Problems}

\begin{probl}
    A \textsl{transitive} subgroup of\/ $S_n$ is a subgroup\/ $G$ such that for each\/ $i, j \in \set{1, \dots, n}$, there is a\/ $\sigma \in G$ with\/ $\sigma(i) = j$. If\/ $K$ is the splitting field over\/ $\kappa$ of a separable irreducible polynomial in\/ $\kappa[x]$ of degree\/ $n$, show that\/ $|\Gal(K/\kappa)|$ is divisible by\/ $n$ and that\/ $\Gal(K/\kappa)$ is isomorphic to a transitive subgroup of\/ $S_n$. Conclude that\/ $[K : \kappa]$ divides\/ $n!$.
\end{probl}

\begin{solution}
    Firstly note that $K/\kappa$ is Galois because it is separable and normal. If $\alpha$ is a root of the given polynomial $f(x)\in\kappa[x]$, then $[\kappa(\alpha):\kappa]=n$ and so $n\mid[K:\kappa]=\Gal(K/\kappa)|$. Since every element of $\Gal(K/\kappa)$ defines a permutation of the roots of $f$, there is a natural monomorphism of groups from $\Gal(K/\kappa)$ to $S_n$. Moreover, if $\alpha_i$ and $\alpha_j$ are two roots of $f$ in $K$, then the $\kappa$-morphism $\sigma\colon\kappa(\alpha_i)\to\kappa(\alpha_j)$ that maps $\alpha_i$ to $\alpha_j$, can be extended to an automorphism of $K$, which shows that the image of $\Gal(K/\kappa)$ in $S_n$ is transitive. 
\end{solution}

%\newpage
\begin{probl}
    Write down all the transitive subgroups of\/ $S_3$ and\/ $S_4$.
\end{probl}

\begin{solution}
    As we have seen in Example~\ref{xmpls:galois-thm}~a), $S_3=\set{1,\tau,\sigma,\sigma^2,\sigma\tau}$ and its subgroups are
    $$
        \langle 1 \rangle
                ,\quad \langle \tau \rangle
                ,\quad \langle \sigma \rangle
                ,\quad \langle \tau\sigma \rangle
                ,\quad \langle \sigma\tau \rangle
                ,\quad S_3,
    $$
    with, say, $\tau=(1,2)$ and $\sigma=(1,2,3)$. Clearly $S_3$ is transitive, while $\langle1\rangle$ and $\langle\tau\rangle$ are not. Moreover, $\langle\sigma\rangle$ is transitive because $\sigma^2=(1,3,2)$. Finally, $\tau\sigma=(2,3)$ and $\sigma\tau=(1,3)$ do not generate transitive subgroups.

    In the case of $S_4$, if $H$ is a subgroup, let's consider the action $\sigma\cdot x=\sigma(x)$ for $\sigma\in H$ and $x\in\set{1,2,3,4}$. The Fundamental Counting Principle \citep{LC} establishes the equation $|\mathcal O_x|=|H:H_x|$, where $\mathcal O_x$ is the orbit of $x$ and $H_x$ its stabilizer, namely $\set{\sigma\in H\mid \sigma(x)=x}$. Thus, for $H$ transitive, we have $4=|H|/|H_x|$, which implies that $4\mid |H|$. Thus, given that $|H|\mid24$, we get $|H|\in\set{4, 8, 12}$, or $H=S_4$.

    \begin{description}
        \item[Case $|H|=4$.] Here $H_x=\set\id$ for all $x$, i.e., $H$ consists of the identity and permutations of type $(a,b)(c,d)$ and $(a,b,c,d)$. Since $\ord(a,b,c,d)=4$, $H$ is generated by one of these or $H$ consists of permutations of the first type. In the former case the possibilities are $\langle(1,2,3,4)\rangle$, $\langle(1,3,2,4)\rangle$, $\langle(1,4,2,3)\rangle$. Indeed. There are $(4-1)!=6$ cyclic permutations and each of the ones in our list generates $2$, namely, $\sigma$ and $\sigma^3$ (note that $\sigma^2$ has order $2$ and $\sigma^4=\id$).
        
        In the latter case, where the elements of $H$ have order $2$ at most, we have $H=\set{\id, (a,b)(c,d),(a,c)(b,d),(a,d)(b,c)}$, and the only possibility is $H=\set{\id, (1,2)(3,4),(1,3)(2,4),(1,4)(2,3)}$.

        \item[Case $|H|=8$.] Here $|H_x|=2$ for all $x\in\set{1,2,3,4}$. Since $24=3\cdot 8$, $H$ is a Sylow $2$-subgroup of $S_4$ and the number $n_2(S_4)$ of Sylow $2$-subgroups is $1$ or $3$. Consider the group $H=\langle(1,2,3,4),(1,3)\rangle$, which is clearly transitive. It has $8$ elements,
        \small
        $$
            H=\set{\id,(1,2,3,4),(1,3)(2,4),(1,4,3,2),(1,3)
                ,(1,4)(2,3),(1,2)(3,4),(2,4)}.
        $$
        \normalsize
        Since $H$ is not normal because
        \small
        $$
            (1,2)(1,2,3,4)(1,2) = (1,3,4,2),
        $$
        \normalsize
        we deduce that $n_2(S_4)=3$. The other two Sylow $2$-subgroups are conjugates of $H$, namely
        \small
        $$
            H^{(1,2)}=\langle(1,3,4,2),(2,3)\rangle\quad\text{and}\quad
            H^{(1,4)}=\langle(1,4,2,3),(3,4)\rangle.
        $$
        \normalsize
        
        \item[Case $|H|=12$.] The subgroup $A_4$ of even permutations is transitive because, according to the first case, $A_4$ includes a transitive subgroup whose nontrivial elements are of type $(a,b)(c,d)$. Note also that $A_4$ doesn't contain elements of order $4$ because $(a,b,c,d)=(a,b)(b,c)(c,d)$, which is odd. It does, however, include all cycles of order $3$ because $(a,b,c)=(a,b)(b,c)$. Thus
        \small
        \begin{align*}
            A_4 &= \set{\id, (1,2)(3,4), (1,3)(2,4), (1,4)(2,3),\\
                &\qquad (1,2,3), (1,3,2),(1,2,4),(1,4,2),\\
                &\qquad (2,3,4),(2,4,3),(1,3,4),(1,4,3)}.
        \end{align*}
        \normalsize
        Let $N$ be another subgroup of $S_4$ of order $12$. Then,
        \small
        $$
            24 = |A_4N| = \frac{12\cdot12}{|A_4\cap N|},
        $$
        \normalsize
        i.e., $|A_4\cap N|=6$. Since $A_4$ includes no transposition, $A_4\cap N$ must include some element $\sigma$ of order $3$ and some $\omega$ of order $2$. Therefore,
        \small
        $$
            A_4\cap N=\set{\id, \sigma,\sigma^2,\omega,\sigma\omega,\omega\sigma}
            \cong S_3.
        $$
        In particular, 
        \small
        $$
            \set{\omega, \sigma\omega, \omega\sigma}
                =\set{(1,2)(3,4), (1,3)(2,4), (1,4)(2,3)}
        $$
        \normalsize
        because these are the three involutions of $A_4$.

        After reindexing, we may assume that $\sigma=(1,2,3)$ and $\omega=(a,b)(c,d)$ (with different elements). Given that $|S_4:N|=2$, we know that $N$ is normal. In particular, the conjugacy classes of $\sigma$ are in $N$. Thus,
        \small
        $$
            \begin{array}{c|c|c|c|c|c|c}
            \tau
                &(1,2) &(1,3) &(1,4) &(2,3) &(2,4) &(3,4)\\
            \hline
            \vphantom{\Big|}\sigma^\tau
                &(1,3,2) &(1,3,2) &(2,3,4) &(1,3,2) &(1,4,3)&(1,2,4)
            \end{array}
        $$
        \normalsize
        which implies that $\set{\id,(1,2,3),(1,3,2),(2,3,4),(1,4,3),(1,2,4)}\subseteq N$. Adding the inverses, we also have, $\set{(2,4,3),(1,3,4),(1,4,2)}\subseteq N$, i.e., the~$8$ elements of order~$3$ of $A_4$ are in $N$. Since $N$ also contains the~$4$ involutions, we conclude that $N=A_4$.
    \end{description}
\end{solution}

\begin{probl}
    Determine all the transitive subgroups\/ $G$ of\/ $S_5$ for which\/ $|G|$ is a multiple of\/ $5$. For each transitive subgroup, find a field\/ $F$ and an irreducible polynomial of degree\/ $5$ over\/ $F$ such that if\/ $K$ is the splitting field of\/ $f$ over\/ $F$, then\/ $\Gal(K/F)$ is isomorphic to the given subgroup. 
    
    \textrm{\rm Hint: This will require use of semidirect products.}
\end{probl}

\begin{solution} Refer to \citep{LC} for further details.

    Let $H$ be a transitive subgroup of $S_5$. First recall that the order of $H$ is multiple of $5$ by the Fundamental Counting Principle. By Cauchy's Theorem, $H$~includes some element $\sigma$ of order $5$. Therefore, after reordering $\set{1,2,3,4,5}$, we may assume that $\sigma=(1,2,3,4,5)\in H$.
    
    Note that $\langle\sigma\rangle$ is transitive because $\sigma^i(1)=i+1$ for $0\le i<5$. Of course, $\langle\sigma\rangle\cong C_5$, the cyclic group of order $5$.

    Therefore, any transitive subgroup must include $\langle\sigma\rangle$. On the other end, we have $S_5$ itself. Any other candidate must satisfy
    $$
        |H|/5\in\set{2,3,4,2\cdot3,2\cdot4,3\cdot4},
    $$
    i.e.,
    $$
        |H| \in \set{10,\cancel{15},20,\cancel{30},\cancel{40},60}.
    $$
    \begin{description}
        \item[Case $|H|=10$.] Here $H$ includes an involution $\zeta$. More precisely, $H=\langle\sigma,\zeta\rangle$. In particular, $H\cong D_{10}$, the dihedral group of $10$ elements.
        
        Regarding $\zeta$, we claim that it cannot be a transposition. Suppose otherwise. After rotating the $5$-cycle and the transposition independently as needed, we reduce ourselves to the case where $\zeta=(1,a)$. Then the following table
        \begin{equation}\label{table:S5}
            \begin{array}{c|c|c}
                \text{product}&\text{result}&\text{order}\\
                \hline
                \vphantom{|^{|^|}}
                (1,2,3,4,5)(1,2)&(1,3,4,5)&4\\
                (1,2,3,4,5)(1,3)&(1,4,6)(2,3)&6\\
                (1,2,3,4,5)(1,4)&(1,5),(2,3,4)&6\\
                (1,2,3,4,5)(1,5)&(2,3,4,5)&4
            \end{array}
        \end{equation}
        shows that none of the products belongs to $H$ because $4,6\nmid10$. On the other hand $(1,2,3,4,5)(1,2)(3,5)=(1,3)(4,5)$, which is an involution and satisfies $(1,2)(3,5)(1,3)(4,5)=(5,4,3,2,1)$, i.e., $\sigma^\zeta=\sigma^{-1}$, as expected in $D_{10}$. In particular, $H\subseteq A_5$.

        \item[Case $\cancel{|H|=15}$.] This case is ruled out: Let $n_3=n_3(H)$ be the number of Sylow $3$-subgroups of $H$. Then $n_3\equiv1\pmod3$ and $n_3=|H:N_H(P_3)|$ for any $P_3\in\Syl_3(H)$. Since the only possibilities for $N_H(P_3)$ are $P_3$ and $H$, we deduce that $n_3\ne5$ and so $n_3=1$. In other words, there is only one $3$-subgroup $P_3$. A similar argument shows that there is only one $5$-subgroup $P_5$. Since $P_3\cap P_5=\grp1$, we deduce that there are $|P_3\cup P_5|=3+5-1=7$ elements of order less than $15$. Thus, the remaining $8$ elements must have order $15$. However, there is no permutation of order $15$ because the order of a permutation is the $\lcm$ of the lengths of the cycles occurring in its cyclic decomposition (its \textsl{cycle type}).

        \item[Case $|H|=20$.] [MSE, \citeyear{2065244}] Using the preceding notation write $n_5=n_5(H)$. Then $n_5\equiv1\pmod5$ and $n_5=|H:N_H(P_5)|\in\set{4,2,1}$. It follows that $n_5=1$ and there is only one $P_5\in\Syl_5(H)$, which is normal. If $Q\in\Syl_2(H)$ then $Q$ is abelian of order $4$. Since every $4$-cycle has order~$4$ it must be contained in some $4$-Sylow subgroup. And given that these subgroups are conjugate to each other, every one of them has some $4$-cycle, i.e., they are all isomorphic to $C_4$. It follows that $H=P_5\rtimes Q$, i.e., $H=C_5\rtimes C_4$.

        \item[Case $\cancel{|H|=30}$.] This case is ruled out: Consider the following

        \textbf{Theorem.} \textit{Suppose that\/ $|G|=2n$, where\ $n$ is odd. Then\/ $G$ has a normal subgroup of index $2$.}

        Since we can apply this to $H$, we deduce the existence of a (normal) subgroup of order $15$. However, as we shown in the Case~$|H|=15$, $S_5$ has no subgroup of order $15$.

        \item[Case $\cancel{|H|=40}$.] [MSE, \citeauthor{89048}] Firstly observe that $H\nsubseteq A_5$ because $40\nmid60$. Pick $\omega\in H\setminus A_5$. The map
        \begin{align*}
            H\setminus A_5\colon&\to H\cap A_5\\
            \zeta&\mapsto\omega\zeta
        \end{align*}
        is a bijection. In consequence, $K=H\cap A_5$ has order $20$. Since every subgroup of index~$2$ is normal, we deduce that $K\normal A_5$. But this is impossible because $A_5$ is known to be simple.

        \item[Case $|H|=60$.] It is well-known that the only subgroup of index $2$ is $A_5$.
    \end{description}
    Let's now consider the second part of the problem. As before, we will proceed following the degree of the subgroup.
    \begin{description}
        \item[Case $|H|=5$.] From Problems~\ref{probl:Fp(u)} and~\ref{probl:k(x):k(f)} we know that, for
        $$
            u = \prod_{i=0}^4x-i\in\F_5[x],
        $$
        the minimal of $x$ is $\pmin_{\F_5(u),x}(t)=u-\prod_{i=0}^4t-i=u(x)-u(t)$. Moreover, since the other roots are $x+1,\dots,x+4$, we see that $K=\F_5(u)(x)$ is Galois over $\F_5(u)$ of degree~$5$. Considering that the only group of order~$5$ is $C_5$, we see that $\Gal(K/\F_5(u))\cong H$, where the isomorphism is given by mapping $\varphi$ to $\id_\sigma$, with $\sigma$ the generator of $H$ and $\varphi(x)=x+1$, which is an automorphism of order~$5$.

        \item[Case $|H|=10$.]

        \item[Case $|H|=20$.] Let $f(x)=x^5-2\in\Q[x]$. This polynomial is irreducible by Eisenstein's criterion. If $\omega=e^{2\pi i/5}$, then the roots of $f$ are $\sqrt[5]2$ and the products $\sqrt[5]2\omega^k$ for $1\le k<4$, and $f$ splits over $K=\Q(\omega,\sqrt[5]2)$. Let $F$ be the splitting field of $f$. Then $F\subseteq K$. Since $\sqrt[5]2$ and $\sqrt[5]2\omega$ are in $F$, we see that $F=K$. To see that $[K:\Q]=20$, first observe that $20\mid[K:\Q]$ because $\pmin_{\Q,\omega}(x)=x^4+x^3+x^2+x+1$ [cf.~Problem~\ref{probl:x^p-1/(x-1)-irred}] and $\pmin_{\Q,\sqrt[5]2}(x)=x^5-2$. Second, $x^5-2$ is irreducible over $\Q(\omega)[x]$ because
        $$
            x^5-2 = (x-\sqrt[5]2)(x-\sqrt[5]2\omega)(x-\sqrt[5]2\omega^2)
                (x-\sqrt[5]2\omega^3)(x-\sqrt[5]2\omega^4)
        $$
        cannot be decomposed as a product of two nonconstant polynomials with coefficients in~$\Q(\omega)$, as some of those coefficients would be a multiple of $\sqrt[5]2$ in that field.
        
        \item[Case $H=A_5$.] Consider the polynomial in $\Z[x]$
        $$
            f(x) = x^5 + 20x + 16.
        $$
        Online calculators give us $D(f)=2^{16}5^6$. Since $D(f)$ is a square, $H\subseteq A_5$.
        
        Reduction modulo $3$ gives us
        $$
            \bar f(x) = x^5+2x+1\text{ in }\F_3[x].
        $$
        We claim that $\bar f$ is irreducible. Suppose otherwise. Then
        \begin{align*}
            x^5+2x+1 &= (x^2+ax+b)(x^3+cx^2+dx+b^{-1})\\
                &= x^5+(a+c)x^4+(ac+b+d)x^3\\
                &\quad+(ad+bc+b^{-1})x^2+(ab^{-1}+bd)x+1.
        \end{align*}
        Therefore,
        \begin{align}
            a^2 &= b+d\label{eq:a}\\
            b^{-1} &= a(b-d)\label{eq:b}\\
            1 &= ab^{-1}+db\label{eq:c}
        \end{align}
        From \eqref{eq:b}, $a,b,b-d\ne0$. By \eqref{eq:a} $b+d=1$. Hence, $d=0$ and $b=1$. From~\eqref{eq:c} and~\eqref{eq:a}, $a=1$, $c=-1$. Thus, the only possibility is
        $$
            x^5+2x+1 = (x^2+x+1)(x^3-x^2+1),
        $$
        which doesn't hold because the RHS is $x^5+x+1$. Then, the claim is true, and $f$ is also irreducible.

        Reduction modulo $7$ gives
        $$
            \bar f(x) = x^5-x+2.
        $$
        Computing mod $7$ we have
        $$
            \begin{array}{l|rrrrrr}
                 x& 1&2&3&-3&-2&-1\\
                 x^2&1&-3&2&2&-3&1\\
                 x^4&1&2&-3&-3&2&1\\
                 x^5&1&-3&-2&2&3&-1\\
                 x^5-x&0&2&2&-2&-2&0
            \end{array}
        $$
        which shows that $(x+3)(x+2)\mid\bar f$. Therefore, $\bar f(x)=(x+3)(x+2)\bar q(x)$ and we deduce that $\bar q(x)$ is irreducible of degree~$3$. In consequence, $3\mid|G_{\bar q}|$ and given that $G_{\bar q}\subseteq G_{\bar f}$, we see that $3\mid|G_{\bar f}|$. Now, Theorem~\ref{thm:galois-mod-p} implies that $|G_{\bar f}|\mid |G_f|$, and so $3\mid|H|$.

        In sum, $15\mid |H|\mid60$, and since $|H|\in\set{10,20,60}$, the only possibility is $|H|=60$, i.e., $H=A_5$.
        
        \item[Case $H=S_5$.] Let $f\in\Q[x]$ be an irreducible polynomial of degree $5$ with $3$ real roots and $2$ complex (non real). Then $K=\Q[x]/\gen{f(x)}$ is Galois over~$\Q$ of degree~$5$. For instance, $f(x)=x^5-10x^3+10x^2-10$ is such a polynomial because
        \begin{enumerate}[a), font=\upshape]
            \item is irreducible by Eisenstein's criterion,
            \item $f'(x)=5(x-(-\sqrt3-1))(x-0)(x-(\sqrt3-1))(x-2)$,
            \item the local max at $x=\sqrt3-1$ is negative.
        \end{enumerate}
        If $\tau\colon K\to K$ is included in the complex conjugation map, then $\tau$ is a transposition in $G=\Gal(K/\Q)$ that swaps the two complex roots of $f$. As before, let $\sigma$ denote the automorphism corresponding to a $5$-cycle. After rotating $\sigma$ and $\tau$ independently, we may assume that $\sigma=1,2,3,4,5)$ and $\tau=(1,a)$. Consider Table~\ref{table:S5}. Taking squares in the cases $a=3$ and $a=4$, we obtain a $3$-cycle. In the cases $a=2$ and $a=5$, $\sigma^2\tau$ has cycle type $2-3$. Thus, their squares are $3$-cycles. This means that in every instance $G$ will include a $3$-cycle. Since, as we have shown in the first part, $|G|\in\set{5,10,20,60,120}$ and $\tau\in G$, we see that the only possibility is $|G|=120$, i.e., $G=S_5$.
    \end{description}
\end{solution}

\begin{probl}
    In the following problems, let\/ $K$ be the splitting field of\/ $f(x)$ over\/ $F$. 
    Determine\/ $\Gal(K/F)$ and find all the intermediate subfields of\/ $K/F$.
    \begin{enumerate}[a), font=\upshape]
        \item $F = \Q$ and\/ $f(x) = x^4 - 7$.
        \item $F = \F_5$ and\/ $f(x) = x^4 - 7$.
        \item $F = \Q$ and\/ $f(x) = x^5 - 2$.
        \item $F = \F_2$ and\/ $f(x) = x^6 + 1$.
        \item $F = \Q$ and\/ $f(x) = x^8 - 1$.
    \end{enumerate}
\end{probl}

\begin{solution}${}$
    \begin{enumerate}[a), font=\upshape]
        \item In this case $f(x)=(x-\sqrt[4]7)(x+\sqrt[4]7)
        (x-\sqrt[4]7i)(x+\sqrt[4]7i)$. Therefore, $K=\Q(i,\sqrt[4]7)$ and we have the following field extensions
        $$
            \begin{tikzcd}[arrows={-},column sep=-0.4cm]
                    &{\Q(i,\sqrt[4]7)}\\
                {\Q(\sqrt[4]7)}
                        \arrow[ru,"2"]
                    &&\Q(i)
                        \arrow[lu,"4"']\\
                    &\Q
                        \arrow[ru,"2"']
                        \arrow[lu,"4"]
            \end{tikzcd}
        $$
        Thus, $[K:F]=8$. Complex conjugation corresponds to a transposition $\tau\in S_4$. Moreover, $\mathcal F(\grp\tau)=\Q(\sqrt[4]7)$.
        
        The Klein group $V_4= \set{\id,(1,2)(3,4),(1,3)(2,4),(1,4)(2,3)}$ is characteristic because automorphisms of $S_4$ preserve cycle structure. Here we have $\mathcal F(V_4)=\Q(i)$.

        \item The table
        $$
            \begin{array}{l|rrrr}
                 x&1&2&-2&-1\\
                 x^2&1&-1&-1&1\\
                 x^4&1&1&1&1
            \end{array}
        $$
        shows that $f(x)=x^4-2$ has no roots in $F$. However, $2^2=(-2)^2=-1$ in $F$. Therefore, is $\alpha$ is a root of $f(x)$, $-\alpha$, $2\alpha$ and $-2\alpha$ are the other roots, i.e.,
        $$
            f(x) = (x-\alpha)(x+\alpha)(x-2\alpha)(x+2\alpha).
        $$
        It follows that $K=F(\alpha)$. Hence, $|\Gal(K/F)|=4$, and since the subgroup is transitive and normal, we must have $\Gal(K/F)=V_4$.

        \item Here $f(x)=(x-\sqrt[5]2)(x-\sqrt[5]2\omega)(x-\sqrt[5]2\omega^2)(x-\sqrt[5]2\omega^3)(x-\sqrt[5]2\omega^4)$, where $\omega=e^{2\pi i/5}$. Therefore, $K=\Q(\sqrt[5]2,\omega)$. In a diagram,
        $$
            \begin{tikzcd}[arrows={-},column sep=-0.4cm]
                    &{\Q(\omega,\sqrt[5]2)}\\
                {\Q(\sqrt[5]2)}
                        \arrow[ru,"4"]
                    &&\Q(\omega)
                        \arrow[lu,"5"']\\
                    &\Q
                        \arrow[ru,"4"']
                        \arrow[lu,"5"]
            \end{tikzcd}
        $$
        where $\pmin_{\Q,\omega}(x)=x^4+x^3+x^2+x+1$ [Problem~\ref{probl:x^p-1/(x-1)-irred}]. As we have seen, $\Gal(K/\Q)=C_5\rtimes C_4$. 

        Put $\alpha=\sqrt[5]2$. If $C_5=\grp\sigma$ and $C_4=\grp\zeta$, with $\sigma(\alpha\omega^i)=\alpha\omega^{i+1}$ (corresponding to the $5$-cycle $(1,2,3,4,5)$) and $\zeta(\omega)=\omega^2$, we have,
        \begin{align*}
            \sigma^i\zeta^j(\alpha) &= \alpha\omega^i\\
            \sigma^i\zeta^j(\omega) &= \omega^{2^j}\\
            \sigma^i\zeta^j(\alpha^h\omega^k)
                &= \alpha^h\omega^{ih+2^jk}.
        \end{align*}
        Now, the subgroup of $G$ given by $\grp\sigma\rtimes\grp{\zeta^2}$ has index~$2$ and it is therefore normal \citep{LC}. Since $\omega^4=\omega^{-1}$, its elements satisfy
        $$
            \sigma^i(\zeta^2)^j(\alpha^h\omega^k)
                = \alpha^h\omega^{ih-jk}.
        $$
        Note also that $\grp{\zeta^2}$ is normal because it has order~$2$. In particular, if we restrict ourselves to finding the fixed field $\mathcal F(\grp\zeta^2)$ inside $\Q(\omega)$, we have 
        $$
            (\zeta^2)^j(\omega^k)=\omega^{4^jk}.
        $$
        Therefore, expressing $\omega^3$ in terms of $\omega^{-1}$, $\omega^2$, $\omega$ and $1$, observe that
        $$
            (\zeta^2)^j(a\omega+b\omega^2+c\omega^{-1})
                = a\omega^{4^j}+b\omega^{4^j2}+c\omega^{-4^j}.
        $$
        For $j=1$: $a\omega^{-1}+b\omega^3+c\omega$. For $j=2$: $a\omega+b\omega^2+c\omega^{-1}$. Hence, these are fixed precisely when $b=0$ and $a=c$, i.e., when the expression belongs to $\Q(\omega+\omega^{-1})$. Since $\omega^{-1}=\bar\omega$, $\omega+\omega^{-1}\in\R$.
        
        Finally, to find the fixed field of $\grp\sigma\rtimes\grp{\zeta^2}$ in $\Q(\alpha,\omega)$ observe that
        $$
            \sigma(\alpha\omega)=\alpha\omega,\quad
            \sigma(\omega^k)=\omega^k
            \quad\text{and}\quad
            \zeta(\alpha)=\alpha.
        $$
        Therefore, $\Q(\alpha\omega,\omega+\omega^{-1})$ is fixed by $\grp\sigma\rtimes\grp{\zeta^2}$. \qedhere
    \end{enumerate}
\end{solution}

\begin{probl}
    Let\/ $\kappa\subseteq L\subseteq K$ be a tower of finite field extensions and\/ $F$ an algebraic closure of\/ $\kappa$. In\/ $\Hom_\kappa(K,F)$ define the equivalence relation
    $$
        \sigma\sim\varsigma\iff\sigma|_L=\varsigma|_L.
    $$
    Suppose that $L/\kappa$ is separable. Then\/ $|\Hom_\kappa(K,F)/{\sim}|=[L:\kappa]$ and every class of the quotient has\/ $[K:L]$ elements.
\end{probl}

\begin{solution}
    \begin{description}
        \item [{Case $L=\kappa[\alpha]$:}] Let $f=\min_{\kappa,\alpha}$. Since $\alpha$ is separable, $f$ has $n=\deg(f)$ roots $\set{\alpha_1,\dots,\alpha_n}$ in $F$. If $\sigma\colon K\to F$ is a $\kappa$-morphism, then $\sigma(\alpha)$ is a root of $\sigma(f)=f$ and so the map
        \begin{align*}
            \ev\alpha\colon
                \Hom_\kappa(K,F)
                    &\to\set{\alpha_1,\dots,\alpha_n}\\
            \sigma&\mapsto\sigma(\alpha)
        \end{align*}
        is well-defined. To see that it is surjective, first note that, by Lemma~\ref{lem:algebraic-extension-1}], there is a $\kappa$-morphism from $\kappa[\alpha]\to\kappa[\alpha_i]$ satisfying $\alpha\mapsto\alpha_i$. Following this morphism with the inclusion $\kappa[\alpha_i]\hookrightarrow F$, we get a morphism $\sigma\colon L\to F$ such that $\ev\alpha=\alpha_i$. By Corollary~\ref{cor:interior-morphism-extension}, we can extend $\sigma$ to an automorphism $\omega\colon F\to F$. Finally, the restriction $\omega|_K\colon K\to F$ satisfies $\ev\alpha=\alpha_i$.

        This proves the equation $|\Hom_\kappa(K,F)/{\sim}|=n=[L:\kappa]$.

        To complete the proof we have to show that, given $\sigma\colon L\to F$ there are exactly $[K:L]$ extensions of $\sigma$ to $K$.

        The proof works by induction on $d=[K:L]$. If $d=1$, then $K=L$ and the conclusion is trivial. If $d>1$, we can pick $\beta\in K\setminus L$. Let $g=\min_{L,\beta}$. Then $1<\deg g\le d$. Take an extension $\varsigma$ of $\sigma$ to $L(\beta)$. It follows that $\varsigma(\beta)$ is a root of $g$.
    \end{description}
\end{solution}